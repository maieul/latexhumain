% Reprise de doc.sty et ltxdockit.sty 
% Commandes utiles pour documenter :
% \cs 	-> indique une commande, et l'index
% \csp	-> même chose que précédente, mais marque la page en gras dans l'index.
% \marg	-> pour un argument obligatoire {<arg>}
% \oarg	-> pour un argument optionnel [<arg>]
% \arg	-> pour un argument cité dans le court du texte : <arg>
% \contenuarg -> pour un exemple d'argument. titi
\newlength{\marglistwidth}
\deflength{\marglistwidth}{(\oddsidemargin+1in)*85/100}%
\newcommand*{\marglistfont}{\bfseries}
\newlength{\marglistsep}
\deflength{\marglistsep}{10pt}
% Repris de doc.sty, ligne 636 ss.
\makeatletter
\DeclareRobustCommand\meta[1]{%
     \ensuremath\langle
     \ifmmode \expandafter \nfss@text \fi
     {%
      \meta@font@select
      \edef\meta@hyphen@restore
        {\hyphenchar\the\font\the\hyphenchar\font}%
      \hyphenchar\font\m@ne
      \language\l@nohyphenation
      #1\/%
      \meta@hyphen@restore
     }\ensuremath\rangle
}
\def\meta@font@select{\itshape}
% Fin de la reprise
\newcommand*{\verbatimfont}{\ttfamily}
\newrobustcmd*{\cmd}[1]{\mbox{\verbatimfont\textbackslash#1}}
\newrobustcmd*{\cs}[1]{\mbox{\verbatimfont\textbackslash#1}}
\let\oldcs\cs

\pretocmd{\cs}{\noindent}{}{}
\apptocmd{\cs}{\index[cs]{#1@\oldcs{#1}}}{}{}

\newcommand*{\hyperpagetextbf}[1]{\textbf{\hyperpage{#1}}}

\newcommand*{\csp}[1]{\noindent\oldcs{#1}\index[cs]{#1@\oldcs{#1}|hyperpagetextbf}}
\providecommand\marg[1]{%
  {\ttfamily\char`\{}\meta{#1}{\ttfamily\char`\}}}
\providecommand\oarg[1]{%
  {\ttfamily[}\meta{#1}{\ttfamily]}}
\providecommand\parg[1]{%
  {\ttfamily(}\meta{#1}{\ttfamily)}}

\renewcommand{\arg}[1]{%
	\meta{#1}%
}
\newcommand{\contenuarg}[1]{{\ttfamily#1}}

\let\oldtextbackslash\textbackslash

\renewcommand{\textbackslash}[0]{{\ttfamily\oldtextbackslash}}
\makeatother