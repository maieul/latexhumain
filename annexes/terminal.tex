\chapter{Introducción a la línea de comandos}\label{terminal}

\begin{intro}
La mayoría de los editores de texto especializados en LaTeX ofrece un botón para ejecutar los principales comandos de compilación:  XeLaTeX, MakeIndex, a veces Biber.

Sin embargo, pronto se vuelve necesario que tengamos que hacer más: por ejemplo, si utilizamos el script de gestión de los índices de fuentes primarias\renvoi{scriptpython}, tenemos que poder ejecutarlo.

Utilizamos para ello la terminal del sistema operativo que permite ejecutar directamente esos comando. Veamos una breve introducción a su utilización en el marco del uso de \XeLaTeX.
\end{intro}

\section{La noción de carpeta actual}\label{repcourant}

Lo que llamamos \forme{carpeta actual} se corresponde a la ubicación donde se sitúa en el árbol de archivos del ordenador. Cuando queremos utilizar la línea de comandos para usar \XeLaTeX, lo primero que hay que hacer es cambiar la carpeta actual para ir a la carpeta en la que se sitúan los archivos que queremos compilar.

\section{Mac OS X y Linux}

En Mac OS X, La terminal está en la carpeta \forme{Utilidades} de la carpeta \forme{Aplicaciones}. 

En GNU/Linux, la Terminal se encuentra en el menú \forme{Aplicación} – 
\forme{Accesorios}, o bien presionas la tecla \forme{Windows} para teclear después las primeras letras de la aplicación: \verb|term| debería bastar. Pulsa sobre el símbolo que aparece.

La carpeta actual suele estar indicada a la izquierda de la línea, donde el símbolo  \verb|~|  representa la carpeta de partida. Para ver una lista de su contenido, tecleamos \verb|ls|. Para validar un comando, hay que pulsar en la tecla \verb|Intro|.

Para desplazarnos a una carpeta, basta con pulsar el comando
\verb|cd| seguido de la carpeta a la que queremos dirigirnos.

Así, el comando \verb|cd proyeto-latex| te traslada a la carpeta \verb|proyeto-latex|, cosa que puedes verificar con el comando \verb|dir|. Para desplazarte a una carpeta superior, pulsa simplemente \verb|cd ..|

En vez de subir de carpeta en carpeta con el comando \verb|cd| hasta la carpeta actual, también se puede teclear \verb|cd| seguido directamente de la ruta completa\renvoi{chemin} de la carpeta a la que nos queremos desplazar. En Mac o en Linux, si no estamos seguros de la ruta, se puede verificar arrastrando simplemente el icono de la carpeta a la ventana de la Terminal: entonces aparece la ruta.

Cuando estás en la carpeta en la que deseas ejecutar un comando, puedes lanzarlo de la misma manera:

\begin{bashcode}
xelatex nombredelficheroparacompilar.tex
biber nombredelficheroparacompilar
makeindex nombredelficheroparacompilar
etc.
\end{bashcode}

Por último, el comando \verb|ls| te permite ver el contenido de la carpeta en la que te encuentras.

\section{Windows}
En Windows, La terminal se llama \enquote{Símbolo del sistema}. El lenguaje es diferente del que aparece en los sistemas de tipo Unix, como 
Linux y Mac OS X. Sin embargo, hay ciertas similitudes.

Para iniciar el símbolo del sistema, pulsa simultáneamente la tecla \verb|Windows| de tu teclado y la tecla \verb|R|. En la ventana que se abre escribe \verb|cmd| y después
\verb|Intro|. Ya estás en una consola.

Todo comando, una vez tecleado, se valida presionando la tecla \verb|Intro|.

La carpeta actual se indica a la izquierda del cursor que parpadea. El comando \verb|dir| te indica lo que hay en la carpeta actual.

Para desplazarte a un directorio, basta con teclear el comando
\verb|cd| seguido de la carpeta a la que te quieres dirigir.

Así, el comando \verb|cd proyeto-latex| te traslada a la carpeta \verb|proyeto-latex|, cosa que puedes verificar con el comando \verb|dir|. Para desplazarte a una carpeta superior, pulsa simplemente\verb|cd ..|

Cuando estás en la carpeta en la que deseas ejecutar un comando, puedes lanzarlo de la misma manera:

\begin{bashcode}
xelatex nombredelficheroparacompilar.tex
\end{bashcode}

Finalmente, el comando \verb|dir| muestra el contenido de la carpeta en la que estás actualmente.

\section{Encontrar los ficheros estándar}\label{trouverfichier}

Para poder personalizar el comportamiento estándar de \LaTeX o de un package, por ejemplo, los estilos de \package{biblatex}, por lo general es necesario ver los ficheros estándar para redefinir un comando u otro o una macro bibliográfica. 

Por tanto, hay que encontrar sus ubicaciones. En el fichero \ext{log} producido durante la compilación, aparece una lista con el conjunto de los ficheros cargados por el compilador. Por lo general, es fácil conocer el nombre del fichero: suele ser el del package o de la clase. Una búsqueda en el fichero permite, por tanto, encontrar la ruta, por ejemplo: 

\noindent\verb|/usr/local/texlive/2012/texmf-dist/tex/latex/biblatex/cbx/verbose-trad2.cbx|.

En general, los ficheros se encuentran en carpetas invisibles mediante la iterfaz estándar del sistema operativo, Para abrirlos, tenemos que usar la línea de comandos y teclear
\begin{description}
\item[En Mac]\verb|open ruta-al-fichero|.
\item[En Windows] \verb|rtua-al-fichero|.  No obstante, por defecto, la opción de copiar-pegar está desactivada en el símbolo del sistema. Pero se puede sortear esto copiando-pegando en la barra de dirección del navegador Firefox.
\item[En Linux] \verb|nombre_del_editor ruta-al-fichero|.
\end{description}
\begin{attention}
No hay que modificar nunca los ficheros estándar. En efecto, en caso de actualización de dichos ficheros, perderías todas tus modificaciones. Por lo general, es mejor redefinir los comandos, aunque haya que agrupar estas redefiniciones en un fichero.
\end{attention}

