\chapter{Facilitar las compilaciones con Latexmk}\label{latexmk}

\begin{intro}
Con \LaTeX es frecuente que tengamos que hacer varias compilaciones con diferentes programas en un orden determinado. El programa Latexmk permite gestionar automáticamente estas compilaciones.
\end{intro}

\section{Principio}

Latexmk es un script instalado con todas las distribuciones \LaTeX. Se utiliza mediante la línea de comandos\renvoi{terminal}.
Latexmk consulta los ficheros de compilación (\ext{log}) para saber cuáles son los scripts que hay que ejecutar y en qué orden. 
Aunque es muy fácil de usar, sin embargo, necesita algunas adaptaciones para que sea compatible con \XeLaTeX. Las presentamos aquí, pero el programa ofrece muchas más configuraciones: remitimos al manual\footcite{latexmk}.

\section[Adaptación para XeLaTeX]{Adaptación para \XeLaTeX}

Para configurar el script, necesitamos crear un fichero \fichier{latexmkrc} al lado del fichero que vamos a compilar. En este fichero, escribimos las líneas siguientes, prestando atención a los punto y coma del final de la línea:

\begin{bashcode}
$pdf_mode = "1";
$pdflatex = "xelatex";
\end{bashcode}

La primera línea significa que pedimos a Latexmk que produzca ficheros \ext{pdf} y no \ext{dvi}, que es el formato \enquote{histórico} producido por \LaTeX.

La segunda línea significa que pedimos a Latexmk que utilice \XeLaTeX y no \LaTeX para producir esos \ext{pdf}.


Dirígete con la terminal\renvoi{terminal} a la carpeta del fichero que quieres compilar. Teclea, a continuación, el comando siguiente:
\begin{bashcode}
latexmk fichero
\end{bashcode}

donde \verb|fichero| es el nombre de tu fichero principal. Entonces veremos desfilar las distintas compilaciones. Latexmk se encarga de hacer tantas compilaciones como sea necesario, teniendo en cuenta eventuales desplazamientos de etiquetas. Al final de estas compilaciones obtendrás el fichero \ext{pdf} deseado.

Sin embargo, si invocas una entrada bibliográfica sin definir\renvoi{bddbiblio} o una etiqueta de remisión interna\renvoi{label} sin definir, Latexmk solo procede a cinco series de compilaciones\footnote{Si, al cabo de cinco compilaciones, el programa no logra un resultado estable, se detendrá con un error.}. Con todo, te indica las entradas o etiquetas problemáticas con unas líneas que tiene esta forma:
\begin{bashcode}
Latexmk: Citation 'XXX' on page YYY undefined 
Latexmk: Reference 'XXX' on page YYY undefined 
\end{bashcode}

Por tanto, te resultará fácil encontrar las entradas bibliográficas (=\enquote{Citation}) y las etiquetas (=\enquote{Reference}) sin definir.

\section{Adaptación para el índice de fuentes primarias}

Si utilizas el script de gestión del índice de fuentes primarias\renvoi{python}, tendrás también que añadir la línea siguiente:
\begin{bashcode}
$makeindex="python index.py;makeindex %S";
\end{bashcode}

Esta línea indica a latexmk que ejecute el script python antes del comando makeindex, el código \verb|%S| designando el fichero en el que makeindex ha de ejecutarse.

