\chapter{Encontrar ayuda}

\begin{intro}
¿Qué podemos hacer cuando nos hemos atascado en un punto concreto y hemos releído los distintos manuales varias veces\footcite[Señalemos, de paso, la posibilidad de descargar una ayuda sobre el conjunto de los errores de compilación con \LaTeX{}:][]{erreurscompilo}? Pedir ayuda a otros usuarios de \LaTeX{}.

Veamos algunos lugares donde podemos hacerlo.
\end{intro}


\section{Foros de internet}

Los foros de Internet sobre \LaTeX son muy numerosos. En inglés, podemos utilizar el de \forme{LaTeX Community}: \url{http://www.latex-community.org/forum/}. En francés, podemos utilizar el del sitio \forme{Developpez.net} \url{http://www.developpez.net/forums/f149/autres-langages/autres-langages/latex/}.


\section{Mensajería instantánea}

Se puede pedir ayuda en diferentes salas de mensajería instantánea que funciona por IRC\footnote{Antepasado lejano de Skype, MSN y otros GoogleTalk.}. Por lo general, podemos encontrar ahí ayuda con bastante rapidez.

Para conectarse a una sala de chat IRC, podemos utilizar el plugin Chatzilla del programa libre Firefox.

En francés, la dirección es \url{irc://irc.rezosup.org/latex}; en inglés
\url{irc://irc.freenode.net/latex}.


\section{Listas de discusión}

En francés, las de la asociación Gutenberg: \url{http://www.gutenberg.eu.org/?Listes-de-diffusion-gerees-par}; en inglés, la lista siguiente: \url{http://groups.google.com/group/comp.text.tex/topics}, que se puede utilizar mediante un programa de gestión de \enquote{Newsgroups}, como, por ejemplo, Mozilla Thunderbird.
