\chapter{Bibliografía elemental}

\nocite{*}
\section{Manuales de los packages \label{manuels}}


Aquí no ofrecemos una lista de todos los manuales de los diferentes packages estudiados. Pero se puede plantear una pregunta: ¿dónde podemos encontrar esos manuales?

La solución más fácil es abrir la terminal:
\begin{itemize}
\item En Mac, abrir la aplicación \forme{Terminal} en la carpeta \forme{Utilidades}.
\item En Windows, abrir \forme{Símbolo del sistema} en \forme{Inicio}: \forme{Todos los programas}: \forme{Accesorios}.
\item En GNU/Linux, pulsar en la tecla \forme{Windows} y escribir \verb|terminal|.
\end{itemize}

Después, escribir en la ventana que aparece:

\begin{bashcode}
texdoc nombredelpackage
\end{bashcode}

Por ejemplo, para el package \package{biblatex}:

\begin{bashcode}
texdoc biblatex
\end{bashcode}

Pulsar un retorno del carro: el manual del package debería abrirse con el programa adecuado (normalmente, el manual está en formato PDF).

\section{Libros generales}

Los libros sobre \LaTeX son abundantísimos. Sin embargo, son pocos los que abordan \XeLaTeX,  \package{polyglossia} y  \package{biblatex}. Por esa razón, recomendamos utilizarlos fundamentalmente para las necesidades más avanzadas de diseño.

\printbibliography[keyword=generaliste]

\section{Libros y textos sobre cuestiones específicas}

\printbibliography[keyword=specifique]


\section{Sitios en internet}

Como en el caso de los libros generales, el interés, en cuestión de contenido, en comparación con la problemática de este libro, puede ser muy variable. 

\printbibliography[keyword=site]
