\chapter{Bibliographie succincte}

\nocite{*}
\section{Manuels des packages \label{manuels}}


Nous ne listons pas ici l'ensemble des manuels des divers package abordés. Toutefois une question peut se poser : où trouver ces manuels ?

La solution la plus simple est d'ouvrir le terminal :
\begin{itemize}
\item Sous Mac, ouvrir l'application \forme{Terminal} dans le dossier \forme{Utilitaires}.
\item Sous Windows, ouvrir \forme{l'invite des commandes} dans \forme{Démarrer} : \forme{Tous les programmes} : \forme{Accessoires}.
\item Sous GNU/Linux, appuyer sur la touche \forme{Windows} et  taper \verb|terminal|.
\end{itemize}

Puis de frapper dans la fenêtre qui apparaît :

\begin{bashcode}
texdoc nomdupackage
\end{bashcode}

Par exemple pour le package \package{biblatex}:

\begin{bashcode}
texdoc biblatex
\end{bashcode}

Frapper un retour à la ligne : le manuel du package devrait s'ouvrir avec le logiciel adéquat (généralement le manuel est au format PDF).

\section{Livres généralistes}

Les livres sur \LaTeX sont pléthores. Cependant rares sont ceux qui abordent \XeLaTeX,  \package{polyglossia} et  \package{biblatex}. C'est pourquoi nous conseillons de les utiliser essentiellement pour les besoins les plus avancés de mise en page.

\printbibliography[keyword=generaliste]

\section{Livres et textes sur des points spécifiques}

\printbibliography[keyword=specifique]


\section{Sites internet}

Comme pour les livres généralistes,  l'intérêt en terme de contenu, par rapport à la problématique de ce livre, peut être très variable. 

\printbibliography[keyword=site]
