\section{Trouver les fichiers standards}\label{trouverfichier}

Pour pouvoir personaliser le comportement standard de \LaTeX ou d'un package (par exemple les styles de \package{biblatex}, il est en général nécéssaire de regarder les fichiers standards pour redefinir telle ou telle commande (ou macro bibliographique). 

Pour ce faire, il faut trouver leurs emplacements. Dans le fichier \ext{log} produit lors de la compilation est listé l'ensemble des fichiers chargés par le compilateurs. Il est général aisé de connaître le nom du fichier : il s'agit souvent de celui du package ou de la classe. Une recherche dans le fichier permet donc de trouver le chemin, par exemple : \\
\verb|/usr/local/texlive/2010/texmf-dist/tex/latex/biblatex/cbx/verbose-trad2.cbx|.

En général les fichiers se trouvent dans des répertoires invisibles via l'interface standard du système d'exploitation. Pour les ouvrir, il faut donc utiliser la ligne de commande, et frapper 
\begin{description}
\item[Sous Mac]\verb|open chemin_du_fichier|.
\item[Sous Windows] ?
\item[Sous Linux] ?

\begin{attention}
Il ne faut jamais modifier les fichiers standards. En effet, en cas de mise à jour de ces fichiers, vous perdriez toutes vos modifications. Il vaut mieux en général redéfinir les commandes, quitte à grouper ces redéfinitions dans un fichier.
\end{attention}