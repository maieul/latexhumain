\chapter{Algunos programas para trabajar con \LaTeX}\label{logiciels}

\begin{intro}
Este capítulo presenta algunos programas para trabajar con \LaTeX. Esta lista no es completa. Solo se ofrecen programas gratuitos y con licencia libre.
\end{intro}

\section{Editores de texto especializados en \LaTeX}\label{editeurs}

\logiciel{TeXMaker}{Multiplataforma}{
	De entre los programas para el \enquote{gran público}, sin duda es el que mejor se adapta a los trabajos largos de redacción: posibilidades de mostrar el esquema, de abrir automáticamente ficheros incluidos, diseño con pestañas, configuraciones avanzadas de los atajos de teclado, etc. La única crítica que se le podría hacer es que tarda bastante en arrancar.
	
	Para pedirle que guarde en UTF-8, hay que ir a las preferencias, en la pestaña \forme{Editor}.
	
	Lamentablement, est programa no ofrece por defecto un botón para compilar con \XeLaTeX. Por tanto, antes que nada, tendrás que ir a las preferencias, en la pestaña \forme{Comandos} y cambiar todos los \enquote{latex} o \enquote{pdflatex} por \enquote{xelatex}, y luego aplicar. 
	

	Para compilar con \XeLaTeX, habrá que usar la herramienta de compilación con \LaTeX.
}




\logiciel{TeXWorks}{Multiplataforma}{
	Sin duda, es el primer programa que hay que utilizar para comenzar con \LaTeX. Las opciones son relativamente limitadas, lo que favorece un dominio rápido. Además, el sistema de visualización de los PDF es muy práctico, ya que permite ver en paralalo la versión PDF y la versión \LaTeX.
	Sin embargo, pronto encontraremos este programa limitado en lo que se refiere a funcionalidades.
	
	Para pedirle que guarde en UTF-8, hay que ir a las preferencias, en la pestaña \forme{Editor}.
}
\logiciel{TeXShop}{Mac}{
	
	Este programa viene con MacTeX. Ofrece botones de composición encima de cada ventana. Aunque es relativamente fácil de usa, sin embargo, carece de las funcionalidades para mostrar el esquema y para abrir automáticamente ficheros incluidos. Tiene pocas configuraciones de edición avanzadas. Su fuerza radica, a nuestro entender, en su rapidez de lanzamiento y en su fluidez.
	
	Para pedirle que guarde en en UTF-8, hay que ir a las preferencias, en la pestaña \forme{Documento}.

}

\section{Programas de gestión bibliográfica con formato \ext{bib}}\label{logicielbiblio}

Hay dos programas principales para la gestión de ficherosI \ext{bib}: JabRef\footnote{\url{http://jabref.sourceforge.net/}, multiplataforma.} y BibDesk\footnote{\url{http://bibdesk.sourceforge.net/}, que viene con MacTeX, disponible únicamente en Mac.}. Ambos ofrecen muchas funcionalidades:  búsquedas de duplicados, clasificación según varios criterios, importación / exportación a formatos distintos de \ext{bib}. La elección entre ambos es, por tanto, un asunto delicado. Los usuarios de Mac prefieren sin duda BibDesk, por la cercanía de su organización con ciertos programas de Apple. 

Para BibDesk, guardar en UTF-8 se hace en las preferencias, en el botón \forme{archivo}. Para  Jabref, se hace en las preferencias, en la pestaña \forme{general}.





