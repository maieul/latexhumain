\chapter{Glosario}


\begin{glossaire}
\item[Argumento] Parámetro que se pasa a un comando y que determina total o parcialmente su resultado.


\item[Campo bibliográfico] Elemento de una referencia bibliográfica, tal como el nombre del autor, título, editor…


\item[Clase] Indicación del tipo editorial del documento que vamos a producir. Su elección se hace al comienzo del documento \LaTeX mediante \cs{documentclass}\oarg{opciones}\marg{clase}. 

La clase influye especialmente en el resultado final y en los comandos disponibles.

\item[Clave bibliográfica] Identificador único atribuido a una referencia bibliográfica. Se aconseja poner solamente caracteres alfanuméricos sin acentos..

\item[Clave de índice] Código que indica la posición de una entrada en un índice. Hay que definir una clave de índice para las entradas que contienen caracteres acentuados. Las claves de índice también pueden servir para índices no alfabéticos (por ejemplo, índices por fecha de reinado.). La sintaxis es: \cs{index}\verb|{|\meta{clave}\verb|@|\meta{entrada}\verb|}|.

\item[Comando] Fragmento de código que el compilador interpreta para realizar un conjunto de operaciones. Un comando es un atajo de escritura. \LaTeX ofrece comandos, los packages añaden más y podemos definir nuestros propios comandos.

\item[Comentario] Texto que el compilador no interpreta. Todo lo que está situado entre el signo \% y el final de la línea es un comentario.

\item[Compilador] \enquote{Programa} que se encarga de interpretar el fichero \ext{tex} para producir un fichero \ext{pdf}. En este libro utilizamos el compilador \XeLaTeX, más reciente que \LaTeX.

\item[Contador] Número entero almacenado en la memoria del ordenador y manipulable mediante diversos comandos. Por ejemplo, los elementos numerados están asociados a un contador.


\item[Entorno] Porción de documento que tiene un significado específico y que, en consecuencia, recibe un tratamiento específico. Se utiliza \cs{begin}\marg{nombre del entorno} para indicar el comienzo de un entorno; se utiliza \cs{end}\marg{nombre del entorno} para indicar su fin. \LaTeX define entornos, los packages añaden otros más y podemos crear nuestros propios entornos.

\item[Flotante] Elemento no textual (imagen o tabla, por ejemplo) cuya inclusión en el flujo del texto la calcula automáticamente \LaTeX, de acuerdo con las recomendaciones que le proporciona el autor.

\item[Macro bibliográfica] Subelemento de un estilo bibliográfico, en general, encargado de gestionar el aspecto de uno o varios campos.

\item[Package] Cargados en el preámbulo, los packages son ficheros que permiten ampliar las funcionalidades de \LaTeX.

\item[Preámbulo] Parte del código \ext{tex} situada antes de \cs{begin}\verb|{document}|. Su contenido no se muestra en el documento final, pero tiene varios usos, especialmente la carga de los packages.

\item[Estilo bibliográfico] Manera de presentar una entrada bibliográfica, independientemente de su contenido. El package \package{biblatex} ofrece varios estilos bibliográficos que se pueden personalizar. 

\item[WYSIWYG] Abreviatura de \textenglish{\emph{What You See Is What You Get}}: \enquote{lo que ves es lo que produces}. Se dice de un programa que muestra en la pantalla el resultado final. Los procesadores de texto como LibreOffice o Microsoft Word son programas WYSIWYG.
\end{glossaire}

