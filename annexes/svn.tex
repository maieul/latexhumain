\chapter{Trabajo en equipo: programas de seguimiento de las revisiones}\label{principesvn}    % Voir s'il ne faut pas mettre à un autre niveau

\begin{intro} 
    La mayoría de los procesadores de texto ofrece sistemas para trabajar con varias personas en un mismo fichero y mantener un historial de revisiones, indicando las modificaciones, sus fechas y sus autores respectivos. 
    
    ¿Cómo podemos hacerlo con \LaTeX{}? La solución más sencilla consiste en utilizar un programa de seguimiento de revisiones, empleado por los programadores que trabajan juntos en un proyecto.
\end{intro}

\section{Principio}
Imaginemos que Bob y Alice trabajan juntos en un proyecto\footnote{El número de colaboradores con estos sistemas es ilimitado: aquí solo ponemos dos para simplificar.}. Quieren poder intercambiar con facilidad sus modificaciones.

Para ello, depositan los ficheros iniciales en un servidor (un ordenador remoto). Luego, cada uno recupera estos ficheros en su propio ordenador, con la ayuda de comandos específicos del programa utilizado.

Bob hace modificaciones (1): envía esas modificaciones al servidor junto con un pequeño mensaje que las resume. Alice puede entonces recuperar esas modificaciones localmente (2). Si Bob no ha modificado más que un fichero, Alice solo recuperará ese fichero: esto permite a Alice modificar otros ficheros entre tanto. Una vez hechas sus modificaciones, Alice las envía al servidor junto con un pequeño mensaje que las resume (3), y Bob puede así recuperarlas. Estas fases se representan en el esquema~\ref{svn} (p.~\pageref{svn}). 

También se pueden hacer otras operaciones: cambiar de nombre, desplazar, duplicar los ficheros en el servidor remoto. Además, estos programas permiten, si dos personas han hecho modificaciones a la vez en el mismo fichero, ofrecer la posibilidad de guardar solo algunas de estas modificaciones.\footnote{La jerga informática llama a esta etapa un \emph{resolución de conflicto}.}

\begin{figure}[ht]
\centering
\chapter{Travail collaboratif : les logiciels de suivi des révisions}\label{principesvn}    % Voir s'il ne faut pas mettre à un autre niveau

\begin{intro}
    La plupart des traitements de texte proposent des systèmes pour travailler à plusieurs sur un même fichier et garder un historique des révisions, indiquant les modifications, leurs dates et leurs auteurs respectifs. 
    
    Comment faire en \LaTeX{} ? La solution la plus simple est d'utiliser un logiciel de suivi des révisions, utilisé par les programmateurs qui travaillent à plusieurs sur un projet.
\end{intro}

\section{Principe}
Imaginons que Bob et Alice travaillent ensemble sur un projet\footnote{Le nombre de collaborateurs avec de tels systèmes est illimité : ici nous n'en prenons que deux pour simplifier.}. Ils souhaitent pouvoir échanger facilement leurs modifications.

Ils déposent  pour cela  les fichiers initiaux sur un serveur (un ordinateur distant). Puis chacun récupère ces fichiers sur son propre ordinateur, à l'aide de commandes spécifiques au logiciel utilisé.

Bob fait des modifications (1) : il envoie ces modifications sur le serveur, en les accompagnant d'un petit message les résumant. Alice peut alors récupérer ces modifications en local (2). Si Bob n'a modifié qu'un seul fichier, seul ce fichier sera récupéré par Alice : cela permet à Alice de modifier d'autres fichiers pendant ce temps. Une fois ses modifications faites, Alice les envoie sur le serveur, en les accompagnant d'un petit message les résumant (3), et Bob peut ainsi les récupérer. Ces étapes sont représentées dans le schéma~\ref{svn} (p.~\pageref{svn}). 

Il est aussi possible de faire d'autres opérations : renommer, déplacer, dupliquer les fichiers sur le serveur distant. En outre, ces logiciels permettent, si deux personnes ont fait des modifications en même temps sur le même fichier, de proposer de ne garder que certaines de ces modifications.\footnote{Le jargon nomme cette étape une \emph{résolution de conflit}.}

\begin{figure}[ht]
\centering
\chapter{Travail collaboratif : les logiciels de suivi des révisions}\label{principesvn}    % Voir s'il ne faut pas mettre à un autre niveau

\begin{intro}
    La plupart des traitements de texte proposent des systèmes pour travailler à plusieurs sur un même fichier et garder un historique des révisions, indiquant les modifications, leurs dates et leurs auteurs respectifs. 
    
    Comment faire en \LaTeX{} ? La solution la plus simple est d'utiliser un logiciel de suivi des révisions, utilisé par les programmateurs qui travaillent à plusieurs sur un projet.
\end{intro}

\section{Principe}
Imaginons que Bob et Alice travaillent ensemble sur un projet\footnote{Le nombre de collaborateurs avec de tels systèmes est illimité : ici nous n'en prenons que deux pour simplifier.}. Ils souhaitent pouvoir échanger facilement leurs modifications.

Ils déposent  pour cela  les fichiers initiaux sur un serveur (un ordinateur distant). Puis chacun récupère ces fichiers sur son propre ordinateur, à l'aide de commandes spécifiques au logiciel utilisé.

Bob fait des modifications (1) : il envoie ces modifications sur le serveur, en les accompagnant d'un petit message les résumant. Alice peut alors récupérer ces modifications en local (2). Si Bob n'a modifié qu'un seul fichier, seul ce fichier sera récupéré par Alice : cela permet à Alice de modifier d'autres fichiers pendant ce temps. Une fois ses modifications faites, Alice les envoie sur le serveur, en les accompagnant d'un petit message les résumant (3), et Bob peut ainsi les récupérer. Ces étapes sont représentées dans le schéma~\ref{svn} (p.~\pageref{svn}). 

Il est aussi possible de faire d'autres opérations : renommer, déplacer, dupliquer les fichiers sur le serveur distant. En outre, ces logiciels permettent, si deux personnes ont fait des modifications en même temps sur le même fichier, de proposer de ne garder que certaines de ces modifications.\footnote{Le jargon nomme cette étape une \emph{résolution de conflit}.}

\begin{figure}[ht]
\centering
\chapter{Travail collaboratif : les logiciels de suivi des révisions}\label{principesvn}    % Voir s'il ne faut pas mettre à un autre niveau

\begin{intro}
    La plupart des traitements de texte proposent des systèmes pour travailler à plusieurs sur un même fichier et garder un historique des révisions, indiquant les modifications, leurs dates et leurs auteurs respectifs. 
    
    Comment faire en \LaTeX{} ? La solution la plus simple est d'utiliser un logiciel de suivi des révisions, utilisé par les programmateurs qui travaillent à plusieurs sur un projet.
\end{intro}

\section{Principe}
Imaginons que Bob et Alice travaillent ensemble sur un projet\footnote{Le nombre de collaborateurs avec de tels systèmes est illimité : ici nous n'en prenons que deux pour simplifier.}. Ils souhaitent pouvoir échanger facilement leurs modifications.

Ils déposent  pour cela  les fichiers initiaux sur un serveur (un ordinateur distant). Puis chacun récupère ces fichiers sur son propre ordinateur, à l'aide de commandes spécifiques au logiciel utilisé.

Bob fait des modifications (1) : il envoie ces modifications sur le serveur, en les accompagnant d'un petit message les résumant. Alice peut alors récupérer ces modifications en local (2). Si Bob n'a modifié qu'un seul fichier, seul ce fichier sera récupéré par Alice : cela permet à Alice de modifier d'autres fichiers pendant ce temps. Une fois ses modifications faites, Alice les envoie sur le serveur, en les accompagnant d'un petit message les résumant (3), et Bob peut ainsi les récupérer. Ces étapes sont représentées dans le schéma~\ref{svn} (p.~\pageref{svn}). 

Il est aussi possible de faire d'autres opérations : renommer, déplacer, dupliquer les fichiers sur le serveur distant. En outre, ces logiciels permettent, si deux personnes ont fait des modifications en même temps sur le même fichier, de proposer de ne garder que certaines de ces modifications.\footnote{Le jargon nomme cette étape une \emph{résolution de conflit}.}

\begin{figure}[ht]
\centering
\include{schemas/svn}
\caption{Fonctionnement des logiciels de suivi des révision}\label{svn}
\end{figure}

Un tel système présente de nombreux intérêts :
\begin{itemize}
\item il est aisé de s'assurer que l'on possède bien la dernière version du projet, contrairement aux échanges classiques par courriels qui peuvent rapidement semer la confusion;
\item on dispose d'un historique des versions permettant de revenir en arrière, le cas échéant\footnote{C'est pourquoi une personne seule peut aussi trouver un intérêt à utiliser un tel système, pour se garantir un suivi du travail.};
\item contrairement aux systèmes des logiciels de traitement de texte, l'historique des versions ne se situe pas à l'intérieur du fichier de travail, ce qui évite une prise de poids et un certain nombre de problèmes (notamment ralentissement et corruption du fichier).
\end{itemize}


\section{Mise en pratique}
Il existe de nombreux logiciels de suivi des révisions  : citons SVN, Git et Mercurial. Le premier est sans doute plus simple en première approche, mais les deux autres possèdent des fonctionnalités plus souples, notamment parce qu'il n'y a pas un serveur unique, comme sur notre schéma, mais plusieurs serveurs qui peuvent être synchronisés.

Un certain nombre d'hébergeurs proposent des services SVN, Git ou Mercurial\footnote{Toutefois par souci de confidentialité il est conseillé d'utiliser un hébergement SVN~/ Git~/ Mercurial interne à l'institution dans laquelle ce projet commun s'inscrit, en demandant, le cas échéant, au service informatique de le fournir.}. 

Git et Mercurial ne fonctionnent pas nécessairement avec un serveur distant. C'est pourquoi on peut facilement s'en servir pour gérer un historique de modification sur un projet mono-rédacteur.

La manipulation de ces outils peut se faire en ligne de commande, mais il existe également des logiciels graphiques.

Nous invitons le lecteur curieux à se renseigner sur internet pour plus d'informations : il y trouvera aisément documentations et tutoriels\footcites[Nous recommandons, pour Git, cet excellent livre :][]{progit}[pour notre part, nous avons écrit un petit tutoriel sur l'Utilisation de Git avec une seule personne :][]{git}.

\caption{Fonctionnement des logiciels de suivi des révision}\label{svn}
\end{figure}

Un tel système présente de nombreux intérêts :
\begin{itemize}
\item il est aisé de s'assurer que l'on possède bien la dernière version du projet, contrairement aux échanges classiques par courriels qui peuvent rapidement semer la confusion;
\item on dispose d'un historique des versions permettant de revenir en arrière, le cas échéant\footnote{C'est pourquoi une personne seule peut aussi trouver un intérêt à utiliser un tel système, pour se garantir un suivi du travail.};
\item contrairement aux systèmes des logiciels de traitement de texte, l'historique des versions ne se situe pas à l'intérieur du fichier de travail, ce qui évite une prise de poids et un certain nombre de problèmes (notamment ralentissement et corruption du fichier).
\end{itemize}


\section{Mise en pratique}
Il existe de nombreux logiciels de suivi des révisions  : citons SVN, Git et Mercurial. Le premier est sans doute plus simple en première approche, mais les deux autres possèdent des fonctionnalités plus souples, notamment parce qu'il n'y a pas un serveur unique, comme sur notre schéma, mais plusieurs serveurs qui peuvent être synchronisés.

Un certain nombre d'hébergeurs proposent des services SVN, Git ou Mercurial\footnote{Toutefois par souci de confidentialité il est conseillé d'utiliser un hébergement SVN~/ Git~/ Mercurial interne à l'institution dans laquelle ce projet commun s'inscrit, en demandant, le cas échéant, au service informatique de le fournir.}. 

Git et Mercurial ne fonctionnent pas nécessairement avec un serveur distant. C'est pourquoi on peut facilement s'en servir pour gérer un historique de modification sur un projet mono-rédacteur.

La manipulation de ces outils peut se faire en ligne de commande, mais il existe également des logiciels graphiques.

Nous invitons le lecteur curieux à se renseigner sur internet pour plus d'informations : il y trouvera aisément documentations et tutoriels\footcites[Nous recommandons, pour Git, cet excellent livre :][]{progit}[pour notre part, nous avons écrit un petit tutoriel sur l'Utilisation de Git avec une seule personne :][]{git}.

\caption{Fonctionnement des logiciels de suivi des révision}\label{svn}
\end{figure}

Un tel système présente de nombreux intérêts :
\begin{itemize}
\item il est aisé de s'assurer que l'on possède bien la dernière version du projet, contrairement aux échanges classiques par courriels qui peuvent rapidement semer la confusion;
\item on dispose d'un historique des versions permettant de revenir en arrière, le cas échéant\footnote{C'est pourquoi une personne seule peut aussi trouver un intérêt à utiliser un tel système, pour se garantir un suivi du travail.};
\item contrairement aux systèmes des logiciels de traitement de texte, l'historique des versions ne se situe pas à l'intérieur du fichier de travail, ce qui évite une prise de poids et un certain nombre de problèmes (notamment ralentissement et corruption du fichier).
\end{itemize}


\section{Mise en pratique}
Il existe de nombreux logiciels de suivi des révisions  : citons SVN, Git et Mercurial. Le premier est sans doute plus simple en première approche, mais les deux autres possèdent des fonctionnalités plus souples, notamment parce qu'il n'y a pas un serveur unique, comme sur notre schéma, mais plusieurs serveurs qui peuvent être synchronisés.

Un certain nombre d'hébergeurs proposent des services SVN, Git ou Mercurial\footnote{Toutefois par souci de confidentialité il est conseillé d'utiliser un hébergement SVN~/ Git~/ Mercurial interne à l'institution dans laquelle ce projet commun s'inscrit, en demandant, le cas échéant, au service informatique de le fournir.}. 

Git et Mercurial ne fonctionnent pas nécessairement avec un serveur distant. C'est pourquoi on peut facilement s'en servir pour gérer un historique de modification sur un projet mono-rédacteur.

La manipulation de ces outils peut se faire en ligne de commande, mais il existe également des logiciels graphiques.

Nous invitons le lecteur curieux à se renseigner sur internet pour plus d'informations : il y trouvera aisément documentations et tutoriels\footcites[Nous recommandons, pour Git, cet excellent livre :][]{progit}[pour notre part, nous avons écrit un petit tutoriel sur l'Utilisation de Git avec une seule personne :][]{git}.

\caption{Funcionamiento de los programas de seguimiento de revisiones}\label{svn}
\end{figure}

Un sistema así presenta muchos aspectos interesantes:
\begin{itemize}
\item es fácil asegurarse de que lo que tenemos es la última versión del proyecto, a diferencia de los intercambios clásicos por correo electrónico que pueden sembrar pronto la confusión;
\item disponemos de un historial de versiones para volver atrás, llegado el caso\footnote{Por eso, incluso una persona sola puede encontrar interesante usar un sistema como este, para garantizarse un seguimiento del trabajo.};
\item a diferencia de lo que ocurre en los sistemas de los procesadores de texto, el historial de las versiones no se encuentra dentro del fichero de trabajo, lo que evita un aumento de peso y una serie de problemas (incluyendo la ralentización y corrupción del fichero).
\end{itemize}


\section{En la práctica}
Hay numerosos programas de seguimiento de revisiones: citemos SVN, Git y Mercurial. El primero es sin duda más sencillo para un primer acercamiento, pero los otros dos tienen funcionalidades más flexibles, especialmente porque no hay un único servidor, como en nuestro esquema, sino varios servidores que se pueden sincronizar.

Cierto número de anfitriones ofrecen servicios SVN, Git o Mercurial\footnote{Sin embargo, en aras de la confidencialidad, es recomendable utilizar un alojamiento SVN~/ Git~/ Mercurial dentro de la institución de la que este proyecto común forma parte, pidiendo, si llega el caso, al servicio informático que lo ofrezca.}. 

Git y Mercurial no funcionan necesariamente con un servidor remoto. Por eso se pueden utilizar fácilmente para gestionar un historial de modificaciones en un proyecto de un solo autor.

La manipulación de estas herramientas se puede hacer en la línea de comandos, pero también hay programas gráficos.

Invitamos al lector curioso a que consulte en internet para más informaciones: allí encontrará con facilidad documentos y tutoriales\footcites[Recomendamos, para Git, este magnífico libro:][]{progit}[por nuestra parte, hemos escrito un pequeño tutorial sobre el Uso de Git con una sola persona:][]{git}.

