\chapter {Instalar \LaTeX{}}\label{install}

\begin{intro}
    Antes de poder utilizar \LaTeX{} (\XeLaTeX), hay que instalarlo en el ordenadoril faut l'installer sur son ordinateur. Veamos cómo hacerlo en los sistemas operativos usuales\footnote{Gracias a Brendan Chabannes por la redacción de la mayor parte de este capítulo.}.
\end{intro}

\section{La noción de distribución}

Eb general, no se instala \LaTeX{} solo, se instala una distribución \LaTeX{}. Una distribución es un conjunto de ficheros que comprende: 
\begin{itemize}
\item los programas \TeX, \LaTeX, y generalmente \XeLaTeX, pero también otros programas que peretenecen  a la  \forme{familia \TeX\footnote{Por hacer una analogía con los programas conocidos, cuando instalamos Microsoft Office, no solo instalamos Word, Excel y Powerpoint, sino también programas que usan estos últimos, por ejemplo, para trazar organigramas.}}, que veremos en su momento;
\item ficheros que permiten ampliar las posibilidades de \LaTeX{}: los packages y las clases;
\item la documentación sobre estos ficheros.
\end{itemize}

Hay dos distribuciones comunes: TeX Live, que se considera portátil y muy adaptada para MacOS X y Linux, y MikTeX solo para Windows. La finalidad de este capítulo consiste en explicar cómo se instalan estas distribuciones.

\section{Instalar TeX Live en Mac Os X}

La forma más sencilla consiste en instalar MacTeX\footnote{\url{http://www.tug.org/mactex/}}. No solamente consta de una versión de TeX Live para Mac, sino también de los programas para facilitar la escritura con \LaTeX{}.

Para instalarla, basta con descargar el fichero de instalación del sitio web en su página de bienvenida. Se trata de un fichero \ext{zip}, que hay que descomprimir. Dentro se encuentra un fichero \ext{mpkg}: haciendo doble clic en él, se lanza el programa de instalación. Es mejor mantener la configuración estándar de la instalación.

Una vez completada la instalación, podemos dirigirnos a la carpeta \forme{TeX} dentro de la carpeta \forme{Aplicaciónes} de nuestro Mac.

Esta carpeta tiene varias aplicaciones:
\begin{glossaire}
\item[BibDesk]que es un programa para gestionar los ficheros bibliográficos con formato \ext{bib}\renvoi{bddbiblio}.
\item[Excalibur]que es un programa de corrección ortográfica pensado para reconocer los comandos \LaTeX{}. Desconocemos su calidad, porque no lo hemos probado.
\item[LaTeXit]que es un programa para escribir aún con más facilidad ecuaciones en \LaTeX{} y para exportarlas a otros programas. Como este libro es uno de los pocos sobre \LaTeX{} que no explica cómo hacer ecuaciones con él, no nos detendremos en este punto.
\item[TeXLive Utility]que sirve par actualizar los distintos módulos de \LaTeX{}: clases, packages, etc.\renvoi{majosX} 
\item[TeXworks et TeXShop]que son dos editores de texto pensados para \LaTeX. Podemos elegir uno de ellos para comenzar en \LaTeX{}\renvoi{commencer}.
\end{glossaire}

Además, se añade un panel de configuración \enquote{Distribution TeX} en las Preferencias del Sistema. En la mayoría de los casos, no tendremos que modificarlo. En efecto, permite elegir entre varias versiones de la distribución (por ejemplo, entre la versión de 2011 y la de 2012), cosa que pocas veces es necesaria.
\subsection{Los caracteres \LaTeX en Mac Os X}\label{claviermac}

Los teclados Apple no indican directamente los caracteres especiales disponibles. Veamos cómo escribir la mayoría de los caracteres necesarios para el uso de \LaTeX.

\begin{longtable}{|l|l|}
\hline
\headlongtable{Carácter} & \headlongtable{Escritura} \\
\hline
\endhead
\hline
\endfoot
\verb+\+ & \verb|⌥ + ⇧ + /| \\
\verb+{+ & \verb|⌥ + (|\\
\verb+}+ & \verb|⌥ + )|\\
\verb+[+ & \verb|⌥ + ⇧ + (| \\
\verb+]+ & \verb|⌥ + ⇧ + )| \\
\verb+|+ & \verb|⌥ + ⇧ + L|  \\
\verb+~+ & \verb|⌥ + N|  \\
\end{longtable}

Y para los teclados de la Suiza francófona, utilizando la barra superior de números:


\begin{longtable}{|l|l|}
\hline
\headlongtable{Carácter} & \headlongtable{Escritura} \\
\hline
\endhead
\hline
\endfoot
\verb+\+ & \verb|⌥ + ⇧ + 7| \\
\verb+{+ & \verb|⌥ + 8|\\
\verb+}+ & \verb|⌥ + 9|\\
\verb+[+ & \verb|⌥ + ⇧ + 5| \\
\verb+]+ & \verb|⌥ + ⇧ + 6| \\
\verb+|+ & \verb|⌥ + ⇧ + 7|  \\
\verb+~+ & \verb|⌥ + N|  \\
\end{longtable}



\section{Instalar TeX Live en GNU/Linux}

La mayor parte de las veces, los paquetes TeX Live están disponibles en los repositorios de nuestra distribución. Sin embargo, la mayoría de las distribuciones GNU/Linux no son compatibles con algunos packages recientes como Biber\footnote{Aunque las cosas están cambiando: las próximas versiones de estos sistemas, como  Debian o Ubuntu (que se llaman, respectivamente \emph{Wheezy} y \emph{Quantal Quetzal}) dispondrán de TeX Live 2012 y de Biber.}. Por eso propiciaremos la instalación a partir del programa que podamos encontrar en el sitio web de TeX Live.

Una vez realizada una instalación completa, pesa aproximadamente tres gigabytes. A medida que pasa el tiempo y las actualizaciones se acumulan, se recomienda tener cuatro gigabytes de espacio libre en el disco duro.

\subsection{Previamente}

Asegúrate, ente todo, de tener el paquete \verb|perltk| instalado en tu sistema. Si no lo tienes, tendrás que lanzar la instalación solo a través de la línea de comandos.

A modo de ejemplo, el paquete se llama \forme{perl-tk} en Debian y Ubuntu, \forme{perl-Tk-804} en Fedora y sus derivados.

En Ubuntu, por ejemplo, hay una forma muy sencilla de instalar este paquete: teclea \url{apt://perl-tk} en la barra de dirección de tu navegador web Firefox.

Dirígete a \url{http://tug.org/texlive/acquire-netinstall.html}. Elige el fichero \emph{install-tl-unx.tar.gz}. Tras su descarga, descomprímelo y dirígete a la carpeta recientemente creada con la ayuda de un terminal, tecleadno, por ejemplo \verb|cd ~/Téléchargements/install-tl-*/|. Entonces puedes lanzar la instalación en modo administrador, por ejemplo, así:

\begin{bashcode}
sudo perl install-tl -gui wizard
\end{bashcode}
Este comando lanzará el programa de instalación de la distribución completa que reconocerá automáticamente la arquitectura de tu plataforma.

\subsection{Instalación y configuración}

Después de escribir tu contraseña, te bastará con seguir las etapas siguientes:

\begin{enumerate}
\item Pulsa simplemente \forme{Siguiente}.
\item Espera a que haya acabado la descarga. Tendrás que elegir una carpeta de instalación: escoge preferentemente un repertorio fuera del sistema de base, como \verb|/$HOME/texlive2012|\footnote{\forme{\$HOME} es una \emph{variable de entorno}. También podrías teclear \forme{/home/justin/} si tu nombre de usuario es \enquote{justin}.}.
\item Mantén A4 para el tamaño del papel.
\item Pulsa sobre Instalar.
\item Al final de la instalación, un mensaje te informará de que tienes que ajustar tu \forme{\$PATH}. Abre el fichero \verb|~/.bashrc| (créalo, si no existe), y añade en él las líneas siguientes:\footnote{\forme{\$PATH} es una variable de entorno que permite al sistema saber dónde buscar los ejecutables. Como TeX Live no se instala en las carpetas por defecto, ya que es extraño al sistema de base, tienes que integrarlo tú. Ajusta el campo entre comillas según el mensaje que se muestra en la pantalla, teniendo cuidado de acabar con \forme{:\$PATH}.}\\
  \begin{english}\verb|export PATH="/$HOME/texlive2012/bin/x86_64-linux:$PATH"|\end{english}
\end{enumerate}

La instalación ha concluido. Puedes cerrar tu terminal para que los cambios aplicados a tu \forme{\$PATH} queden guardados. 

\section{MiKTeX en Windows}

Los distemas Windows son considerablemente distintos de los sistemas GNU/Linux y Mac OS X. Por tanto, aunque se puede hacer una instalación manual de la distribución TeX Live en Windows, el procedimiento es tedioso y complejo.

Por eso recurriremos a una distribución pensada únicamente para Windows, que automatiza las tareas de instalación de la distribución \LaTeX{} y que permite administrar la instalación cumpliendo con las prácticas que se aplican en este sistema. Esta distribución de llama MiKTeX. Su instalador es muy completo: además de una distribución \LaTeX{}, instala también un programa gráfico para actualizar los paquetes y un editor de texto.

\subsection{Instalación}

Hay que dirigirse a la página de descargas de la distribución: \url{http://miktex.org/download}. Pulsa sobre el texto \emph{Other downloads}. Elige el fichero \emph{Net Installer}, y no \emph{Basic Installer}.
%
\begin{attention}
Observarás que se ofrecen instaladores para las versiones llamadas, respectivamente, de 32 bits y de 64 bits de Windows. Si no conoces estos términos, elige la versión de \emph{32 bits}. Esta última, en efecto, puede ejecutarse en los dos tipos de plataformas, mientras que a la inversa no es posible. Si sabes que tienes un sistema de 64 bits, puedes elegir sin miedo el instalador que le corresponde.
\end{attention}

Una vez descargado el instalador en tu ordenador, un doble clic lanza el procedimiento. Primero hay que aceptar la licencia del programa. Luego, elige \enquote{Download MiKTeX}, y, en la pantalla siguiente, \enquote{Complete MiKTeX}.

Se te pide que elijas una fuente de descarga. Elige una fuente cercana a tu domicilio, por lo que para un usuario que viva al norte de Francia, los servidores franceses, ingleses o alemanes sirven perfectamente.

En la fase siguiente hay que elegir una carpeta de descarga. Elige, preferentemente, una carpeta que se encuentre en \enquote{Mis Documentos}, como, por ejemplo, \begin{english}\verb|C:\Documents and Settings\Usuario\Mis documentos\miktex|\end{english}. Cuando hayas concluido esta fase, el instalador descarga todos los componentes que necesita, te advierte de que ha acabado su trabajo y luego se detiene.

Lánzalo ahora por segunda vez. Pero, en vez de elegir \enquote{Download MiKTeX}, selecciona ahora \enquote{Install MiKTeX}, y luego, de nuevo \enquote{Basic MiKTeX}. 
Es preferible que hagas una instalación como administrador para todos los usuarios del ordenador. Después elige la carpeta que contiene los componentes; en nuestro ejemplo anterior, se trataba de \begin{english}\verb|C:\Documents and SettingsUsuario\Mis documentos\miktex|\end{english}.

Mantén las opciones por defecto en las dos pantallas siguientes: son correctas. Puedes acabar el proceso.

\section{Actualizar los packages}\label{maj}

La mayoría de los packages aparecen en una lista en \url{http://www.ctan.org/tex-archive/} que sirve como depósito para todos los proyectos que tienen que ver con \TeX y \LaTeX. Por lo general, es en los servidores de CTAN (\textenglish{\emph{Comprehensive TeX Archive Network}}) donde los programas de actualización de los packages van a buscarlos.

\begin{plusloins}
	Para estar al corriente de los nuevos packages y de actualizaciones de los antiguos packages, puedes apuntarte a la lista \url{https://lists.dante.de/mailman/listinfo/ctan-ann}. 
\end{plusloins}

\begin{attention}
    Las actualizaciones de la distribución TeX Live solo se ofrecen por un período de un año aproximadamente tras su salida (por lo general, entre mayo y julio). Pasado ese plazo, cuando intentes actualizar tu distribución TeX Live, te encontrarás con un mensaje de este tipo:

    \begin{quotation}
/home/justin/texlive2011/bin/x86\_64-linux/tlmgr: The TeX Live versions supported by the repository\\
  (2012--2012)\\
do not include the version of the local installation\\
  (2011).  Goodbye.
    \end{quotation}
    
    ou bien
    
    \begin{quotation}
    TeX Live 2011 is frozen forever and will no
longer be updated.
	\end{quotation}

En tal caso, tendrás que instalar una nueva versión de la distribución TeX Live, comenzando de nuevo su proceso de instalación, o podrás seguir usando la antigua que ya no tendrá más actualizaciones. TeX Live tiene gran estabilidad, pero no es necesariamente así para todos los packages que puede que tengamos que usar. Si uno de los packages que utilizas se ve afectado por un error que te molesta, no tendrás más remedio que poner al día la distribución TeX Live. En caso contrario, perfectamente te puedes mantener con la versión antigua.
\end{attention}

\subsection{En Mac OS X}\label{majosX}

Para actualizar los packages,Lo más sencillo es usar el programa \forme{TeXLive Utility}, que viene con MacTeX. Una vez abierto el programa, basta con elegir la pestaña \forme{Manage Updates}, después de seleccionar el package que quieres actualizar --- si no aparece, es que está actualizado --- antes de pulsar sobre \forme{Update}. Hay que notar que también se puede utilizar la pestaña \forme{Manage Packages} para gestionar la instalación de nuevos packages.

\subsection{En Linux}

Hay dos formas de actualizar tu instalación \LaTeX en GNU/Linux.

\begin{enumerate}
\item La primera forma de actualizarlo consiste simplemente en usar \verb|tlmgr| en un terminal. El comando que permite la actualización es el siguiente:
\verb|tlmgr update --self --all|. Si se necesita instalar un nuevo paquete, hay que utilizar el comando \verb|tlmgr install paquete|.
\item Puedes lanzar la interfaz gráfica de gestión de paquetes escribiendo el comando \verb|tlmgr gui|.
\end{enumerate}

\subsection{En Windows}

En el menú Inicio, encontrarás un apartado llamado precisamente MiKTeX, seguido de un número de versión. En este apartado, hay un subapartado dedicado al \enquote{Mantenimiento} --- que responde al nombre de de \forme{Maintenance (Admin)} ---, en el que tienes acceso a dos herramientas útiles: el gestor de actualizaciones de packages (\forme{Updates}), y el gestor de instalación de nuevos packages (\forme{Package Manager}).

Fíjate en que, la mayor parte de las veces, no necesitarás instalar explícitamente un nuevo package. En efecto, en el momento de la compilación, MiKTeX detecta que solicitas un componente que no está en tu sistema y te ofrece instalarlo.
