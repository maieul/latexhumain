\chapter{Unidades de medida en \LaTeX{}}\label{unite}

\begin{intro}
    Explicamos aquí las principales unidades de medida en \LaTeX{}, pero también la manera de indicar tamaños relativos: para indicar, por ejemplo, que queremos tener la mitad de la longitud de la línea.
\end{intro}

\section{Unidades de medida}

Como \LaTeX{} es de origen anglófono, el separador de los decimales debe ser un punto y no una coma. Así, par indicar un tamaño de 1,5 cm, hay que escribir: \verb|1.5cm| y no \verb|1,5cm|.

Las unidades de medida pueden dividirse en dos categorías: unidades absolutas y unidades relativas al tamaño de la fuente actual. Solo presentamos aquí las unidades más frecuentes.



\begin{longtable}{|llp{15em}|}
    \hline
    \headlongtable{Unidad} & \headlongtable{Nombre} &{\centering \textbf{Explicación}}\\
    \hline
    \endhead
    \hline
    \endfoot
    \multicolumn{3}{|c|}{\emph{Unidades absolutas}} \\
    pt & Punto tipográfico pica & Unidad básica de medida tipográfica en los sistemas anglosajones. Es la que se usa en casi todos los programas. Corresponde aproximadamente a 0,35 mm. \\
    mm & milímetro &  \\
    cm  & centímetro & \\
    in    & Pulgada anglosajona & Corresponde aproximadamente a 2,54 cm. \\
    \multicolumn{3}{|c|}{\emph{Unidades relativas}} \\
    em & Cuadratín & Corresponde a la longitud de una \verb|m| en la fuente actual. \\
    ex  & Ojo & Corresponde a la altura de una \verb|x| en la fuente actual. \\
\end{longtable}

\section{Longitudes relativas}\label{longueurrelative}

También se pueden indicar longitures relativas a ciertos elementos. Basta con poner un número decimal delante de un comando de longitud. Así, para una raya horizontal de la mitad de la anchura de una línea:
\inputminted{exemples/annexes/unites/demiligne.tex}

\noindent\noindent\rule{0.5\linewidth}{1pt}


Veamos los principales comandos de longitud:

\begin{longtable}{|lp{15em}|}
    \hline
    \headlongtable{Comando} & {\centering\textbf{Explicación}} \\
    \hline
    \endhead
    \hline
    \endfoot
    \csp{baselineskip} & Distancia entre la base de una línea y la de la línea siguiente. \\
    \csp{linewidth} & Ancho de línea. En algunos entornos (por ejemplo, \enviro{quotation}) las líneas no tienen el mismo ancho que en el texto principal. \\
    \csp{textheight}& Altura máxima del texto en una página. Se corresponde \emph{grosso modo} con la distancia entre el margen superior y el margen inferior. \\
    \csp{textwidth} & Ancho máximo del texto en una página. Se corresponde \emph{grosso modo} con la distancia entre el margen izquierdo y el margen derecho.  \\
     \csp{parindent} & Longitud del sangrado del párrafo. \\
     \csp{parskip}    & Distancia vertical entre dos párrafos. \\
     
\end{longtable}

\section{Elasticidad}\label{elastique}

Se puede definir cierta elasticidad para las longitudes, lo que permite adaptarse al espacio disponible y a las limitaciones del diseño de página\footnote{Por ejemplo, en lo que se refiere a las líneas viudas y huérfanas.}. Para ello, se utilizan las palabras clave \verb|minus| y \verb|plus|.
Así, si queremos que aparezca un espacio en blanco\renvoi{espace} vertical de una longitud comprendida entre 1~cm y 3~cm, pero que se acerque a los 2,5~cm, escribimos:

\begin{latexcode}
\vspace{2.5cm minus 1.5cm plus 0.5cm}
\end{latexcode}

Esta sintaxis se utiliza en los comandos de definiciones de los encabezados de apartado\renvoi{apparencetitre}, con una forma un tanto diferente: \cs{@plus} y \csp{@minus} son simplemente sinónimos de \LaTeX de las palabras clave \TeX \verb|plus| y \verb|minus|.

De manera más general, con una sintaxis de longitud \verb|a minus b plus c| la longueur finale \verb|m| es tal que\footnote{No incluimos aquí una ecuación, sino una inecuación. El título de este libro está, por tanto, plenamente justificado.} $ a - b \leq m \leq a + c $.  

