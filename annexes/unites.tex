\chapter{Unités de mesure en \LaTeX{}}\label{unite}

\begin{prealable}
    Nous expliquons ici les principales unités de mesure en \LaTeX{}, mais aussi les manières d'indiquer des tailles relatives à une autre taille : par exemple indiquer que nous souhaitons avoir la moitié de la longueur de la ligne.
\end{prealable}

\section{Unités de mesure}

\LaTeX{} étant d'origine anglophone, le séparateur des décimales doit être un point et non pas une virgule. Ainsi pour préciser une taille de 1,5 cm, il faut écrire : \verb|1.5cm| et non pas \verb|1,5cm|.

Les unités de mesures peuvent être divisées en deux catégories : unités absolues et unités relatives à la taille  de la police courante. Nous ne présentons ici que les unités les plus courantes.



\begin{longtable}{|llp{15em}|}
    \hline
    Unité & Nom & Explication \\
    \hline
    \endhead
    \hline
    \endfoot
    \multicolumn{3}{|c|}{\emph{Unités absolues}} \\
    pt & Point typographique pica & Unité de base des mesures typographiques dans les système anglo-saxons. C'est celle qui est utilisée dans quasiment tout les logiciels. Correspond approximativement à 0,35 mm. \\
    mm & millimètre &  \\
    cm  & centimètre & \\
    in    & Pouce anglo-saxon & Correspond approximativement à 2,54 cm. \\
    \multicolumn{3}{|c|}{\emph{Unités relatives}} \\
    em & Cadratin & Correspond à la largeur d'un \verb|m| dans la police courante. \\
    ex  & Œil & Correspond à la hauteur d'un \verb|x| dans la police courante. \\
\end{longtable}

\section{Longueurs relatives}\label{longueurrelative}

Il est également possible d'indiquer des longueurs relatives à certains éléments. Pour ce faire, il suffit d'indiquer un nombre décimal devant une commande de longueur.

Ainsi pour un tableau avec deux colonnes correspondant à la moitié de la largeur de la ligne :

\inputminted{exemples/annexes/unites/tableaulongueur.tex}
\begin{tabular}{|p{0.5\linewidth}|p{0.5\linewidth}|}
	\hline 
	Cellule d'une demi-ligne & Cellule d'une demi-ligne\\
	\hline 
\end{tabular}


Voici les principales commandes de longueur :

\begin{longtable}{|lp{15em}|}
    \hline
    Commande & Explication \\
    \hline
    \endhead
    \hline
    \endfoot
    \csp{baselineskip} & Distance entre le bas d'une ligne et celui de la ligne suivante. \\
    \csp{linewidth} & Largeur d'une ligne. Dans certains environnement (par exemple \enviro{quotation}) les lignes n'ont pas la même longueur que dans le texte principal. \\
    \csp{textheight}& Hauteur maximale du texte dans une page. Correspond \emph{grosso-modo} à la longueur entre la marge haute et la marge basse. \\
    \csp{textwith} & Largeur maximale de texte dans une page. Correspond \emph{grosso-modo} à la longueur entre la marge gauche et la marge droite.  \\
     \csp{parindent} & Longueur de l'indentation du paragraphe. \\
     \csp{parskip}    & Distance verticale entre deux paragraphes. \\
     
\end{longtable}

\section{Elasticité}\label{elastique}

Il est possible de définir une certaine élasticité pour les longueurs, qui permet de s'adapter à l'espace disponible et aux contraintes de mise en page\footnote{Par exemple concernant les veuves et orphelines.}. Pour ce faire, on utilise les mot clefs \verb|minus| et \verb|plus|.
Ainsi si nous souhaite afficher un espace blanc \renvoi{espace} vertical d'une longueur comprise entre 1~cm et 3~cm, mais tournant idéalement autour de 2,5~cm, nous écrivons :

\begin{latexcode}
\vspace{2.5cm minus 1.5cm plus 0.5cm}
\end{latexcode}

Cette syntaxe est utilisé dans les commandes de définitions des en-têtes de section, sous une forme légèrement différente : \cs{@plus} et \cs{@minus} ne sont que des alias \LaTeX des mot clefs \TeX \verb|plus| et \verb|minus|.\renvoi{apparencetitre}

Plus globalement, avec une syntaxe de longueur sous la forme \verb|a minus b plus c| la longueur finale \verb|m| est telle que $ a - b \leq m \leq a + c $.  

