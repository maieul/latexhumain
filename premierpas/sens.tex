\chapter{Dar sentido a tu documento (1): primeros pasos}

\begin{intro}
Vamos a ver ahora cómo \emph{dar sentido} a nuestro documento, es decir, cómo poner señales y etiquetas para marcar el \emph{relieve} semántico del texto.
\end{intro}

\section{Dar forma no es dar sentido}\label{sensforme}

Cuando leemos un libro, no todos sus elementos se presentan de la misma manera: unos están en negrita, otros en cursiva, otros subrayados, otros en color, etc.

Todo esto constituye lo que llamamos \emph{dar forma} al texto. Si nuestro libro está bien planteado, estos cambios de forma se corresponden con cambios de significado: la cursiva puede indicar el título de una obra, o bien una cita o cierto énfasis, la negrita puede indicar un concepto, una definición o cualquier otro significado.

Como vemos, \emph{dar forma} es distinto de \emph{dar sentido}. Esto último es lo que hace, en teoría, el autor del trabajo, mientras que el editor se encarga normalmente de dar forma y diseñar la página.

Por lo demás, es lo que sucedía antes, cuando los autores todavía entregaban textos manuscritos a sus editores: indicaban los elementos a los que dar sentido por medio de signos y a este sentido le daba forma el editor\footnote{En un procesador de texto de tipo WYSIWYG, esta distinción se suele hacer con la ayuda de estilos. Por desgracia, es muy frecuente que los usuarios no sepan emplearlos.}.

En \LaTeX, el principio es el mismo: hay comandos para dar sentido que se transforman en comandos para dar forma. Mejor aún: se pueden definir comandos propios para dar sentido. El interés es evidente: se puede cambiar rápidamente de dar forma por un conjunto de datos que dan sentido.

Esto resultará más claro con un ejemplo. Imaginemos que escribimos un libro de introducción a la historia del cristianismo antiguo. El libro cita varios autores. Deseamos destacar esos autores y, para hacerlo, decidimos poner sus nombres en versalitas.

Por tanto, cada vez que citamos un autor, indicamos que queremos tener su nombre en versalitas. Llega el momento en que imprimimos nuestro libro y nos damos cuenta de que haber elegido versalitas no es lo más pertinente, sino que sería mejor ponerlos en negrita. No nos queda más remedio que buscar todas las versalitas que aparecen en nuestro texto, comprobar que se trata de versalitas que indican el numbre de un autor y sustituirlas por negrita ---¡Menudo trabajo engorroso!

Por el contrario, si, en vez de señalar en cada aparición que se necesitan versalitas, indicamos simplemente que se trata del nombre de un autor ---por ejemplo, escribiendo: \cs{autor}\verb|{Tertuliano}|--- sólo tendremos que cambiar una línea para indicar que queremos poner los nombres de los autores en negrita. Mejor todavía: podremos crear con gran facilidad un índice de autores\renvoi{indexauteur}.

\LaTeX{} ofrece algunos comandos sencillos para dar sentido: por ejemplo, los que hemos visto anteriormente para indicar los niveles de los títulos\renvoi{niveautitre}.

Vamos a presentar aquí algún otro comando y entorno para dar sentido. En el capítulo siguiente, indicaremos algunos específicos para las citas\renvoi{citertexte}. En un tercer capítulo, explicaremos cómo crear tus propios comandos\renvoi{creercommandes}, y presentaremos entonces la manera de dar forma.

\section{Comandos para dar sentido}

\subsection{Resaltar un texto}

Puede que quieras, en un momento dado, resaltar un fragmento de tu escrito. Para hacerlo, existe el comando \cs{emph}\marg{texto con énfasis}.
Ejemplo:

\begin{latexcode}
Se puede uno preguntar si ciertos textos apócrifos no solamente se han \emph{utilizado}, sino también \emph{leído} en la liturgia africana.
\end{latexcode}

Concretamente esto se convierte en cursiva. 

\begin{quotation}
Se puede uno preguntar si ciertos textos apócrifos no solamente se han \emph{utilizado}, sino también \emph{leído} en la liturgia africana.
\end{quotation}

Sin embargo, a diferencia de un comando que indicara directamente que hay que poner el texto en cursiva, este comando podría, si se quisiera, generar un resultado diferente, por ejemplo, ponerlo en color. 

Otra propiedad interesante es la gestión de las anidaciones: por defecto, un comando \cs{emph} dentro de otro comando \cs{emph} produce un texto con caracteres rectos.

\subsection{El paratexto: notas a pie de página y en el margen}

\LaTeX ofrece dos comandos para indicar paratextos\footnote{Dejamos de lado la cuestión de los aparatos críticos, asunto que trataremos más adelante\renvoi{reledmac}.}: para notas a pie de página y notas en el margen (la posición de éstas últimas varía, cuando es un documento a doble cara, en función de si la página es par o impar). Estos comandos son, respectivamente, \csp{footnote} y \csp{marginpar}.

\begin{latexcode}
Lorem\footnote{Nota a pie de página.} ipsum dolor amat.
Aliquam sagittis\marginpar{Nota marginal.} magna.
\end{latexcode}

\begin{attention}
    Resultaría tentador emplear este comando para citar una referencia bibliográfica en una nota a pie de página. De hecho, hay un comando apropiado para eso que estudiaremos a su debido tiempo\renvoi{footcite}.
\end{attention}
\begin{attention}
  Algunas malas lenguas dirán que se trata de dar forma y no de dar sentido. Tienen razón en parte, en la medida en que, en ocasiones, diferenciar entre dar forma y dar sentido no siempre está claro.
    
  Las personas muy perfeccionistas podrán definir sus propios comandos para diferenciar los distintos sentidos de una nota al margen o a pie de página.
\end{attention}

\begin{plusloins}
 Algunos prefieren incorporar las notas al final del texto.  Aunque no aprobamos esta opción, sí indicamos que se pueden producir con la ayuda del package \package{endnotes}.
\end{plusloins}

\subsection{Listas}

\LaTeX ofrece tres tipos de listas: las listas numeradas, las listas no numeradas y las listas de descripción.

\subsubsection{Listas numeradas}

Una lista numerada es un entorno      \enviro{enumerate}.
Cada elemento de la lista se marca mediante el comando \csp{item}.

\begin{latexcode}
\begin{enumerate}
    \item Primer elemento
    \item Segundo elemento
    \item Tercer elemento
\end{enumerate}
\end{latexcode}

\begin{quotation*}
\begin{enumerate}
    \item Primer elemento
    \item Segundo elemento
    \item Tercer elemento
\end{enumerate}
\end{quotation*}

\begin{plusloins}
Hay un package \package{etaremune} que ofrece el entorno \enviro{etaremune} para obtener una lista numerada al revés, es decir, con el número mayor al principio de la lista.

\end{plusloins}
\subsubsection{Listas no numeradas}

Una lista no numerada es un entorno \enviro{itemize}.
Cada elemento de la lista se marca mediante el comando \cs{item}.

\begin{latexcode}
\begin{itemize}
    \item Un elemento
    \item Otro
    \item Otro más
\end{itemize}
\end{latexcode}

\begin{quotation*}
\begin{itemize}
    \item Un elemento
    \item Otro
    \item Otro más
\end{itemize}
\end{quotation*}

\subsubsection{Listas de descripciones}

Una lista de descripciones hace coincidir valores uno por uno. Una lista de estas características puede resultar útil para léxicos, glosarios,cronologías, etc. Para cada pareja, el primer valor se pasa como argumento al comando \cs{item}. Las listas de definiciones son entornos \enviro{description}.


\begin{latexcode}
\begin{description}
    \item[325]Concilio de Nicea.
    \item[381]Concilio de Constantinopla.
    \item[431]Concilio de Éfeso.
\end{description}
\end{latexcode}

\begin{quotation*}
\begin{description}
    \item[325]Concilio de Nicea.
    \item[381]Concilio de Constantinopla.
    \item[431]Concilio de Éfeso.
\end{description}
\end{quotation*}

\subsection{Listas anidadas}

Se pueden anidar listas, sean del tipo que sean. Por defecto, no puede haber más de cuatro niveles de anidamiento.

\begin{latexcode}
\begin{itemize}
    \item Un elemento de primer nivel
    \begin{enumerate}
            \item Primer sub-elemento
            \item Segundo sub-elemento
    \end{enumerate}
    \item Otro elemento de primer nivel
\end{itemize}
\end{latexcode}

\begin{quotation*}
\begin{itemize}
   \item Un elemento de primer nivel
    \begin{enumerate}
            \item Primer sub-elemento
            \item Segundo sub-elemento
    \end{enumerate}
    \item Otro elemento de primer nivel
\end{itemize}
\end{quotation*}

\begin{plusloins}
En ocasiones es necesario personalizar el aspecto de las listas o incluso detener la numeración de una lista para retomarla después. Bertrand Masson ha escrito un excelente tutorial sobre el modo de utilizar el package \package{enumitem} para lograrlo\footcite{bebert_liste}. 
\end{plusloins}
