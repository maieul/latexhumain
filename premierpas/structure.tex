 \chapter{Cómo estructurar tu trabajo}
\begin{intro}
Tras haber descubierto los fundamentos de \LaTeX{}, es hora de aprender la forma de estructurar tu trabajo.
\end{intro}

\section{Distintos niveles de títulos}\label{niveautitre}

\LaTeX{} ofrece por defecto seis o siete niveles de títulos, según la clase seleccionada.
Para introducir un título en \LaTeX{} ---aparte del título del trabajo--- basta con emplear un comando de título que tiene la sintaxis siguiente: \csp{\meta{título}}\oarg{título breve}\marg{título largo}.

El título breve es opcional, como lo indica el hecho de que vaya entre corchetes\renvoi{syntaxecommande}. Sirve para el índice de contenidos\renvoi{toc} y,eventualmente, para los encabezados de las páginas\renvoi{entete}.

Evidentemente \csp{\meta{título}} hay que sustituirlo por el tipo de título. Veamos los niveles de título disponibles, desde el más general al más detallado. Cuanto más alto aparece un título en la jerarquía, más bajo es su número de nivel.



     \begin{longtable}{|l||l|l|}
    \hline     
     \headlongtable{Comando}                & \headlongtable{Sentido}                         & \headlongtable{Número de nivel}     \\
     \hline
    \endhead
    \hline
    \endfoot
     \csp{part}            & Título de parte             & -1     \\
     \csp{chapter}         & Título de capítulo         & 0           \\
    \csp{section}            & Título de apartado          & 1            \\
    \csp{subsection}        & Título de subapartado     & 2            \\
    \csp{subsubsection}    & Título de sub-subapartado& 3            \\
    \csp{paragraph}        & Título de párrafo         &4            \\

    \csp{subparagraph}        & Título de sub-párrafo & 5            \\
    \end{longtable}



Algunas observaciones importantes :
\begin{itemize}
\item el nivel \cs{chapter} sólo existe en la clase \classe{book};
\item a cada nivel de título se le asigna un número. Este número se utiliza en la presentación del índice de contenidos para definir su profundidad;\label{numeroniveau}\renvoi{tocdepth}
\item los niveles cuyos números son inferiores a 1 provocan un salto de página;
\item los niveles cuyos números son superiores a 3 no provocan un salto de párrafo. Los títulos se ponen en  \enquote{capital}. %lettrine?
\end{itemize}

\subsection{Títulos sin numerar}\label{titresansnumero}

Por defecto, todos los títulos se numeran automáticamente\footnote{Veremos más adelante cómo cambiar la numeración\renvoi{apparencecompteur}.}. Se puede obtener un título sin numerar escribiendo un asterisco tras el nombre dek comando: \csp{chapter*}\marg{Capítulo sin numerar}.


Sin embargo, un título sin numerar no se añadirá al índice de contenidos\renvoi{toc}. 

Para resolver este problema, hay que utilizar el comando:

\csp{addcontentsline}\verb|{toc}|\marg{1}\marg{2}, donde: \label{addcontentsline}

\begin{description}
    \item[\arg{1}] es el tipo de título;
    \item[\arg{2}] es el texto del título;
\end{description}

Un ejemplo es más clarificador:


\begin{latexcode}
\chapter*{Introducción}
\addcontentsline{toc}{chapter}{Introducción}
\end{latexcode}

%%%%%
\begin{attention}
<<<<<<< HEAD
Es necesario llamar sistemáticamente al comando \cs{addcontentsline} \emph{después} del comando de apartado, ya que éste último puede provocar un salto de página.
=======
Il faut systématiquement appeler la commande \cs{addcontentsline} \emph{après} la commande de sectionnement, cette dernière pouvant provoquer une coupure de page.
>>>>>>> master
\end{attention}

\begin{plusloins}
El lector atento sin duda se preguntará por qué es preciso escribir \verb|toc| como primer argumento. Éste se corresponde a la extensión del fichero que contendrá el índice de contenidos: remitimos al capítulo dedicado a este tema\renvoi{toc}.
\end{plusloins}

\section{Cómo estructurar los ficheros}\label{inclusion}

Hasta ahora, lo has puesto todo en un solo fichero. Una funcionalidad interesante de \LaTeX{} es la posibilidad de incorporar dentro de un fichero otros ficheros, para dividir así tu trabajo en varios ficheros, cada uno de los cuales contendrá solamente una parte del documento final.

Por ejemplo, se puede hacer un fichero por capítulo de una tesis, o incluso por texto citado en un ejemplar.. Sólo se compila un fichero \enquote{padre}, y este documento llama a los ficheros \enquote{hijos}.

¿Por qué hacerlo así?
\begin{itemize}
\item Para poder cambiar con más facilidad el orden de las partes. 
\item Para poder \enquote{reciclar} más fácilmente algunas partes.
\item Para poder compilar solamente algunas partes.
\end{itemize}

Concretamente, ¿cómo se hace?
\begin{enumerate}
\item El fichero \enquote{padre} tiene que empezar siempre por una llamada de clase y contener \cs{begin}\verb|{document}| y \cs{end}\verb|{document}|.
\item Los ficheros \enquote{hijos} no tienen que contener ninguna llamada de clase ni los comandos \cs{begin}\verb|{document}| y \cs{end}\verb|{document}|.
\item Se incluyen en el fichero \enquote{padre} mediante uno de los siguientes comandos:
\begin{itemize}
    \item \csp{include}\marg{ruta-al-fichero}, que implica siempre un salto de página.
    \item \csp{input}\marg{ruta-al-fichero}, que no implica salto de página.\label{input}
\end{itemize}
\end{enumerate}

El comando \cs{input}, a diferencia de  \cs{include}, también se le puede llamar en un fichero \enquote{hijo}, e incluso en un fichero \enquote{nieto} etc.

Te recomendamos poner el conjunto de llamadas a los packages en un fichero aparte. Así puedes disponer de un juego de packages para todos tus documentos: basta con llamar cada vez a este fichero.


\subsection{Cómo indicar la ruta del fichero}\label{chemin}

El concepto de ruta de archivo en informática se refiere a la estructura de carpetas en un equipo.

En \LaTeX{}, la ruta del fichero se cuenta a partir del fichero \enquote{padre}, el que se compila, incluido allí a la hora de hacer una inclusión en un fichero \enquote{hijo}.

Se indica la ruta del fichero separando cada carpeta con \verb|/|\footnote{Esta regla se aplica también en Windows, que tradicionalmente separa las carpetas con \texttt{\textbackslash} en las rutas.}. Así, si queremos incluir el fichero llamado \verb|c.tex| situado en la carpeta \verb|b|, y ubicado él en la carpeta \verb|a|, que se encuentra al lado del fichero \enquote{padre}, tenemos que escribir en nuestro fichero \enquote{padre}: \cs{input}\verb|{a/b/c}|
o bien
\cs{include}\verb|{a/b/c}|.

\begin{attention}

<<<<<<< HEAD
No es recomendable tener caracteres especiales en el nombre de las carpetas y de los ficheros.
=======
Il est déconseillé d'avoir des caractères spéciaux, notamment des accents et des espaces, dans le nom des dossiers et des fichiers. 

Il vaut mieux se limiter aux lettres latines non accentuées, aux chiffres arabes, au tiret standard (\enquote{-}) et au tiret du bas (\enquote{\_}).
>>>>>>> master
\end{attention}

Te recomendamos poner los ficheros \enquote{hijos} en una o varias subcarpetas.

\section{La clase \classenoidx{book}: estructura global del documento}\index[classe]{book}\label{sectionbook}

La clase \classe{book} ofrece, además de los niveles de títulos, una manera de estructurar tu trabajo en cuatro partes: preámbulos (prólogo, resumen, introducciones, etc.); cuerpo del trabajo; apéndices; herramientas de navegación (índice, glosarios, bibliografía, índice de contenidos, etc.). 

Cada una de estas partes se indica por medio de un comando inicial, respectivamente: \cs{frontmatter}; \cs{mainmatter}; \cs{appendix}\footnote{Este comando existe también en la clase \classe{article}.}; \cs{backmatter}.

Esta estructura en partes globales influye en la presentación de los números de página (romanos o arábigos) y en la numeración de los títulos.

Así, por defecto: \begin{description}
\item[\csp{frontmatter}] presenta títulos sin numerar pero que aparecen en el índice de contenidos. Además, los números de página aparecen en números romanos con minúsculas. 
\item[\csp{mainmatter}] presenta títulos numerados. La numeración de las páginas se reinicia y aparece con números arábigos.
\item[\csp{appendix}] presenta los números de los capítulos en forma de letras mayúsculas. El texto \forme{capítulo} se sustituye por \forme{apéndice}.
\item[\csp{backmatter}] elimina los números de los capítulos y presenta los capítulos en el índice de contenidos.
\end{description}

