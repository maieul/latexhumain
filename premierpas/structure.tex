\chapter{Commencer avec \logiciel{(Xe)LaTex}}

\begin{prealable}
Nous supposons que vous avez installé LaTex et un \concept{éditeur de texte} spécialisé en \logiciel{LaTex}. Voyez en annexe\renvoi{editeurlatex}.

La première chose à faire est de vérifier que ce logiciel de traitement de texte enregistre bien en \concept{utf-8}\footnote{Cela se trouve en général dans les préférences du logiciels, dans une rubrique \emph{enregistrement} \emph{encodage} : consultez le manuel de votre logiciel le cas échéant.}. Nous reviendrons plus loin\renvoi{unicode} sur l'intérêt d'un tel encodage, sachez simplement que cela permet d'utiliser des signes non latins\footnote{cyrilliques, grecs, sanskrits, hébraïques etc. Et même extra-terrestres.}.

\end{prealable}

\section{Un premier document}

Dans votre \notion{éditeur de texte}, tapez le code ci-dessous\footnote{Comme nous l'avons expliqué en introduction, la coloration que vous voyiez ici à un sens syntaxique : ne vous préoccupez pas de savoir comment cela apparaîtra dans votre éditeur, et ne pensez pas que cela apparaîtra comme cela une fois compilé.} puis cliquez sur le bouton de composition avec XeLaTex\footnote{Cela encore une fois dépend de votre \notion{éditeur de texte}. Pour le moment vous pouvez vous contenter de ce bouton, mais un jour vous devrez apprendre à faire quelques lignes de commandes : ne vous inquiétez pas, tout sera expliqué.}.

\begin{listing}[ht]
\inputminted{exemples/premierpas/structure/1.tex}
\caption{Un code pour découvrir \logiciel{(Xe)LaTex}}
\end{listing}

Regardez le \concept{PDF} obtenu, nous allons lire le code que vous avez copié et le commenter ligne par ligne, afin de comprendre les principes de base de \logiciel{LaTex}.
\FloatBarrier

\section{Structure d'un document \logiciel{LaTex}}

\subsection{La classe du document}
La première ligne \oldmint{latex}|\documentclass[12pt]{book}| déclare la \notion{classe} du document, ici \classe{book}. Une \notion{classe} de document est un \enquote{modèle}, un \enquote{style}, un \enquote{type} prédéfinis de document. Le choix de la \notion{classe} influencera entre autre :
\begin{itemize}
\item Le nombre de niveau de titre disponible
\item Les marges appliquées
\item Les pieds et entêtes de pages
\end{itemize}

Il existe en standard plusieurs \notion[classe]{classes} de documents : citons \classe{book}, pour rédiger un livre ; \classe{article} pour un article (si, si !), \classe{beamer} pour un présentation sous forme de diapositive à projeter. Dans ce manuel, nous aborderons essentiellement les deux premières. Nous aborderons aussi, dans un chapitre spécifique\renvoi{bredele}, la \notion{classe} \classe{Bredele}, qui n'est pas livré en standard avec \logiciel{LaTex}. Elle est a été écrite spécifiquement pour des travaux de type mémoire de Master ou thèse de doctorat suivant les normes universitaires fran\c caises les plus courantes.

Tout document \logiciel{(Xe)LaTex} doit commencer par une déclaration de \concept{classe}, avant toutes autres lignes. La syntaxe est la suivante : \oldmint{latex}|\documentclass[options]{classe}|

 Les options viennent spécifier certaines propriétés de la classe. Dans notre exemple, nous précisions que la taille de la police du texte courant doit être de 12pt. Vous pouvez préciser plusieurs options, en les séparant par des virgules. Voici quelques options disponibles et utiles en sciences humaines. 

\begin{description}
\item[10pt] pour une police de base en 10pt.
\item[11pt] pour une police de base en 11pt.
\item[12pt] pour une police de base en 12pt.
\item[onecolumn] pour un texte sur une seule colonne. C'est le cas par défaut sur les classes citées.
\item[twocolumn] pour un texte sur deux colonne.
\item[oneside] pour une impression en recto seulement.
\item[twoside] pour une impression en recto-verso. Cet argument et le précédent servent essentiellement pour la \notion{classe}\classe{book}. En effet, cette classe produira des marges gauches et droites de tailles différentes si le document est prévu pour une impression recto-verso. La taille des marges sera prévu selon que la page est recto ou verso.
\end{description}

Nous en préciserons d'autre au fur et à mesure du manuel, lorsque les notions requises auront été abordées.

\subsection{L'appel aux packages}

Voyons les deux lignes suivantes : 
\begin{minted}{latex}
\usepackage{xunicode}
\usepackage{polyglossia}
\end{minted}

Il s'agit, comme vous auriez pu le deviner, d'appel à des \concept[package]{packages}. Un \concept{package} est un ensemble de fichiers qui ajoutent des fonctionnalités à \logiciel{LaTex}, c'est l'équivalent d'un \concept{plugin} sous \logiciel{Firefox}. Le premier \concept{package} utilisé est \package{xunicode}. Il permet de gérer \concept[unicode]{l'unicode}, autrement appelé \concept{utf-8}\footnote{En réalité \concept{utf-8} n'est pas tout à fait \concept{unicode}, mais simplement un variante de ce dernier. Toutefois, pour le besoin du document, nous assimilerons les deux.} ; c'est cela qui nous permettra d'avoir des caractères non-latin\renvoi{utf-8}.

Le second permet de gérer facilement un document multilingue\renvoi{i18n} et les changements typographiques que cela implique.

Certains \concept[package]{packages} peuvent recevoir des options qui modifieront leur comportement standard. La syntaxe est alors : \oldmint{latex}|\usepackage[options]{package}|. 

Tout au long de cet ouvrage, nous aborderons divers packages.

\subsection{Le fran\c cais, langue par défaut}

Tout de suite après, la ligne \oldmint{latex}|\setmainlanguage{french}| indique nous utilisons comme langue principale du document le fran\c cais. \renvoi{i18n}, et que donc le compositeur de texte devra prendre en compte la typographie fran\c caise. Cette ligne n'est compréhensible par le \concept{compilateur} que parce que nous avons chargé au préalable \package{polyglossia}.

\begin{anedocte}
Vous entendrez peut-être parler du \concept{package} \package{babel}. Ce package est très souvent utilisé à la place de \package{polyglossia}, notamment parce qu'il est plus ancien. Toutefois, nous avons choisis pour notre part de nous limiter à \package{polyglossia}.
\end{anedocte}
