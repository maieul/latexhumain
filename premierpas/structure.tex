\chapter{Commencer avec \logiciel{(Xe)LaTex}}

\begin{prealable}
Nous supposons que vous avez installé LaTex et un \concept{éditeur de texte} spécialisé en \logiciel{LaTex}. Voyez en annexe\renvoi{editeurlatex}.

La première chose à faire est de vérifier que ce logiciel de traitement de texte enregistre bien en \concept{utf-8}. \footnote{Cela se trouve en général dans les préférences du logiciels, dans une rubrique \emph{enregistrement} \emph{encodage} : consultez le manuel de votre logiciel le cas échéant.} Nous reviendrons plus loin\renvoi{unicode} sur l'intérêt d'un tel encodage, sachez simplement que cela permet d'utiliser des signes non latins (cyrilliques, grecs, sanskrits, hébraïques etc. Et même extra-terrestres.).

\end{prealable}

\section{Un premier document}


