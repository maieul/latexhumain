\chapter{Dar sentido (2): el arte de citar en LaTeX}\label{citertexte}

\begin{intro}
Abordaremos en este capítulo las citas \emph{explícitas y textuales}, es decir, aquéllas en las que el autor del trabajo no se limita a remitir a una fuente o a un estudio, sino que cita pasajes de esa fuente.

Una cita puede hacerse de dos maneras: en el cuerpo del texto, donde normalmente se pone entre comillas, o bien en un párrafo específico. En ese caso, se suele presentar con marcas tipográficas particulares: cambio del tamaño de la fuente, del margen, etc.

Hemos visto antes que había que distinguir entre sentido y forma\renvoi{sensforme}. Así que vamos a presentar aquí los comandos que sirven para marcar las citas.

\end{intro}
\begin{attention}
Toda cita tiene que ir acompañada de una referencia, normalmente en una nota a pie de página. Sin embargo, no todas las referencias acompañan necesariamente a una cita textual. Por este motivo, para la gestión de las referencias bibliográficas remitimos al apartado dedicado a ellas.
\end{attention}

\section{Citas en el cuerpo del texto}\label{guillemets}

Las citas en el cuerpo de un texto suelen ponerse entre comillas francesas: \verb|«»|. Cuando se cita un texto que cita otro texto, la cita dentro de la cita se pone entre comillas redondas \verb|“”|. 

\begin{quotation}
    Como dice con mucha razón xxx : \enquote{Cuando yyy declara \enquote{zzz} no declara nada en absoluto}.
\end{quotation}

Los teclados que tenemos en nuestros ordenadores no suelen tener, por lo general, un acceso directo a las comillas inglesas\footnote{Una importante excepción es la disposición del teclado Bépo.}(\verb|"|). 
La mayoría de los programas WYSIWYG convierten automáticamente estas comillas en comillas francesas. Son raros los editores de texto que lo permiten\footnote{Es, a nuestro juicio, uno de los motivos que hace que el sitio web de un diario nacional considerado \enquote{de referencia} no utilice comillas francesas en su página de bienvenida (con fecha de 10 de abril de 2011), ya que los redactores no dedican tiempo a teclear las complicadas combinaciones de teclas necesarias para la inclusión de comillas francesas.}. 

Además, en virtud del principio de separación del sentido y de la forma, aludido anteriormente\renvoi{sensforme}, resulta más pertinente emplear un comando específico para indicar que se trata de una cita en el cuerpo del texto.

Así que vamos a utilizar el package \package{csquotes} que ofrece comandos para las citas.

El package tiene un primer comando muy útil: \csp{enquote}\marg{cita}, que sirve para las citas en el cuerpo del texto.

\begin{latexcode}
Como dice con mucha razón xxx: \enquote{Cuando yyy declara \enquote{zzz} no declara nada en absoluto}.
\end{latexcode}


\begin{quotation}
Como dice con mucha razón xxx: \enquote{Cuando yyy declara \enquote{zzz} no declara nada en absoluto}.
\end{quotation}

Observamos que \package{csquotes} se encarga automáticamente de seleccionar las comillas correctas. Por defecto, no se pueden anidar más de dos niveles de citas. No obstante, una opción del package permite tener más niveles. Por ejemplo, para tener tres: 

\begin{latexcode}
\usepackage[maxlevel=3]{csquotes}
\end{latexcode}

El package ofrece otras opciones y comandos: consulta el manual\footcite{csquotes}.

\section{Citas en un bloque aparte}


LaTeX ofrece tres entornos básicos para citar en un bloque aparte.

\subsection{El entorno \environoidx{quote}}\index[enviro]{quote}

Está pensado para citas breves de un párrafo\footcite[6]{AugustinSermo296}.

\inputminted{exemples/premierpas/citation/rome.tex}

Este código produce el resultado siguiente:

\begin{quote}
Le corps de Pierre gît à Rome, disent les hommes,
le corps de Paul gît à Rome, le corps de Laurent aussi,
les corps d'autres martyrs y gisent,
mais Rome est misérable,
elle est dévastée, affligée, saccagée, incendiée.
\end{quote}


\subsection{El entorno \environoidx{quotation}}\index[enviro]{quotation}

Está pensado para citas más largas\footcite{BreveHippone}.



\inputminted{exemples/premierpas/citation/concile.tex}


\begin{quotation}
Que nada, salvo las escrituras canónicas se lea en la iglesia bajo el nombre de escrituras divinas.

Las escrituras canónicas son: Génsis, Éxodo, Levítico, Números, Deuteronomio, Josué de Nun, Jueces, Rut, 4~libros de los reinos, 3~libros de los paralipómenos, Job
 Salmos, 5~libros de Salomón, 12 libros de los profetas menores, del mismo Isaías, Jeremías, Ezequiel, Daniel, Tobías, Judit, Ester, 2~libros de Esdras, 2~libros de los Macabeos.

Del Nuevo Testamento son:
4~evangelios, un libro de los hechos de los apóstoles, 14~cartas del apóstol Pablo, 2~de Pedro, 3~de Juan, 1~de Judas, 1~de Santiago, el apocalipsis de Juan.

Que se consulte a la Iglesia de ultramar para la confirmación de este canon.

Además, que se permita leer las pasiones de los mártires cuando se celebre su aniversario.
\end{quotation}



\begin{attention}
El fondo gris que ves aquí es una particularidad del libro que tienes a la vista. El entorno \enviro{quotation} estándar no tiene dicho fondo.
\end{attention}

\subsection{El entorno\environoidx{verse} y el package \packagenoidx{verse}}\index[enviro]{verse}\index[pkg]{verse}


El entorno \enviro{verse} permite citar de manera tápida y sencilla poemas, y gestiona bastante bien los casos de corte de versos demasiado largos. Todos los versos, salvo el último, tienen que terminar con \verb|\\|. Si el poema consta de varias estrofas, basta con dejar una línea en blanco entre cada una de ellas. No hay que poner \verb|\\| al final del último verso de cada estrofa.    

Este entorno es bastante limitado: no permite numerar ni sangrar los versos ni tampoco añadir un título
Cet environnement est très limité: il ne permet pas de numéroter ni d'indenter les vers, ni encore de rajouter un titre. Si te ves en la obligación de citar versos con frecuencia, es mejor llamar en el preámbulo al package \package{verse}\footcite{verse}. La cita del poema se hace de la misma forma.

Con este package hay que indicar \csp{poemlines}\marg{n} para numerar los versos citados:  \arg{n} 
define la frecuencia con la que se numeran los versos. 

Citemos, por ejemplo, un poema completo, cuyos versos están numerados, como es una costumbre bastante tradicional, de cinco en cinco. Al escribir esto\footcite{demain}:

\inputminted{exemples/premierpas/citation/demain.tex}

obtenemos esto:


\begin{verse}
\poemlines{5}
 Pisó las calles de Madrid el fiero \\
 monóculo galán de Galatea\\
y, cual suele tejer bárbara aldea\\
 soga de gozques contra forastero,
 
 rígido un bachiller, otro severo,\\
crítica turba al fin, si no pigmea,\\
 su diente afila y su veneno emplea\\
en el disforme cíclope cabrero.
 
A pesar del lucero de su frente,\\
lo hacen obscuro, y él, en dos razones\\
que en dos truenos libró de su occidente,
 
«Si quieren -respondió- los pedantones\\
luz nueva en hemisferio diferente,\\
den su memorïal a mis calzones».

%Góngora, Sonetos 288 (1615).

\end{verse}


Si no se cita un poema completo, sino sólo un pasaje, hay que indicar en qué verso empieza el pasaje citado, para que la numeración de los versos que se citan sea la correcta.

Se utiliza el comando \csp{setverselinenums}\marg{primer verso}\marg{primer verso numerado}: \arg{primer verso} indica el número del primer verso que se cita y \arg{primer verso numerado} dónde tiene que empezar la numeración. 

Así, \cs{setverselinenums}\marg{12}\marg{15} indica que el extracto citado comienza en el verso doce del poema y que el primer verso numerado será el quince. 
Con \cs{setverselinenums}\marg{12}\marg{12}, el primer verso citado será el primer verso numerado.

Si sólo se cita la tercera estrofa, por ejemplo, hay que escribir:

\inputminted{exemples/premierpas/citation/demainfin.tex}

para obtener, con la numeración correcta:


\begin{verse}
\poemlines{5}
\setverselinenums{9}{10}


«Si quieren -respondió- los pedantones\\
luz nueva en hemisferio diferente,\\
den su memorïal a mis calzones».

\end{verse}



\begin{plusloins}
En la tipografía francesa se suele poner un corchete derecho [ al comienzo de un verso cortado. Ni el entorno ni el package \package{verse} siguen esta regla. Para obtener el corchete derecho, hay que cargar el package  \package{gmverse} poniéndole la opción \option{squarebr} :

\begin{latexcode}
\usepackage}[squarebr]{gmverse}
\end{latexcode}

Depués hay que insertar, una vez iniciado el entorno \enviro{verse}, el comando \csp{versehangrightsquare}.
\end{plusloins}

El package \package{verse} permite también sangrar de manera muy flexible una estrofa. Se indica con el comando \csp{indentpattern}\marg{$n_1 n_2 n_x$} el sangrado de cada verso que hay en la estrofa incluida dentro de un entorno \enviro{patverse}: \arg{$n_1$} corresponde al primer verso, \arg{$n_2$} al segundo y así sucesivamente. El primer verso nunca se sangra, pero basta con poner delante \csp{vin}. Con el código siguiente:

\inputminted{exemples/premierpas/citation/apple.tex}

se obtiene\footcite{EdwardLear}:
  

\begin{verse} 
\poemlines{1}
\indentpattern{010110} 
\begin{patverse} 

\vin There was a young lady of Ryde \\
 Who ate some apples and died.  \\
 The apples fermented \\
 Inside the lamented \\
 And made cider inside her inside. \\
\end{patverse}  
\end{verse}



 Si se necesita repetir un sangrado a lo largo de un poema, se sustituye \enviro{patverse} por \enviro{patverse*}. Así, para citar un extracto largo en hexámetros dactílicos, en vez de indicar el sangrado de cada verso, basta con indicar \cs{indentpattern}\verb|{01}| e incluir el poema completo dentro del entorno \enviro{patverse*} para obtener un sangrado en cada segundo verso.
 
El package \package{verse} ofrece además otras posibilidades que exceden la simple cita de poesía a lo largo de un texto. Al igual que otro package, \package{poemscol}, es una verdadera herramienta para editar poesía\footcites(Remitimos aquí a los manuales de estos packages){verse}{poemscol}. Por el contrario, ni uno ni otro permiten la elaboración de ediciones bilingües: para ello hay que utilizar los packages \package{reledmac} y \package{reledpar}\renvoi{reledmac}. 


\section{Citas truncadas y modificadas}

El package \package{csquotes} ofrece dos comandos específicos para indicar una cita truncada o modificada.

\subsection{Citas truncadas}

El comando \csp{textelp}\arg{texto} señala un texto truncado. El argumento \arg{texto} se incluye después de la interrupción, entre corchetes (por defecto); para no incluir nada, se deja vacío.


\inputminted{exemples/premierpas/citation/conciletronque.tex}


\begin{quotation}
Que nada, salvo las escrituras canónicas se lea en la iglesia bajo el nombre de escrituras divinas.
\textelp{Sigue la lista de las escrituras canónicas.}

Que se consulte a la Iglesia de ultramar para la confirmación de este canon.

Además, que se permita leer las pasiones de los mártires cuando se celebre su aniversario.
\end{quotation}


\begin{plusloins}
Se puede decidir la forma de señalar la interrupción y el añadido: consulta el manual\footcite{csquotes_ellipses}.
\end{plusloins}

\subsection{Citas modificadas}

Para señalar una modificación en una cita se utiliza \csp{textins}\marg{texto modificado}.

\inputminted{exemples/premierpas/citation/modif.tex}

\begin{quotation}
Como decía muy justamente xxx:
 \enquote{Aunque yyy \textins{ha declarado}
\enquote{zzz}  \textins{no ha declarado nada} en absoluto}.
\end{quotation}


El comando \csp{textins*} es una variante, que permite declarar los cambios menores necesarios para el nuevo contexto del enunciado: mayúsculas, cambio de persona, etc.
