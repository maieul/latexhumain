\section{Création de tableau}

\begin{attention}
La création de tableau en \LaTeX nécessite  une extrême rigueur. 
C'est à nos yeux l'un des éléments les plus complexe de \LaTeX. C'est pourquoi nous ne ferons ici qu'un bref aperçu : on trouvera aisément des exemples plus détaillés dans d'autres ouvrages ou sur internet\footcite[On pourra consulter des fichiers assez didactiques :][]{bebert_tableaux}.

\end{attention}

\subsection{Syntaxe de base}

La création d'un tableau simple se fait grâce à l'environnement \enviro{tabular}, d'après la syntaxe suivante :

\begin{minted}{latex}
\begin{tabular}{<type de colonnes>}
	Cellule & Cellule & Cellule \\
	Cellule & Cellule & Cellulle \\
\end{tabular}
\end{minted}

Le caractère \verb|&| marque un changement de cellule. Le \verb|\\| marque un changement de ligne.

L'argument \arg{type de colonnes} de l'environnement \enviro{tabular} sert à préciser :
\begin{enumerate}
	\item Le nombre de cellules.
	\item Le formatage du texte à l'intérieur des cellules.
	\item Éventuellement les séparateurs verticaux des cellules.
	
\end{enumerate}

Il existe trois types possibles de colonnes,  qui précisent l'alignement des textes : \label{typecolonnes}

\begin{itemize}
	\item \verb|l| Alignement à gauche.
	\item \verb|c| Centré.
	\item \verb|r| Alignement à droite.
\end{itemize}

Supposons que nous ayons un tableau à trois colonnes. Nous souhaitons que les cellules de la première colonnes soient alignées à gauche, celles de la seconde centrées et celles de la dernière alignées à droite\footnote{Il s'agit ici d'une dérogation au principe de séparation de fond et de forme. Toutefois le lecteur l'ayant assimulé pourra aisément contourner cette dérogation et revenir dans le droit chemin}.  
Il nous faut  écrire :

\inputminted{exemples/premierpas/nontextuels/tableau3colonnes.tex}


\begin{tabular}{lcr}
Première cellule à gauche & Première cellule au centre & Première cellule à droite \\
Seconde cellule à gauche & Seconde cellule au centre & Seconde cellule à droite \\
\end{tabular}



Par défaut, le tableau va prendre la largeur minimale nécessaire à son affichage. 

On peut utiliser l'environnement \enviro{tabular*} qui propose un argument supplémentaire, correspondant à la largeur totale du tableau :

\begin{minted}{latex}
\begin{tabular*}{<largeur du tableau>}{<type de colonnes>}
	Cellule & Cellule & Cellule \\
	Cellule & Cellule & Cellulle \\
\end{tabular}
\end{minted}


Toutefois il peut être pratique d'indiquer des tailles de colonnes : en particulier lorsque nos cellules contiennent des paragraphes.

Pour ce faire trois types supplémentaires de colonne nous sont proposés par le package \package{array}\footnote{En réalité le premier type ne nécessite pas le package}. Ces types, à la différences des précédents, déterminent une largeur de colonne. Ils nécessitent pour être utilisés de connaître les unités de mesure de \LaTeX\renvoi{unite}. 

Ces trois types de colonnes sont :
\begin{description}
	\item[\verb|p{largeur}|] texte justifié. La plus haute ligne du texte se trouve sur la ligne de base.
	\item[\verb|m{largeur}|] texte justifié. La ligne du milieu du texte se trouve sur la ligne de base.
	\item[\verb|b{largeur}|] texte justifié. La plus basse ligne du texte se trouve sur la ligne de base.
\end{description}

La ligne de base est une ligne spécifique, correspondand, \emph{grosso-modo} au milieu d'une cellule. Exemple :

\inputminted{exemples/premierpas/nontextuels/tableau3colonnesarray.tex}
\begin{tabular}{p{3cm} | m{3cm} | b{3cm} | c}
	Long texte dont la plus haute ligne se trouve sur la ligne de base. Lorem ipsum dolor amat etc. 
	& 
	Long texte dont la ligne du milieu se trouve sur la ligne de base. Lorem ipsum dolor amat etc. 
	&
	 Long texte dont la plus basse ligne se trouve sur la ligne de base. Lorem ipsum dolor amat etc. 
	 &
	 Ligne de base
\end{tabular}




\begin{plusloins}

Il est possible de créer ses propres types de colonne. Afin par exemple d'en avoir avec une typographie particulière. Consultez le manuel du package \package{array}\footcite{array_colonnes}.

\end{plusloins}
\begin{plusloins}
On peut utiliser le package \package{slahshbox} afin d'avoir un cellule divisée en deux dans un sens diagonal, pour les tableaux à doubles entrées.
\end{plusloins}
\begin{plusloins}
Le package \package{tabularx} propose un environnement \package{tabularx} permettant de calculer automatiquement la taille des colonnes.
\end{plusloins}
\subsubsection{Séparation des cellules}

On peut tracer des filets de séparations des cellules : pour un filet  horizontal, il suffit d'utiliser la commande \cs{hline}. Pour un filet  vertical, il faut mettre le caractère \verb+|+\footnote{Ce caractère, appelé \forme{pipe} ne doit pas être confondu avec un +l+.} entre les types de cellules.

On peut tracer des filets horizontaux plus petits en utilisant la commande \csp{cline}\verb|{<col1>-<col2>}|, où \arg{col1} et \arg{col2} doivent être remplacés par les numéros de colonnes.

\inputminted{exemples/premierpas/nontextuels/tableauseparateur.tex}
\begin{tabular}{l||c|r}
\hline
A & B & C \\
\hline
A1 & B1 & B2\\

\end{tabular}


\subsubsection{Fusion de cellules}

Il est possible de fusionner des cellules. Pour une fusion de colonnes, il faut utiliser la commande suivante :
\csp{multicolumn}\marg{nombcol}\marg{type}\marg{contenu}.

\begin{description}

\item[\arg{nombcol}] est le nombre de colonne à fusionner.
\item[\arg{type}] est le type de colonne qui en résulte, selon la syntaxe vu plus haut.
\item[\arg{contenu}] est le contenu de la ligne fusionné.
\end{description}

Pour une fusion de lignes, il faut charger le package \package{multirow}. Celui-ci propose une commande \cs{multirow} dont voici la syntaxe simplifiée\footcite[Simplifiée car nous ne présentons pas les arguments optionnels : nous renvoyons à la documentation :][]{} :
\csp{multirow}\marg{lignes}\marg{largeur}\marg{contenu}.


\begin{description}
\item[\arg{lignes}] est le nombre de lignes concernées par la fusion.
\item[\arg{largeur}] est la largeur du texte. On peut mettre \verb|*| pour laisser \LaTeX{} s'en charger.
\item[\arg{contenu}] est le contenu de la cellule issue de la fusion.
\end{description}

\inputminted{exemples/premierpas/nontextuels/tableaufusion.tex}
\begin{tabular}{|c|c|c|}
\hline
	\multicolumn{3}{|c|}{Colonnes fusionnées = ABC0} \\
	\hline
	A1 & B1 & C1 \\
	\hline
	\multirow{2}{*}{Lignes fusionnés = A2-A3}
	& B2 & C2 \\
	\cline{2-3}
	& B3 & C3 \\
\hline
\end{tabular}

On remarquera les cellules vides correspondante aux anciennes cellules A2 et A3. La commande \cs{multirow} ne modifie pas la structure du tableau : elle indique simplement que le texte se situe sur plusieurs colonnes.

\subsection{Long tableau}

Si vous fabriquez ainsi un long tableau, vous constaterez que les cellules dépassent dans la marge inférieure, jusqu'à sortir de la page, au lieu, comme il semblerait logique, de continuer sur les pages suivantes.

Pour éviter cela, il faut donc utiliser l'environnement \enviro{longtable} du package éponyme\sindex[pkg]{longtable}.

\begin{minted}{latex}
\usepackage{longtable}
…
\begin{longtable}{types colonnes}
…
\end{longtable}
\end{minted}

La syntaxe d'un environnement \enviro{longtable} est identique à celle d'un environnement \enviro{tabular}. Toutefois le package propose deux  commandes supplémentaires pour préciser les en-têtes et pieds de tableau, qui sont répétés sur chaque pages. Ces commandes sont, respectivement, \csp{endhead} et \csp{endfoot}. Il existe aussi aussi une commande \csp{endfirsthead} pour préciser l'en-tête  spécifique à la première page du tableau et une commande \csp{endlastfoot} pour préciser le pied  spécifique à la dernière page du tableau.

Ces quatre commandes sont optionnelles, elles  doivent s'utiliser en début de tableau : il faut placer les éléments correspondants \emph{avant} chacune des commandes, de la manière suivante :

\begin{minted}{latex}
\begin{longtable}{types colonnes}
Cellules d'en-tête.
\endhead
Cellules d'en-tête pour la première page.
\endfirsthead
Cellules de pied.
\endfoot
Cellules de pied pour la dernière page.
\endlastfoot
Cellules de corps de tableau.
\end{longtable}
\end{minted}

\begin{attention}
Un élément important avec \enviro{longtable} est la nécessité de compiler deux fois le fichier \ext{tex}. 

En effet, \XeLaTeX{} --- Comme d'ailleurs \LaTeX{} --- n'arrive pas à déterminer la taille des colonnes à la première compilation : il procède à un calcul page par page. Cependant il stocke la taille maximale des colonnes dans un fichier temporaire, portant l'extension \ext{aux}. Lors de la seconde compilation, il se sert de ce fichier temporaire pour indiquer la taille des colonnes sur chacune des pages.
\end{attention}
\subsection{Création depuis un fichier \ext{csv}}

Évidemment la syntaxe des tableaux est assez compliquée et risque d'entraîner des erreurs. Il est toutefois possible d'utiliser un fichier au format \ext{csv}, que n'importe quel tableur est capable de produire. Il faut pour cela utiliser le package \package{csvtools}

Le format \ext{csv} est un format standard de données tabulaires. Malheureusement il ne dispose ni de mise en forme ni de possibilité de fusion de cellule.

\begin{plusloins}Une solution peut être alors d'utiliser la macro Calc2LaTeX pour le logiciel OpenOffice.org, mais nous n'avons l'avons pas testée.
\end{plusloins}



Toutefois pour des données simples, comme par exemple un tableau statistiques, il suffit amplement. 
Il vous faut donc exporter votre tableau au format CSV, en choisissant la virgule comme séparateur de champs. La première ligne correspondant à l'en-tête. Certains logiciels d'export au format CSV ne laissent pas le choix quant aux séparateurs de champs. Dans ce cas, il faut utiliser la commande \csp{setcsvseparator}\marg{separateur} pour définir les champs. Ainsi si le séparateur est un point-virgule : 
\cs{setcsvseparator}\verb|{;}|.

Il est possible de créer à partir d'un fichier \ext{csv} un environnement \enviro{tabular} ou\enviro{longtable}.



Le premier se crée  avec la commande
\csp{CSVtotabular}\marg{fichier}\marg{type col.}\marg{premiere}\marg{milieu}\marg{dernier}.

Le second avec la commande
\csp{CSVtolongtable}\marg{fichier}\marg{type col.}\marg{premiere}\marg{milieu}\marg{dernier}.


\begin{description}
\item[\arg{fichier}]est le nom du fichier \ext{csv}. La méthode pour indiquer le chemin du fichier est la même que d'habitude.\renvoi{chemin}
\item[\arg{type col.}] indique les alignements de colonne, comme vus précédemment\renvoi{typecolonnes}.
\item[\arg{premiere}] indique le contenu de la première ligne du  tableau que vous allez généré.
\item[\arg{milieu}] indique le contenu des lignes du milieu.
\item[\arg{dernier}] indique le contenu de la dernière ligne.
\end{description}

On utilise la commande \csp{insertbyname}\marg{nom} pour insérer le contenu de la colonne \arg{nom}.

Exemple avec le fichier \fichier{csv.csv} :

\inputminted{exemples/premierpas/nontextuels/csv.csv}

\inputminted{exemples/premierpas/nontextuels/csvtolongtable.tex}


Le résultat est le tableau \ref{csvtolongtable}\renvoi{csvtolongtable}.

\begin{table}[h]
\setcsvseparator{;}
\CSVtolongtable{exemples/premierpas/nontextuels/csv.csv}{|l|c|r|}{
\hline
\emph{Colonne A}	& \emph{Colonne B} & \emph{Colonne C} \\ 
\hline}
{\insertbyname{A}&\insertbyname{B}&\insertbyname{C}\\}
{\emph{\insertbyname{A}}&\emph{\insertbyname{B}}&
\emph{\insertbyname{C}}\\
\hline}
\caption{Exemple d'utilisation de la commande \cs{csvtolongtable}}
\label{csvtolongtable}
\end{table}


\begin{plusloins}
Pour afficher un graphique statistique, nous recommandons d'utiliser le package \package{TikZ}\renvoi{TikZ}. Toutefois la maîtrise de cet outils peut être relativement complexe. C'est pourquoi nous listons ici quelques packages utiles : 
\begin{itemize}
\item Le package \package{csvpie} propose une fonction pour tracer un diagramme circulaire à partir d'un fichier \ext{csv}\footcite[Sa documentation se trouve dans celle du package \package{csvtools} :][]{csvtools_pie}. 
\item Le package \package{datatool} proposent un certain nombre de graphiques statistiques. Il nécessite d'enregistrer les données selon une forme liée au package. Nous renvoyons à sa documentation\footnote{datatool}.
\end{itemize}
\end{plusloins}
