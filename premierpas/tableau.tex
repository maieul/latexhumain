\section{Création de tableau}

La création de tableau en \LaTeX bien que de prime abord simple nécessite pour autant une extrême rigueur.

\subsection{Création depuis un fichier \ext{cvs}}

Évidemment une telle syntaxe est bien compliquée et risque d'entraîner des erreurs. Il est toutefois possible d'utiliser un fichier au format \ext{cvs}, que n'importe quel tableur est capable de produire.

Le format \ext{cvs} est un format standard de données tabulaires. Malheureusement il ne dispose ni de mise en forme ni de possibilité de fusion de cellule.

\begin{anedocte}Une solution peut être alors d'utiliser la macro Calc2LaTeX pour le logiciel OpenOffice.org, mais nous n'avons pas testé cette macro.
\end{anedocte}

Il vous donc exporter votre tableau au format cvs. La première ligne correspondant à l'entête.

Toutefois pour des données simples, comme par exemple un tableau statistiques, il suffit amplement. Il est possible de créer à partir d'un fichier \ext{cvs} un environnement \enviro{tabular} ou\enviro{longtable}.

Le premier se crée  avec la commande

\begin{minted}{latex}
\CSVtotabular{fichier}{alignement}{premiere}{milieu}{dernier}
\end{minted}

le second avec la commande

\begin{minted}{latex}
\CSVtolongtable{fichier}{alignement}{premiere}{milieu}{dernier}
\end{minted}

\begin{description}
\item[fichier]est le nom du fichier \ext{.cvs}. La méthode pour indiquer le chemin du fichier est la même que habituelle.\renvoi{chemin}
\item[alignement] indique les alignements de colonnes, comme vu précédemment\renvoi{alignement}.
\item[premiere] indique le contenu de la première ligne (dans le tableau que vous allez généré).
\item[milieu] indique le contenu des lignes du milieu.
\item[dernier] indique le contenu de la dernière ligne.
\end{description}

Pour les arguments 

Exemple avec le fichier \fichier{csv.csv} :


\inputminted{exemples/premierpas/nontextuels/cvstolongtable.tex}


Le résultat est le tableau \ref{cvstolongtable}\renvoi{cvstolongtable}.

\begin{table}[p]
\CSVtolongtable{exemples/premierpas/nontextuels/csv.csv}{l|c|r|}{
\emph{Colonne A}	& \emph{Colonne B} & \emph{Colonne C} \\ 
\hline}
{\insertbyname{A}&\insertbyname{B}&\insertbyname{C}\\}
{\emph{\insertbyname{A}}&\emph{\insertbyname{B}}&\emph{\insertbyname{C}}\\}
\caption{Exemple d'utilisation de la commande \commande{cvstolongtable}}
\label{cvstolongtable}
\end{table}


\begin{anedocte}Pour afficher un graphique statistique, nous recommandons d'utiliser le package \package{TikZ}\renvoi{TikZ} et les nombreux packages dérivés. Toutefois le \package{cvstools} possède une fonction pour tracer un diagramme circulaire à partir d'un fichier \ext{cvs}. Nous renvoyons à sa documentation.

\end{anedocte}
