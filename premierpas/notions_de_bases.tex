\chapter[Comenzar con XeLaTeX]{Comenzar con \XeLaTeX}\label{commencer}

\begin{intro}
Supongamos que tienes instalado \LaTeX\renvoi{install} y un editor de texto\renvoi{logiciels} especializado en \LaTeX. Consúltalo en el anexo. 

Lo primero que hay que hacer es comprobar que este programa de tratamiento de texto permite escribir bien en UTF-8\footnote{La configuración se encuentra normalmente en las preferencias del programa, en un apartado \emph{registro} o \emph{codificación}: consulta el manual de tu editor en todo caso.}. Volveremos más adelante\renvoi{utf8} sobre el interés que tiene una codificación de este tipo, por de pronto, has de saber que permite emplear signos no latinos\footnote{Cirílicos, griegos, sánscritos, hebreos, etc. E incluso extraterrestres.}.

\end{intro}

\section{Un primer documento}

Teclea en tu editor de texto el código siguiente\footnote{Como hemos explicado en la introducción (p.~\pageref{colorationsyntax}), el resaltado que ves aquí, si lees la versión informática de este libro, tiene un sentido sintáctico: no te preocupes si es distinta en tu editor y no creas que tu texto aparecerá así una vez compilado.}, después pulsa el botón de compilación con \XeLaTeX \footnote{Su ubicación depende de tu editor de texto. De momento, puedes conformarte con ese botón, pero algún día tendrás que aprender a escribir algunas líneas de comandos: no te preocupes, lo explicaremos todo.}:

\begin{attention}
Si estás con Mac, los caracteres necesarios para el uso de  \LaTeX no se ven directamente en tu teclado. Hemos recogido una lista en un anexo sobre cómo introducirlos\renvoi{claviermac}.
\end{attention}
\inputminted{exemples/premierpas/notions/1.tex}

Observa el PDF que obtenemos: para comprender los principios básicos de \LaTeX, vamos a leer el código que has compiado y a comentarlo línea a línea


\section{Estructura de un documento \LaTeX}

\subsection{La clase de documento}
La primera línea \csp{documentclass}\verb|[12pt,a4paper]{book}| declara la clase de cocumento, aquí \classe{book}. Una clase corresponde a una elección editorial ---formato y organización general del documento. La elección de la clase influye entre otros aspectos:
\begin{itemize}
\item En el número de niveles de título disponibles.
\item Los márgenes aplicados.
\item Los encabezados y pies de página.
\end{itemize}

Hay varias clases de documentos estándar: citemos\classe{book}, para redactar un libro; \classe{article} para un artículo (¡sí, sí!); \classe{beamer} para elaborar una presentación con diapostivas que se proyectan. En esta obra, abordaremos principalmente las dos primeras, pero también haremos una breve presentación de  \classe{beamer}\renvoi{beamer}.


Todo documento \LaTeX  ha de comenzar con una declaración de clase, antes de cualquier otra línea.
La sintaxis es: 
\cs{documentclass}\oarg{opciones}\marg{clase}.

\begin{attention}
De ahora en adelante en este libro, cualquier texto situado entre corchetes (\arg{así}), tienes que sustituirlo en tu fichero \ext{tex} con un valor textual.
\end{attention}

Las opciones especifican algunas propiedades de la clase. En nuestro ejemplo, indicamos que el tamaño de la fuente del texto que escribimos debe ser de 12~pt y que el formato del papel es A4. Puedes indicar más opciones separándolas mediante comas.

\label{optionsclasse}

Veamos algunas opciones disponibles y útiles en humanidades:

\begin{description}
\item[10pt] para una fuente básica de 10~pt.
\item[11pt] para una fuente básica de 11~pt.
\item[12pt] para una fuente básica de 12~pt.
\item[onecolumn] para un texto con una sola columna. Es la opción por defecto de las clases mencionadas.
\item[twocolumn] para un texto con dos columnas.
\item[oneside] para una impresión a una sola cara. \label{nbsides}
\item[twoside] para una impresión a doble cara\footnote{Esta opción y la anterior se utilizan fundamentalmente para las cuestiones de encuadernación. Efectivamente, la opción \option{twoside} produce márgenes izquierdo y derecho de tamaños diferentes, en función de si la página es par o impar. Además, los números de las páginas se sitúan a la izquierda y a la derecha alternativamente.}.\label{rectoverso}
\end{description}

Especificaremos otras opciones a medida que avance la obra, cuando hayamos abordado los conceptos fundamentales.

\subsection{La llamada a los paquetes}

Veamos las dos líneas siguientes: 

\begin{latexcode*}{linenos,firstnumber=2}
\usepackage{fontspec}
\usepackage{polyglossia}
\end{latexcode*}

Se trata, como habrás adivinado, de una llamada a paquetes\footnote{Hemos elegido voluntariamente no traducir este término para evitar confusiones. En la versión española, lo recogemos con la denominación habitual de ``paquete'' (N. del T.).}. Un paquete es un conjunto de ficheros que añaden funcionalidades a \LaTeX, y serían el equivalente de un  plugin en Firefox. 

El primer paquete es \package{fontspec}. Resulta útil en \XeLaTeX  para una tipografía avanzada, en particular, para disponer de acentos en el documento PDF resultante. 

El paquete \package{fontspec} carga automáticamente \package{xunicode}, sin que necesites indicarlo. Éste último permite gestionar el unicode, también llamado UTF-8\footnote{En realidad, UTF-8 no es Unicode del todo, sino una ampliación de éste último. No obstante, para simplificar, los identificamos.}. Así, podemos emplear caracteres no latinos\renvoi{utf8} en el documento \ext{tex}.

\begin{plusloins}
En internet hay documentación que explica que, para escribir una \enquote{é}, hay que te escribir \verb|\'{}|, y aparecen listas de comandos con caracteres acentuados.

Eso tenía validez en otros tiempos. Desde hace bastante, no merece la pena aprender estos compandos: puedes escribir tranquilamente tus letras acentuadas de manera \enquote{normal}. 
\end{plusloins}

El segundo paquete que cargamos a mano es \package{polyglossia}, que permite gestionar con facilidad un documento multilingüe\renvoi{i18n} y los cambios tipográficos que exige un documento de estas características.

Estos tres paquetes son exclusivos de \XeLaTeX{}: no funcionan\renvoi{TeXLaTeX} con \LaTeX.

Algunos paquetes pueden recibir opciones para modificar su comportamiento estándar. La sintaxis es, en ese caso, \csp{usepackage}\oarg{opciones}\marg{paquete}.

A lo largo de esta obra estudiaremos distintos paquetes.

\begin{attention}
De ahora en adelante, cada vez que describamos las funcionalidades de un paquete, se entiende que habremos cargado el paquete previamente en el preámbulo, mediante el comando \csp{usepackage}\marg{paquete}.
\end{attention}

\begin{plusloins}
Cuando hablamos de un paquete, solemos hacer referencia a su manual. Hay una forma muy sencilla de encontrar el manual de un paquete a través de la consola: explicamos cómo hacerlo en un anexo\renvoi{manuels}.
\end{plusloins}

\subsection{El español, lengua por defecto\label{french}}

Inmediatamente después, la línea \csp{setmainlanguage}\verb|{spanish}| indica que empleamos como lengua principal del documento el español\renvoi{i18n}, y que el programa que maqueta el texto deberá tener en cuenta la tipografía española. Esta línea no la entiende el compilados si no hemos cargado previamente\package{polyglossia}.

\begin{plusloins}
Puede que hayas oído hablar del paquete \package{babel}. Se suele utilizar en lugar de \package{polyglossia}, principalmente, porque es más antiguo. No obstante, hemos preferido limitarnos a \package{polyglossia}, porque es el que hemos utilizado para nuestros trabajos y es el que tiene más utilidades, por ejemplo, para las lenguas que se escriben en un alfabeto no latino.

Puedes encontrar información abundante sobre \package{babel} en Internet.

\end{plusloins}

\subsection{El cuerpo del documento}

Todo lo que hemos visto hasta ahora, antes de \cs{begin}\verb|{document}|, forma parte de lo que se llama el preámbulo del documento.\label{preambule} Se trata de informaciones que no figuran en el documento final, pero que resultan útiles para su composición y se las llama también metadatos. A cualquier paquete de desees emplear hay que llamarlo en el preámbulo.

Todo lo que se encuentra entre la línea \cs{begin}\verb|{document}| y \cs{end}\verb|{document}| constituye el cuerpo del documento, el contenido propiamente dicho de tu trabajo.

Por último, el compilador no analiza nada de lo que figure después de \cs{end}\verb|{document}|. Así que puedes poner ahí lo que quieras, aunque no te lo aconsejamos.

\subsection{Título, autor y fecha: la noción de comando}\label{notioncommande}

\begin{latexcode*}{linenos,firstnumber=7}
\title{Título de una obra}
\author{Nombre de su autor}
\date{Fecha}
\maketitle
\end{latexcode*}

Estas tres primeras líneas definen, respectivamente, el título (\csp{title}), el autor (\csp{author}) y la fecha (\csp{date}) del trabajo. En lo que respecta a ésta última, no indicarla equivale a indicar la del día de la compilación, y hay que escribir \cs{date}\verb|{}| para que no aparezca fecha alguna.

La última línea hace que aparezcan estas informaciones. Si tu documento es de la clase \classe{book}, en ese caso el compilador las dispone en una página aparte. Si es de la clase \classe{article}, las incluye sin provocar un salto de página.

Se puede romper esta norma, mediante una opción cuando hacemos la llamada a la clase de documento\renvoi{optionsclasse}.
\begin{description}
\item[notitlepage] para no tener una página de título específica;
\item[titlepage] para tener una página de título específica.
\end{description}

Ahora podemos definir la noción de comando. Un comando es un fragmento de código que el compilador interpreta para realizar un conjunto de operaciones, es un atajo de escritura. 
Aquí el comando \csp{maketitle} muestra informaciones tales como el título, la fecha y el autor del trabajo, informaciones que el compilador conoce gracias a los comandos empleados con anterioridad.

Un comando puede recibir argumentos, algunos optativos y otros obligatorios. Estos argumentos modifican su comportamiento.
\label{syntaxecommande}Llamamos a un comando con la sintaxis: 
\csp{nombre}\oarg{opt1}\oarg{…}\oarg{optn}\marg{obl1}\marg{…}\marg{obln}.

Se indican entre corchetes los argumentos optativos, y entre llaves los argumentos obligatorios. Estos argumentos pueden contener, a su vez, comandos.

El orden de los argumentos depende de cada comando, y los argumentos optativos no van sistemáticamente delante de los argumentos obligatorios: pueden ir después o intercalarse. Fíjate en que algunos comandos no reciben argumento: es el caso de \cs{maketitle}.

\begin{attention}
A cada corchete o llave de apertura debe corresponderle un corchete o llave de cierre porque, de lo contrario, te expones a causar un error de compilación.
\end{attention}

La gran potencia de \LaTeX reside precisamente en el uso de comandos para evitar la repetición de tareas frecuentes. Por eso aprenderemos a definir nuestros propios comandos\renvoi{creercommandes}.



\subsection{El cuerpo del texto: la forma de escribir}

\subsubsection{Análisis de nuestro ejemplo}
Observa ahora las líneas siguientes y su resultado al compilarlas.


\begin{latexcode*}{linenos,firstnumber=12}
Lorem ipsum dolor sit amet, consectetuer adipiscing elit ?
Morbi commodo ; ipsum sed pharetra gravida !
Nullam sit amet enim. Suspendisse id : velit vitae ligula.
Aliquam erat volutpat.
Sed quis velit. Nulla facilisi. Nulla libero. 

Quisque facilisis erat a dui.
Nam malesuada ornare dolor.
Cras gravida, diam sit amet rhoncus ornare, 
erat      elit consectetuer erat, id egestas pede nibh eget odio.
\end{latexcode*}


Podemos observar varias cosas.
\begin{itemize}
\item Una línea vacía produce un cambio de párrafo. Varias líneas vacías producen un solo cambio de párrafo.
\item Un salto de línea, por el contrario, se comporta como un espacio\footnote{En materia de tipografía, este término es femenino. No así en español (N. del T.)}. Es una gran diferencia respecto a los programas WYSIWYG, que traducen automáticamente un salto de línea como un salto de párrafo.
\item Varios espacios seguidos producen un solo espacio. 
\end{itemize}

Por tanto, ya conoces las normas fundamentales para redactar un texto con \LaTeX.

\subsubsection{Avancemos}

Ya lo hemos dicho, \LaTeX produce un formato y una tipografía más correctas que un programa de tipo WYSIWYG. Pero hay que ofrecerle un código correcto, para que pueda determinar cómo componerlo tipográficamente.

\LaTeX produce automáticamente un espacio fino ante los signos de puntuación dobles, \verb|!:;?| principalmente, como se ha de hacer en buena tipografía francesa\footnote{Un espacio fino es un espacio más pequeño que un espacio normal. Esto se aplica a la tipografía francesa, pero no a la española que desconoce este espacio (N. del T.)}. Sin embargo, recomendamos incluir espacios en el fichero \ext{tex} ante estos signos de puntuación dobles, para facilitar la lectura.

\begin{attention}
Los espacios delante de los signos de puntuación dobles son una característica propia de la tipografía francesa. Casi no se dan en las demás lenguas. Por esa razón, si escribes en una lengua distinta a la de Molière, no tienes que incluir esos espacios. Por tanto, queda en tus manos decidir si los incluyes o no en tu fichero fuente: piensa que \LaTeX los incluirá por ti llegado el caso, pero no los eliminará en las lenguas que no sean francés.
\end{attention}

Por el contrario,\emph{es obligatorio incluir un espacio detrás de cada signo de puntuación}. Por lo que atañe a los puntos suspensivos, es mejor no escribir tres puntos seguidos, sino utilizar el comando \csp{ldots} que podrá un espacio correcto entre los puntos\footnote{También se puede configurar el editor de texto para que sustituya automáticamente tres puntos seguidos por este comando.}.

En lo que respecta a las comillas, dedicaremos un apartado más adelante a la manera de hacer las citas\renvoi{guillemets} en \LaTeX. Así que no hablamos de eso ahora.

Prestemos atención a algunas ligaduras como \verb|œ| y  \verb|æ|. A diferencia de la mayoría de los procesadores de texto, \LaTeX no sustituye automáticamente las secuencias \verb|oe| y \verb|ae| por \verb|œ| o \verb|æ|. Por tanto, hay que escribir por uno mismo estos caracteres o configurar el editor para que realice esta sustitución.

Señalemos también tres tipos de guiones\label{tirets}:
\begin{enumerate}
\item \verb|-| que produce un guión corto (-), que se emplea para las palabras compuestas;
\item \verb|--| que produce un guión intermedio (--), en teoría se utiliza para separar un rango de números;
\item \verb|---| que produce un guión largo (---), para incisos\footnote{Algunos editores prefiern emplear los guiones intermedios.}.
\end{enumerate}
 
 Por último, en ocasiones resulta útil incluir un espacio inseparable para evitar que dos palabras queden separadas por un salto de línea, por ejemplo, entre el nombre del rey y su número de reino: \enquote{Juan~\textsc{xxiii}}.  El espacio inseparable se indica con el carácter \verb|~|.

Por otro lado, como has podido observar, \LaTeX interpreta de manera especial algunos caracteres: \verb|\{}~|, a los que tenemos que añadir\verb|%_&$#^| \footnote{No estudiaremos el uso que hace \LaTeX de todos estos signos, pues algunos sirven fundamentalmente para componer fórmulas matemáticas.}.

¿Qué hay que hacer si queremos incluir uno de estos caracteres? Hay que ponerles delante el signo~\verb|\|. Así, para incluri el carácter \verb|%|, hay que escribir \verb|\%|. 

Sin embargo, hay tres excepciones:
\begin{description}
\item[\textbackslash] que se escribe con el comando \csp{textbackslash} ;
\item[\textasciitilde] que se escribe con el comando \csp{textasciitilde} ; 
\item[\textasciicircum] que se escribe con el comando \csp{textasciicircum}. 
\end{description} 
\subsection{Comentarios}

La línea siguiente es: 
\begin{latexcode*}{linenos,firstnumber=22}
%Fin del documento
\end{latexcode*}

En \LaTeX hay una norma muy sencilla: todo lo que se encuentra a la derecha de un signo \verb|%| es un comentario.
Es decir, el compilador no lo interpreta y, por tanto, no aparece en el documento final. 

Recomendamos el uso de comentarios para indicar las estructuras más grandes del documento y para comentar los comandos que crees tú mismo\renvoi{creercommandes}. 

También los puedes utilizar, por ejemplo, para hacer un comentario para tu uso particular línea por línea de un texto que traduzcas.

Por el contrario, no te recomendamos utilizarlos para notas personales en el curso de la redacción. Te indicaremos más adelante cómo definir un comando personalizado para generar un fichero que las muestre, para una revisión, y otro que las oculte, para el documento final\renvoi{commentaireredac}.



\subsection{El concepto de entorno}

Hemos visto hasta ahora los conceptos de paquete, preámbulo y comando. Nos queda por definir un último concepto: el de entorno.

Un entorno es un fragmento de documento que tiene un significado específico y, en consecuencia, se somete a un tratamiento específico. Por ejemplo, para indicar una cita, una lista, etc. Descubriremos entornos a medida que vayamos avanzando. 


El comienzo de un entorno se señala  \arg{nombre} con \csp{begin}\marg{nombre} y se termina \csp{end}\marg{nombre}.

En la clase \classe{article} hay un entorno útil: \enviro{abstract}. En este entorno se coloca un resumen del artículo:

\begin{latexcode}
\begin{abstract}
Aquí escribimos un resumen del artículo 
\end{abstract}
\end{latexcode}


Se pueden anidar varios entornos:

\begin{latexcode}
\begin{1}
blabla blab
\begin{2}
blabl blab
\end{2}
blabl
\end{1}
\end{latexcode}

Por el contrario, no se pueden superponer entornos: así, el código siguiente no funciona y provoca un error en la compilación.


\begin{latexcode}
\begin{1}
blabla blab
\begin{2}
blabl blab
\end{1}
blabl
\end{2}
\end{latexcode}

\subsection{Conclusión}

Has aprendido aquí los principales conceptos de \LaTeX. 
 Por ahora, quizá te parezca esto muy difuso: pero, a medida que progrese tu lectura, lo irás comprendiendo mejor\footnote{¡Al menos, eso es lo que esperamos!}\ldots


