\chapter{Dar sentido (3): crea tus propios comandos}\label{creercommandes}

\begin{intro}
Hemos hablado largo y tendido del interés de separar el hecho de dar sentido del hecho de dar forma\renvoi{sensforme}.
Hemos indicado que la mejor manera de hacerlo consistía en crear comandos para dar sentido que invocarían por sí mismos comandos para dar forma.

Veamos ahora cómo crear esos comandos personalizados.
\end{intro}

\section{Creación de un comando personalizado}

Queremos crear un comando personalizado que sirva para indicar que hablamos de un autor: \csp{autor}\marg{nombre}.

Nuestro comando se descompone en varias partes: su nombre, aquí \contenuarg{autor}, y sus argumentos, aquí uno solo: \arg{nombre}.

Un comando LaTeX puede recibir hasta nueve~argumentos que, por lo general, son suficientes. Si notamos con \arg{N} el número de argumentos, la sintaxis de la declaración de un nuevo comando es la siguiente:
\csp{newcommand}\verb|{\|\arg{nombre del comando}\verb|}|\oarg{N}\marg{código}.

\begin{attention}
   Los nombres de comandos sólo deben contener caracteres latinos no acentuados. 
    
    Los nombres son sensibles a mayúsculas y minúsculas: \verb|\a| es distinto de \verb|\A|.
\end{attention}
En el interior de la parte\arg{código}, se puede:
\begin{itemize}
    \item poner texto;
    \item utilizar comandos para dar forma o para dar sentido;
    \item llamar a los argumantos que se han pasado mediante la sintaxis \verb|#x|, donde \verb|x| representa el rango del argumento.
\end{itemize}

Retomemos nuestro caso de un comando para indicar los nombres. De momento, sólo queremos que los nombres de autor vayan seguidos de un asterisco (*).

\begin{latexcode}
\newcommand{\autor}[1]{#1*}
Es bien sabido que \autor{Tertuliano} no era montanista, sino tertulianista.
\end{latexcode}

\newcommand{\autor}[1]{#1*}

Lo que nos da:

\begin{quotation}
Es bien sabido que \autor{Tertuliano} no era montanista, sino tertulianista.
\end{quotation}

\contenuarg{Tertuliano} se pasa como primer argumento. Así que aparece en lugar de \#1 de la declaración del comando.

Podríamos querer pasar como segundo argumento informaciones complementarias, como los datos de su vida.

\begin{latexcode}
\newcommand{\autor}[2]{#1* (#2)}
Es bien sabido que \autor{Tertuliano}{150 ? - 220 ?}
no era montanista, sino tertulianista.
\end{latexcode}

\renewcommand{\autor}[2]{#1* (#2)}

\begin{quotation}
Es bien sabido que \autor{Tertuliano}{150 ? - 220 ?}
no era montanista, sino tertulianista.
\end{quotation}

\begin{plusloins}
El lector atento recordará que hay comandos con argumentos optativos \renvoi{syntaxecommande}. Sin duda querrá crear algunos.

Aunque \LaTeX proporciona un mecanismo propio para permitirlo, éste es poco flexible, ya que sólo permite hacer opcional el primer argumento y no los siguientes.

Así que recomendamos consultar los packages \package{xargs} e \package{ifthenelse}\footcites{xargs}{ifthen}. Ofrecemos en esta obra un ejemplo de comando que funciona con estos packages\renvoi{commandevariante}.
\end{plusloins}

Muy bien, pero, ¿cómo hay que hacer para dar forma? Hay que utilizar un comando para dar forma en el interior de una declaración de comando para dar sentido. Por ejemplo, para ponerlo en versalitas, el comando es \cs{textsc}.

\begin{latexcode}
\newcommand{\autor}[2]{\textsc{#1}* (#2)}
\end{latexcode}

\renewcommand{\autor}[2]{\textsc{#1}* (#2)}

\begin{quotation}
Es bien sabido que \auteur{Tertuliano}{150 ? - 220 ?}
no era montanista, sino tertulianista.
\end{quotation}

Describimos más adelante un conjunto de comandos para dar forma\renvoi{mef}.

\begin{attention}
Aunque la posibilidad de crear tus propios comandos es una gran potencia de \LaTeX, sin embargo, no siempre les gusta a los editores.
   
    Así que te aconsejamos hablarlo previamente con tu editor y, sobre todo, indicarle que un marcado en forma de comandos normalmente tiene que facilitar su tarea.
    
    Además, te aconsejamos agrupar todas tus declaraciones de comandos en un ficher único, que se invoca por medio de un comando \cs{input}\renvoi{input}.
    
    También puedes crear tu propio package, para comandos que utilices a menudo en varios proyectos\footcite[Eso desbordaría el marco de esta obra: te remito a otros documentos. Por ejemplo][]{creer_sty}.
\end{attention}


\begin{plusloins}\label{commentaireredac}
Se puede crear de esta manera con gran facilidad un comando \csp{comentario} para recoger comentarios personales durante la redacción, con el fin de tenerlos a la vista en una revisión en papel.
Por ejemplo:

\begin{latexcode}
\newcommand{\comentario}[1]{\marginpar{#1}}
\end{latexcode}

Para proceder a la impresión definitiva, basta con cambiar este comando por:

\begin{latexcode}
\newcommand{\comentario}[1]{}
\end{latexcode}
\end{plusloins}

\subsection{Código de un comando en varias líneas}\label{commandepourcent}

Hasta ahora, nuestos comandos eran relativamente sencillos. Pero puede ocurrir que su código se complique. En ese caso, desearíamos escribirlo en varias líneas.

Se puede, pero hay que poner simplemente un signo de porcentaje (\%) al final de cada línea para que no se produzcan espacios indeseables. Ejemplo sin \%{}:

\begin{latexcode}
\newcommand{\autor}[2]{
    \textsc{#1}* (#2)
    }
\end{latexcode}

\renewcommand{\autor}[2]{\underline{ }%
    \textsc{#1}* (#2)\underline{ }%
}

\begin{quotation}
Es bien sabido que \autor{Tertuliano}{150 ? - 220 ?}
no era montanista, sino tertulianista.
\end{quotation}

Se observan espacios excesivos delante y detrás de la mención del autor, que hemos subrayado para clarificar la explicación. Por el contrario, si se ponen \%:

\begin{latexcode}
\newcommand{\autor}[2]{%
    \textsc{#1}* (#2)%
    }
\end{latexcode}

\renewcommand{\autor}[2]{%
    \textsc{#1}* (#2)%
}

\begin{quotation}
Es bien sabido que \autor{Tertuliano}{150 ? - 220 ?}
no era montanista, sino tertulianista.
\end{quotation}

Ya no hay espacios indeseables.


\subsection{Versión con asterisco de \oldcs{newcommand}}

La creación de comandos en \LaTeX es una funcionalidad muy poderosa, pero potencialmente peligrosa. Tomemos el comando siguiente:


\begin{latexcode}
\newcommand{\autor}[1]{\textsc{#1}*}
\end{latexcode}

Imaginemos que lo invocamos pero nos olvidamos de cerrar la llave:

\begin{latexcode}
Es bien sabido que \autor{Tertuliano
no era montanista, sino tertulianista.
\end{latexcode}

¿Qué ocurrirá? \LaTeX interpreta que el conjunto del final del texto es un argumento del comando, lo que impide que el compilador funcione bien y no obtendremos nada o, en todo caso, un resultado incorrecto.

Para limitar estos problemas, \LaTeX ofrece la posibilidad de definir comandos cortos, cuyos argumentos no sobrepasan la longitud de un párrafo. En caso de olvidar el cierre de paréntesis, \LaTeX detiene automáticamente el argumento hasta el final del párrafo.

Para definir estos comandos cortos basta con emplear \csp{newcommand*}.

\begin{latexcode}
\newcommand*{\autor}[1]{\textsc{#1}*}
\end{latexcode}

\subsection{Espacio tras los comandos sin argumento}

Se pueden definir comandos sin argumento. Tomemos un comando personalizado \csp{cf}, que permite obtener rápidamente la forma \forme{\cf}

\begin{latexcode}
\newcommand{\cf}[0]{\emph{cf.}}
\end{latexcode}

Como nuestro comando no recibe argumento, es necesario, cuando lo llamemos, separarlo del texto siguiente con un espacio, para que \LaTeX sepa dónde acaba el comando: 

\begin{latexcode}
Esto es muy interesante \cf eso.
\end{latexcode}

Pero el espacio desaparece durante la compilación:

\begin{quotation}
Esto es muy interesante \cf eso.
\end{quotation}

La solución consiste en emplear en la definición del comando el comando \csp{xspace} del package homónimo\index[pkg]{xspace}. Éste permite gestionar automáticamente los espacios que van tras la llamada a un comando, teniendo en cuenta la tipografía.

\begin{latexcode}
\newcommand{\cf}[0]{\emph{cf.}\xspace}
\end{latexcode}

\renewcommand{\cf}[0]{\emph{cf.}\xspace}

\begin{quotation}
Esto es muy interesante \cf eso.
\end{quotation}

\begin{plusloins}
El package \package{foreign}\footcite{foreign} ofrece toda una serie de comandos similares para indicar abreviaturas o expresiones extranjeras frecuentes.
\end{plusloins}


\section{Comandos para dar forma\label{mef}}

Recogemos aquí los principales comandos para dar forma. Para necesidades avanzadas, se encuentra fácilmente documentación en internet. En lo que respecta a las cuestiones de espaciado, se les dedica un capítulo aparte\renvoi{espacement}.

\subsection{Tamaño de la fuente}\label{taille}

Se puede definir de forma absoluta el tamaño de la fuente indicando el número de puntos. Sin embargo, el método mejor consiste en definirlo a partir del tamaño de base por medio de los comandos siguientes. El tamaño obtenido depende del tamaño de base definido en el preámbulo\renvoi{preambule}.

\begin{longtable}{|l|l|}
      \hline
     \headlongtable{Comando}                 &    \headlongtable{Efecto}                                 \\
      \hline
     \endhead
    \hline
    \endfoot
     \csp{tiny}             &     \tiny abcdefghijklmnopqrst              \\
     \csp{scriptsize}         &     \scriptsize abcdefghijklmnopqrst          \\
     \csp{footnotesize}     &     \footnotesize abcdefghijklmnopqrst         \\
     \csp{small}            &    \small abcdefghijklmnopqrst             \\
     \csp{normalsize}        &     \normalsize abcdefghijklmnopqrst         \\
     \csp{large}            &    \large abcdefghijklmnopqrst             \\
     \csp{Large}            &     \Large abcdefghijklmnopqrst             \\
     \csp{LARGE}        &     \LARGE abcdefghijklmnopqrst             \\
     \csp{huge}            &     \huge abcdefghijklmnopqrst             \\
     \csp{Huge}            &    \Huge abcdefghijklmnopqrst             \\
\end{longtable}

\label{commandesdetaille}Estos comandos, sin embargo, tienen un comportamiento peculiar: se trata de comandos llamados \enquote{conmutadores}\renvoi{bascule}. A diferencia de la mayoría de los comandos, no reciben argumento. Se deben poner entre llaves y hay que incluir el texto cuyo tamaño queremos cambiar después del comando, dentro de las llaves.

Ainsi :
\begin{latexcode}
{\large Texto más grande}
\end{latexcode}

Y no:

\begin{latexcode}
\large{Texto más grande}
\end{latexcode}

\subsection{Estilo de la fuente}

Damos una lista, que no es exhaustiva, de comandos útiles para personalizar los estilos de fuente.


\begin{longtable}{|l|l|}
    \hline
    \headlongtable{Comando}                & \headlongtable{Efecto} \\                                
    \hline
    \endhead
    \hline
    \endfoot
    \csp{textit}            & \textit{Cursiva}                            \\
    \csp{emph}            & \emph{Texto con énfasis}                    \\
    \csp{textbf}            &  \textbf{Negrita}                            \\
    \csp{textsc}            & \textsc{Versalitas}                    \\
    \csp{underline}        & \underline{Subrayado}     (debe evitarse)                \\
    \csp{textsuperscript}    &  \textsuperscript{Superíndice}                    \\
    \csp{textsubscript}        & \textsubscript{Subíndice} (necesita el package \package{subscript}) \\
\end{longtable}

\subsection{Colores}

Los colores no se admiten de forma nativa en \LaTeX. Hay que acudir al package \package{color} o al package \package{xcolor} que ofrecen, ambos, un comando \csp{textcolor}\marg{nombre del color}\marg{texto en color}.


La diferencia radica sustancialmente en el número de colores disponibles por defecto y en la facilidad para definir otros nuevos, pero también en la posibilidad de aplicar color a otros elementos que no son texto. Sólo presentamos aquí el package \package{xcolor}, de forma sucinta. Remitimos a la documentación para usos avanzados\footcite{xcolor}.

\subsubsection{Colores estándar}
El package \package{xcolor} ofrece los siguientes colores básicos: 

% Definición de la función para manipular colores 
\newcommand{\ejemplocolor}[1]{#1 & \fcolorbox{black}{#1}{~} \\[1pt]}
% Fin de la definición

\begin{longtable}{|l|l|}
    \hline
    \headlongtable{Nombre del color}         & \headlongtable{Color}                                 \\
    \hline
    \endhead
    \hline
    \endfoot
    \exemplecouleur{black}
    \exemplecouleur{blue}
    \exemplecouleur{brown}
    \exemplecouleur{cyan}
    \exemplecouleur{darkgray}
    \exemplecouleur{gray}
    \exemplecouleur{green}
    \exemplecouleur{lightgray}
    \exemplecouleur{lime}
    \exemplecouleur{magenta}
    \exemplecouleur{olive}
    \exemplecouleur{orange}
    \exemplecouleur{pink}
    \exemplecouleur{purple}
    \exemplecouleur{red}
    \exemplecouleur{teal}
    \exemplecouleur{violet}
    \exemplecouleur{white}
    \exemplecouleur{yellow}
    
\end{longtable}

\subsubsection{Colores adicionales}

Se pueden pasar las opciones \option{dvipsnames}, \option{svgnames} o \option{x11names} cuando se llama al package \package{xcolor}, cada una de las cuales ofrece un juego de colores. Como no podemos dar una lista de todos esos colores en esta obra, remitimos a la documentación de \package{xcolor}\footcite{xcolor_jeu}.

\subsubsection{Definir tus propios colores}

Puedes definir tus propios colores. Hay que utilizar  \csp{definecolor}, de acuerdo con la sintaxis siguiente: 
\cs{definecolor}\marg{nombre}\marg{método}\marg{definición}.

Los colores se pueden definir de varias maneras\footnote{Se puede consultar cualquier libro científico sobre los colores para más detalles. También se puede consultar la documentación del package \package{xcolor}.}. Las dos más importantes son\footcite[Existe también el método de definición por la longitud de onda para los colores del arco iris, por porcentaje de gris para los distintos niveles de gris, así como por tinta, saturación y luminosidad, pero esto exigiría un curso de física de la luz. Para las personas interesadas en los detalles, consúltese][]{xcolor_methode} :
\begin{itemize}
\item El método aditivo, añadiendo rojo, verde y azul (como en una pantalla de ordenador).
\item El método sustractivo, superponiendo cian, magenta, amarillo y negro (como en una impresora a color).
\end{itemize}

En ambos casos hay que dar un valor comprendido entre 0 (incluido) y 1 (incluido) a cada uno de los componentes\footcite[Se puede encontrar una serie de código de colores en la página][]{codecouleur}.

\definecolor{rojoborgoña}{HTML}{6B0D0D}

Tomemos, por ejemplo, la definición del color \textcolor{rojoborgoña}{rojo borgoña}\footnote{Con Anne Sylvestre los autores proclaman con orgullo \enquote{Que Burdeos me perdone, pertenezco a Borgoña} \parencite{romaneconti}.}.
Se puede definir del modo siguiente con el método aditivo:

\begin{latexcode}
\definecolor{rojoborgoña}{rgb}{0.41,0.05,0.05}
\end{latexcode}

O bien con el método sustractivo\footnote{Lo ideal sería que la elección del método tuviera en cuenta el soporte de destino. Pwero el package \package{xcolors} ofrece sistemas de conversión de un método a otro.} de la forma siguiente:

\begin{latexcode}
\definecolor{rojoborgoña}{cmyk}{0,0.88,0.88,0.58}
\end{latexcode}

Por último, mencionaremos el método HTML, que tiene la ventaja de que cuenta con muchas páginas en internet que indican los códigos de color HTML\footnote{En realidad, el método HTML es un método aditivo con valores comprendidos entre 0 y 225, notados en hexadecimal.}.

El mismo color se define así:

\begin{latexcode}
\definecolor{rojoborgoña}{HTML}{6B0D0D}
\end{latexcode}




