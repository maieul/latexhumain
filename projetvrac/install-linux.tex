\documentclass[12pt]{book}
\usepackage[T1]{fontenc}
\usepackage[utf8]{inputenc}
\usepackage[frenchb]{babel}

\begin{document}

\section{Installer TeX Live sous GNU/Linux}

Il y a plusieurs manières de procéder pour installer TexLive sur une distribution GNU/Linux. Il est possible de télécharger l'installateur pour GNU/Linux depuis le site de TeX Live, mais la procédure d'installation de TeX Live est alors complexe, tout comme la mise à jour. Nous ne détaillerons donc pas cette procédure d'installation.

La meilleure façon de procéder est en effet de recourir au gestionnaire de paquets de votre distribution. C'est l'approche que nous allons documenter pour les distributions GNU/Linux grand public les plus usitées, car la mise à jour de tous les paquets de TeX Live en est facilitée : ils sont mis à jour en même temps que les logiciels de votre distribution GNU/Linux.

Pour l'installation complète telle que présentée ci-dessous, il faut prévoir environ trois gigaoctets d'espace libre en comptant la taille des paquets téléchargés et leur place installée.

\subsection{Debian et ses dérivés}

Il est très facile d'installer la distribution TeX Live avec Debian, Ubuntu et ses dérivés, ou bien Linux Mint. Il vous suffit d'ouvrir le gestionnaire de paquet \emph{Synaptic}, de chercher le paquet  \verb|texlive-full|, et de l'installer en acceptant toutes ses dépendances.

\subsection{Les distributions basées sur les paquets RPM}

\subsubsection{Mageia et Mandriva}

Ces deux systèmes sont très proches, par conséquent les instructions pour l'une valent pour l'autre. Il faut dans un premier temps ouvrir le \emph{centre de contrôle}. Il suffit alors de cliquer sur le bouton \emph{Installer et désinstaller des logiciels}, puis de chercher le paquet \verb|texlive| et appliquer les changements. 

\subsubsection{OpenSUSE}

Le processus est similaire. Ouvrez l'application \emph{Installer des logiciels}, puis cherchez les paquets \verb|texlive|, \verb|texlive-xetex|, et \verb|texlive-fonts-extra|. Appliquez alors les changements.

\subsubsection{Fedora}

Sous Fedora, ouvrez l'application \emph{Ajouter/supprimer des logiciels}. Cliquez dans la colonne de gauche sur la catégorie \og{}Éditeurs\fg{}, puis sélectionnez les paquets \verb|texlive| et \verb|texlive-texmf-latex|. Appliquez les changements et acceptez les dépendances.

\end{document}
