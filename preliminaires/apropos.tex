

\phantomsection\addcontentsline{toc}{section}{Sobre este libro}\section*{Sobre este libro}\thispagestyle{plain}


Es evidente que este libro se ha maquetado con \XeLaTeX. Además de los paquetes de los que trata, he utilizado el paquete \package{minted} para las citas de código; los paquetes \package{mdframed} y \package{framed} para las cajas a color.

Este libro se distribuye bajo la licencia \emph{Creatives Commons - Paternité - Partage des Conditions Initiales à l'Identique 3.0 France}. En síntesis\footnote{Para los detalles, remito al texto íntegro de la licencia: \url{http://creativecommons.org/licenses/by-sa/3.0/fr/legalcode}.}, esto significa que puedes distribuirlo, duplicarlo, publicarlo e incluso modificarlo si te atienes a dos condiciones:
\begin{enumerate}
\item citar mi nombre\footnote{Y no atentar contra mis derechos morales.};
\item ofrecer los mismos derechos a los destinatarios de tus distribuciones\footnote{Las imágenes de pasos y de relámpago sirven para indicar que las inserciones se han sacado del dominio público y se han modificado (ligeramente) por mí. A ellas no les afectan estas reglas. Véase \url{http://www.openclipart.org/detail/154855/green-steps-by-netalloy} y \url{http://thenounproject.com/noun/high-voltage/}. La imagen de la cubierta es de  Duane  Bibby, con una pequeña modificación. Véase \url{http://www.ctan.org/lion.html}.}.
\end{enumerate}

Por supuesto, si quieres apoyarme, puedes comprar esta obra en versión de papel o sencillamente enviarme una nota ---puedes encontrar fácilmente cómo contactar conmigo en Internet.

Si quieres mejorar esta obra, bienvenido. El código está disponible en GitHub\footnote{En la dirección \url{https://github.com/maieul/latexhumain}.}, un servicio que funciona con la ayuda de la herramienta de colaboración Git\renvoi{svn} pero con una interfaz de edición en línea.

¡No dudes en pedirme un acceso a la edición del proyecto!
