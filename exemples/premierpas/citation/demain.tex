\begin{verse}
\poemlines{5}
 Pisó las calles de Madrid el fiero \\
 monóculo galán de Galatea\\
y, cual suele tejer bárbara aldea\\
 soga de gozques contra forastero,
 
 rígido un bachiller, otro severo,\\
crítica turba al fin, si no pigmea,\\
 su diente afila y su veneno emplea\\
en el disforme cíclope cabrero.
 
A pesar del lucero de su frente,\\
lo hacen obscuro, y él, en dos razones\\
que en dos truenos libró de su occidente,
 
«Si quieren -respondió- los pedantones\\
luz nueva en hemisferio diferente,\\
den su memorïal a mis calzones».

%Góngora, Sonetos 288 (1615).

\end{verse}
