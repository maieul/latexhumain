\chapter{Personalizar encabezados y pies de página}\label{entete}

\begin{intro}
Vamos a ver cómo personalizar los encabezados y pies de página con la ayuda del package \package{fancyhdr}.
\end{intro}


\section{Utilizar uno de los estilos estándar}\label{stylesentete}

\LaTeX ofrece cuatro estilos de página estándar, que definen el contenido de los encabezados y pies de página. Para usar uno de esos estilos se emplea el comando \csp{pagestyle}\marg{estilo}.


También se puede modificar el estilo de una página concreta mediante el comando \csp{thispagestyle}\marg{estilo}.

Estos estilo son:
\begin{description}
\item[empty] sin encabezado ni pie de página.
\item[plain] sin encabezado pero con pie de página que contiene el número de la página centrado. Es el estilo correspondiente a las páginas de inicio de capítulo\renvoi{chapitrepagestyle}. 
\item[heading] si pie, los encabezados con el título del capítulo, del apartado o del subapartado y el número de página. Es el estilo por defecto. \label{styleentetes}
\item[myheading] similar al anterior, pero el encabezado se puede personalizar\footcite[Obtuvimos esta información del manual de][\pno~222, que no explica cómo hacerlo]{latex_companion}.
\end{description}



Enseguida vemos las limitaciones de estos estilos. Por tanto, ¿cómo conseguimos tener a la vez los títulos en el encabezado y los números de página en el pie de página? ¿Cómo indicar el nombre o la fecha en el pie de página?

\section{Primeros ejemplos con \packagenoidx{fancyhdr}}\index[pkg]{fancyhdr}

El package \package{fancyhdr} ofrece otro estilo que es fácil de personalizar por medio de comandos específicos. Para que funcione \package{fancyhdr} hay que escribir las líneas siguientes en el preámbulo:

\begin{latexcode}
\usepackage{fancyhdr}
\pagestyle{fancy}
\end{latexcode}

\begin{plusloins}
Las páginas de inicio de capítulo tienen automáticament el estilo \forme{plain}. Para desactivar este estilo por defecto, hay que modificar el comando \cs{chapter}. Hablamos de ello en otro capítulo\renvoi{entetechapter}. 
\end{plusloins}


Luego se pueden utilizar los seis comandos de \package{fancyhdr} que permiten definir el contenido de los encabezados y pies de página:

\begin{description}
\item[\csp{lhead}] recibe como argumento el contenido de la parte izquierda del encabezado, justificado a la izquierda.
\item[\csp{chead}] recibe como argumento la parte central del encabezado, centrada.
\item[\csp{rhead}] recibe como argumento la parte derecha del encabezado, justificada a la derecha.
\item[\csp{lfoot}] recibe como argumento la parte izquierda del pie de página, justificada a la izquierda.
\item[\csp{cfoot}] recibe como argumento la parte central del pie de página, centrada.
\item[\csp{rfoot}] recibe como argumento la parte derecha del pie de página, justificada a la derechae.
\end{description}



Imaginemos que deseamos mostrar el número de página en el pie de página centrado, indicando también el número total de páginas. Utilizaremos el package \package{totpages} que nos permite, después de dos compilaciones o más, si el número de páginas varía entre las compilaciones, obtener el número total de páginas.

\begin{latexcode}
\usepackage{fancyhdr}
\pagestyle{fancy}
\usepackage{totpages}
\cfoot{{\thepage} / \ref{TotPages}}
\end{latexcode}



El comando \csp{thepage} indica el valor del contador \compteur{page}\renvoi{compteur}, correspondiente al número de página.

Podemos ver el resultado en esta página.
\cfoot{{\thepage} / \ref{TotPages}\cfoot{\thepage}} 

\section{Páginas a doble cara y alternancia de izquierda a derecha}

Cuando un trabajo se imprime a doble cara, quizá queramos que el encabezado y el pie de página izquierdos de las páginas impares se correspondan con el encabezado y el pie de página derechos de las páginas pares y viceversa.

El package \package{fancyhdr} ha previsto este caso. Ofrece dos comandos: 
\begin{itemize}
\item \csp{fancyhead}\oarg{posición}\marg{texto del encabezado}
\item \csp{fancyfoot}\oarg{posición}\marg{texto del pie de página}
\end{itemize}

El argumento \arg{posición} puede recibir uno o varios de los valores siguientes:
\begin{description}
\item[CO] centro de las páginas impares (de la derecha).
\item[CE] centro de las páginas pares (de la izquierda).
\item[LO] izquierda de las páginas impares (= interior de las páginas de la derecha, si escribimos de izquierda a derecha).
\item[RO] derecha de las páginas impares (= exterior de las páginas de la derecha si escribimos de izquierda a derecha).
\item[LE] izquierda de las páginas pares (= exterior de las páginas de la izquierda, si escribimos de izquierda a derecha).
\item[RE] derecha de las páginas pares (= interior de las páginas de la izquierda, si escribimos de derecha a izquierda).
\end{description}

De este modo, para incluir el número de página en el exterior del pie de página, basta con escribir:

\begin{latexcode}
\fancyfoot[LE,RO]{\thepage}
\end{latexcode}

Por supuesto, si le decimos a \LaTeX que gener un fichero destinado a la impresión de una sola cara\renvoi{rectoverso}, considera que solo hay páginas a una cara, es decir, impares.

\section{Títulos en el encabezado: el mecanismo de los marcadores}

En la clasee \classe{book}, los encabezados contienen por defecto los títulos de los capítulos en la página de la izquierda y los de los apartados en la página de la derecha. Utilizando el estilo \verb|fancy|, se incluyen automáticamente los títulos de los capítulos en el lado interior y los de los apartados, en el lado exterior. 

Si utilizamos \package{fancyhdr} para personalizar solo el pie de página, puede que queramos mantener para el resto la presentación por defecto. Para ello hay que comprender el mecanismo de los marcadores de \LaTeX. 

El principio básico es sencillo: los comandos de marcado almacenan en la memoria marcadores. Dichos marcadores se llaman mediante otros comandos. 

Los dos comandos de marcado son:
\begin{itemize}
\item \csp{markboth}\marg{marcador izquierdo}\marg{marcador derecho}
\item \csp{markright}\marg{marcador derecho}
\end{itemize}

Los dos comandos para llamar a los marcadores son: 
\begin{itemize}
\item\csp{leftmark} que devuelve el argumento \arg{marcador izquierdo} del último comando \cs{markboth}.
\item\csp{rightmark}  que devuelve el argumento \arg{marcador derecho} del primer comando \cs{markright} o \cs{markboth}  situado en la página actual. En cambio, si la página en curso no contiene ninguno de estos comandos, entonces \csp{rightmark} devuelve el argumento \arg{marcador derecho} del último \cs{markright} o \cs{markboth} utilizado.
\end{itemize}

Concretamente, los comandos \cs{markboth} y \cs{markright} se invocan automáticamente en la clase \classe{book} con los comandos \cs{chapter} y \cs{section} por medio de los comandos \csp{chaptermark} y \csp{sectionmark}. 

\subsection{Llamar a los marcadores en los estilos fancy}

Para restablecer la presentación original, tenemos que incluir los comandos \cs{leftmark} y \cs{rightmark} en nuestros comandos \cs{fancyhead}.

\begin{latexcode}
\fancyhead[LE,RO]{}
\fancyhead[RE]{\leftmark}
\fancyhead[LO]{\rightmark}
\end{latexcode}

Con esto obtenemos nuevamente el título del capítulo a la izquierta y el título del apartado a la derecha.

\begin{attention}
Los encabezados y los pies de página los calcula  \LaTeX \emph{después} del resto de la página. Por eso obtenemos en el encabezado el contenido del último apartado de la página.
\end{attention}

\subsection[Redefinir \oldcs{chaptermark} y \oldcs{sectionmark}]{Redefinir los comandos \cs{chaptermark} y \cs{sectionmark}}

Ahora imaginemos que ya no queremos ver los títulos de los apartados en los encabezados. Lo más sencillo consiste entonces en redefinir el comand \cs{sectionmark}, para anularlo:

\begin{latexcode}
\renewcommand{\sectionmark}[1]{}
\end{latexcode}

\begin{attention}
Hay que tener en cuenta que indicamos que el comando \cs{sectionmark} recibe un argumento, incluso aunque no lo usemos. Pero cuando la clase \classe{book} llama a este comando, le pasa un argumento que es el título del apartado.
\end{attention}

Como ahora solo se muestra el títiulo del capítulo,Comme  on n'affiche  plus  alors que le titre du chapitre, quizá no sea necesaro precisar que se trata de un capítulo. Entonces hay que redefinir el comando \cs{chaptermark}, por ejemplo, de la manera siguiente:

\begin{latexcode}
\renewcommand{\chaptermark}[1]{\markboth {\MakeUppercase{%
    \thechapter~#1}}{}}
\end{latexcode}


\begin{plusloins}
Si te entretienes ojeando el fichero \fichier{book.cls}, verás que el comando \cs{chaptermark} está definido dos veces. En realidad, su definición está condicionada por la opción de clase \option{twoside} o \option{oneside}\renvoi{nbsides}.
\end{plusloins}

\begin{plusloins}
El comandoe \cs{thechapter} sirve para mostrar el número del capítulo, almacenado en un contador \compteur{chapter}. Hablamos más adelante de los contadores y de los comandos \cs{the\meta{contador}}\renvoi{apparencecompteur}.
\end{plusloins}


\section{Raya en encabezados y pies de página}

El package \package{fancyhdr} incluye una raya entre el cuerpo del texto y el encabezado. Se puede modificar el grosor de esta raya, así como añadir una raya delante del pie de página.

Para ello redefinimos los comandos \csp{headrulewidth} y \csp{footrulewidth}, que deben devolver una longitud\renvoi{unite}.
Por ejemplo, para suprimir la raya debajo del encabezado pero para añadir una de 0,05 mm. encima del pie de página escribimos:

\begin{latexcode}
\renewcommand{\headrulewidth}[0]{0pt}
\renewcommand{\footrulewidth}[0]{0.05mm}
\end{latexcode}
