\chapter{Apariencia de los textos}


\begin{intro}
En este capítulo veremos cómo cambiar la apariencia de nuestros textos:Dans ce chapitre nous verrons comment changer l'apparence de nos textes : fuentes, interlineado y estilos de los niveles de títulos.  
\end{intro}

\section{Fuentes}

Hay tres grandes familias de fuentes: 
\begin{glossaire}
\item[Con serifas]para el cuerpo del texto. Estas serifas mejoran el confort de lectura, al menos en la versión de papel. 
\item[A palo seco]normalmente para los títulos en las cubiertas.
\item[De espacio fijo]que se caracterizan por el hecho de que la anchura de los caracteres no varía en función de las letras, sino que permanece uniforme de una letra a otra. Estas fuentes las utilizan los informáticos para citar código informático. En nuestro dominio, pueden servir eventualmente para una transcripción diplomática de un texto antiguo. Para eso utilizaremos el entorno \enviro{verbatim}.
\end{glossaire}

El package \package{fontspec}\footcite{fontspec} permite definir una fuente para cada familia de caracteres. Así, para este libro hemos elegido las fuentes\enquote{Linux Libertine} para los caracteres con serifas, \enquote{Linux Biolinum} para los caracteres a palo seco\footnote{Solo utilizamos esta fuente para la versión informática. Para la versión en papel, hemos encomendado el diseño de la cubierta a Laura Pigeon, que ha elegido su propia fuente.} y \enquote{DejaVu  Sans Mono} para los caracteres con espacio fijo.

Hemos indicado en nuestro preámbulo que deseamos utilizarlas:

\begin{latexcode}
\setmainfont[Mapping=tex-text]{Linux Libertine}
\setsansfont[Mapping=tex-text]{Linux Biolinum}
\setmonofont[Scale=0.75]{DejaVu  Sans Mono}
\end{latexcode}

El comando \csp{setmainfont} sirve para definir la fuente con serifas (cuerpo del texto), \csp{setsansfont} la de a palo seco y \csp{setmonofont} la fuente con espacio fijo. El argumento optativo puede tomar cierto número de parámetros para la fuente. 

Aquí hemos indicado con el parámetro \option{Scale} que deseamos reducir un 25 \% el tamaño de la fuente \enquote{DejaVu Sans Mono}, pues se trata de una fuente bastante grande. 

La opción \option{Mapping=tex-text} de los dos primeros comandos permite gestionar de manera particular ciertas secuencias de caracteres, por ejemplo, reemplazar tres trazos cortos (\verb|---|) por un trazo largo (---)\renvoi{tirets}. 

Hay otros parámetros, por ejemplo, para gestionar sutilmente las ligaduras, que se detallan en el manual de \package{fontspec}\footcite{fontspec_optionspolices}.

\subsection{Fuente específica de ciertas lenguas}\label{policenonlatine}

Unicode permite almacenar todos los caracteres que hay en la tierra\renvoi{utf8}. Sin embargo, las fuentes no son necesariamente compatibles con todos los caracteres de la norma unicode o bien no muestran correctamente ciertas familias de caracteres. 

Tomemos, por ejemplo, la palabra siguiente, en hebreo bíblico\footnote{\bibleverse{Gn}(1:1).}:

\begin{quotation}
\texthebrew{בְּרֵאשִׁ֖ית}
\end{quotation}

Si mantenemos una muestra con la fuente \emph{Linux Libertine}, obtenemos un bonito error de composición:

\begin{quotation}
\texthebrew{\normalfont{בְּרֵאשִׁ֖ית}}
\end{quotation}

La solución es decir a \LaTeX que utilice una fuente específica en el interior del comando \csp{texthebrew} o del entorno \enviro{hebrew}\renvoi{i18n}. En nuestro caso, puede ser la fuente \emph{Ezra SIL}.

Para hacer eso, pediremos a \package{fontspec} que defina una nueva familia de fuentes, \csp{\hebrewfont}, mediante el comando \csp{newfontfamily}. La definición de esta nueva familia permitirá a  \LaTeX cambiar automáticamente la fuente cuando se le señale un cambio a la lengua hebrea.

\begin{latexcode}
\newfontfamily\<leng>font[<parámetros>]{<fuente>}
\end{latexcode}

\begin{description}
\item[\meta{leng}]debe cambiarse por el nombre de la lengua en \package{polyglossia}.
\item[\meta{parámetros}]contienen una lista de parámetros tipográficos para el uso de esta fuente.
\item[\meta{fuente}]es el nombre de la fuente.
\end{description}

Lo que da en nuestro caso:
\begin{latexcode}
\newfontfamily\hebrewfont[Scale=0.8,Script=Hebrew]{Ezra SIL}
\end{latexcode}

\begin{itemize}
\item El parámetro \contenuarg{Scale=0.8} permite mostrar la fuente a un80~\% de su tamaño normal. En efecto, para un número de puntos determinado, no todas las fuentes tienen necesariamente la misma altura de letra. Si dejamos el tamaño estándar, tendríamos un resultado poco estético.
\item El parámetro \contenuarg{Script=Hebrew} permite que la presentación de los glifos se haga según las características de la lengua hebrea. Concretamente, esto permite una presentación correcta de los signos de vocalización\footnote{El alfabeto herbreo es un alfabeto que solo nota las consonantes. Sin embargo, para permitir una interpretación más fácil --- o más orientada, según el punto de vista, algunos textos tienen las vocales anotadas con la ayuda de pequeños signos alrededor de las letras. Es particularmente el caso de los textos de la biblia hebraica \parencite[Voir][4-7]{Verheij2007}.}. 
\end{itemize}

Así obtenemos una presentación correcta:
\begin{quotation}
\texthebrew{בְּרֵאשִׁ֖ית}
\end{quotation}
\section{Interlineado}\label{interligne}

El package \package{setspace} --- que carece de documentación oficial --- permite cambiar con facilidad el interlineado del documento ofreciendo tres interlineados: sencillo, es el ajuste estándar, doble y \enquote{uno y medio}. En general, un interlineado de 1,5 se recomienda para los trabajos académicos, salvo para las notas a pie de página, donde se requiere un interlineado sencillo. Esto queda bien: el package solo aplica el cambio de interlineado al cuerpo del texto. Las notas a pie de página se mantienen por tanto.

\begin{plusloins}
El package \package{setspace}  \enquote{se contenta} con definir nuevos comandos para ejecutar secuencias complejas de comandos \TeX y \LaTeX necesarios para la definición del interlineado teniendo en cuenta el tamaño de la fuente.

Si quieres definir un tamaño distinto de interlineado, puedes tratar de inspirarte en el código del package. Pero eso requiere dominar bastante bien \LaTeX. 
\end{plusloins}

Los tres comandos que nos interesan son:
\begin{itemize}
\item\csp{singlespacing} ; 
\item\csp{onehalfspacing} ; 
\item\csp{doublespacing}.
\end{itemize}

 Se utilizan de la manera siguiente:

\begin{latexcode}
\doublespacing
Este texto tiene interlineado doble.
\onehalfspacing
Este texto tiene un interlineado de 1,5.
\singlespacing
Este texto está en interlineado sencillo. 
\end{latexcode}

Mientras no se utilice ningún comando de cambio de interlineado, el texto mantiene el interlineado actual. 

\subsection{\forme{Comando con argumentos} y \forme{comando de alternancia}}\label{bascule}

Hasta ahora, salvo algunas excepciones, solo hemos visto comandos que reciben varios argumentos. Aquí nuestro comandos no reciben argumento alguno, sino que cambian parámetros hasta nuevo orden. A tales comandos se les llama \emph{comandos de alternancia}. Así, nuestros comandos de alternancia del package \package{setspace} cambian el interlineado de nuestro texto. El uso de los comandos de alternancia es imprescindible para la redefinición de algunos comandos de \LaTeX, incluyendo los comandos de los títulos\renvoi{apparencetitre}.

Un comando de alternancia entre llaves solo tiene efecto dentro del par de llaves. Un comando de alternancia utilizado en un entorno solo tiene efecto en ese entorno. Si un comando de alternancia se utiliza en el preámbulo, en general produce un efecto en el conjunto del documento.\label{porteebascule} 

Ya hemos presentado más arriba comandos de alternancia: los que modifican el tamaño de los caracteres\renvoi{commandesdetaille}.

Se recomienda, por tanto, incluir el comandoI \cs{onehalfspacing} al principio del fichero, por ejemplo, en el preámbulo, para que actúe sobre todo el cuerpo del texto.

\subsection{Entornos de cambio de interlineado}

El package \package{setspace} ofrece igualmente tres entornos que permiten limitar el alcance de un cambio de interlineado. 

\begin{latexcode}
\begin{singlespace}
Texto con interlineado sencillo
\end{singlespace}
\begin{onehalfspace}
Texto con un interlineado de 1,5.
\end{onehalfspace}
\begin{doublespace}
Texto con interlineado doble.
\end{doublespace}
\end{latexcode}

\subsection[Redefinir un entorno: quotation]{(Re)definir un entorno: ejemplo con el entorno \enviro{quotation}}

En general, las citas tienen un interlineado más reducido que el cuerpo del texto. Entonces, para lograr este efecto en un documento con interlineado de 1,5, estaríamos tentados a teclear lo siguiente:

\begin{latexcode}
\begin{singlespace}
\begin{quotation}
Una cita bastante larga.
\end{quotation}
\end{singlespace}
\end{latexcode}

Un método así rompe el principio de separación de fondo y de forma. Además, multiplica las líneas de código. Sería más sensato redefinir el entorno \enviro{quotation}. Veamos cómo hacerlo de manera sencilla\footcite[Nos hemos basado en la clase \classe{bredele} :][]{bredele}. Ya hemos visto más arriba un ejemplo de redefinición de entorno, para el entorno \enviro{latin}\renvoi{redefinirlatin}. 

Se trata aquí no solo de redefinir el entorno, sino también de reutilizar las propiedades del entorno antiguo. Procedemos a incluir este código al comienzo del documento, en el preámbulo:

\begin{latexcode}
\let\oldquotation\quotation
\let\endoldquotation\endquotation
\renewenvironment{quotation}
    {\begin{oldquotation}\singlespace}
        {\end{oldquotation}}
\end{latexcode}

Comentarios: 

\begin{description}
\item[línea 1]el comando \csp{let} es un comando \TeX que copia un comando en otro comando. Aquí copiamos el comando \csp{quotation} en el comando \csp{oldquotation}\footnote{Aunque \enquote{old} sea un término inglés, lo utilizamos porque es una convención prefijar así los comandos copiados de otro comando.}. El comando \csp{quotation} corresponde al inicio del entorno \enviro{quotation}, es decir, a lo que se ejecuta mediante \cs{begin}\verb|{quotation}|.
\item[línea 2]copiamos el comando \csp{endquotation}, que corresponde al final del entorno  \enviro{quotation} --- es decir, a lo que se ejecuta con \cs{end}\verb|{quotation}|, en el interior de un comando \csp{endoldquotation}.
\end{description}

Al crear estos dos comandos \csp{oldquotation} y \csp{endoldquotation} hemos creado un entorno \enviro{oldquotation}.

\begin{description}
\item[línea 3]redefinimos el entorno \enviro{quotation}.
\item[línea 4]al comienzo de este entorno\renvoi{redefinirlatin}, abrimos el entorno \enviro{oldquotation} y después ejecutamos el comando \csp{singlespace}.
\item[línea 5]al final del entorno \enviro{quotation}, cerramos el entorno \enviro{oldquotation}. El comando \csp{singlespace} solo se aplica al contenido del entorno en el que está situado\renvoi{porteebascule}. Por eso no es necesario anularlo mediante un comando \csp{onehalfspace}. 
\end{description}

\begin{plusloins}
Si has comprendido nuestras explicaciones sobre los comandos de inicio y de fin de entorno, comprenderás que habríamos podido obtener el mismo resultado con:

\begin{latexcode}
\let\endoldquotation\endquotation
\renewcommand{\quotation}{\oldquotation\singlespace}
\end{latexcode}

\end{plusloins}

\section{Personalizar los títulos}

\subsection{Redefinir la numeración}\label{apparencecompteur}
Hemos abordado más arriba la noción de contador. A cada nivel de título, \LaTeX asocia un contador \compteur{nivel}. Así el contador asociado al nivel de título \cs{chapter} es \compteur{chapter}\renvoi{tocdepth}.

El valor de un contador se puede mostrar mediante el comandoL \csp{the\meta{contador}}. Así, si escribes:
 
\begin{latexcode}
\thechapter : \thesection
\end{latexcode}

Verás que, tras la compilación, aparece el número del capítulo, tal como aparece en el encabezado, y el número de apartado, tal como aparece en el título. Aquí, por ejemplo, obtenemos:


\begin{quotation}
\thechapter : \thesection
\end{quotation}

Como los contadores se presentan con la ayuda de comandos, es fácil redefinirlos. Si ojeas el fichero\renvoi{trouverfichier} \fichier{book.cls}, puedes constatar\footnote{Líneas 287-288, en el momento en que escribimos.}, las dos (re)definiciones de comando siguientes:

\begin{latexcode}
\renewcommand \thechapter {\@arabic\c@chapter}
\renewcommand \thesection {\thechapter.\@arabic\c@section}
\end{latexcode}

\begin{description}
\item[Línea 1]\csp{thechapter} muestra el valor del contador \compteur{chapter}, devuelto por \csp{c@chapter}. Este valor no se puede mostrar directamente, tiene que estar previamente formateado por el comando \csp{@arabic}, que lo muestra en forma de cifras árabes.
\item[Línea 2]\csp{thesection} devuelve \cs{thechapter} seguido de punto, seguido del valor del contador \compteur{section} que se muestra en forma de cifras árabes.
\end{description}

Imaginemos ahora que deseamos:
\begin{enumerate}
\item Mostrar el número de capítulo en números romanos en mayúsculas.
\item Mostrar un paréntesis de cierre tras el número de apartado.
\end{enumerate}

Nos basta con redefinir así estos comandos:

\begin{latexcode}
\renewcommand\thechapter{\Roman{chapter}}
\renewcommand\thesection{\arabic{section}}
\end{latexcode} 






Y obtenemos una presentación con la forma:

% Aquí se redefine temporalmente 

\makeatletter
\let\oldthechapter\thechapter
\let\oldthesection\thesection
\renewcommand \thechapter {\@Roman\c@chapter}
\renewcommand \thesection {\thechapter.\@arabic\c@section)}
\makeatother
\begin{quotation}
\thechapter~Capítulo 

\thesection~Apartado
\end{quotation}
% Se restablece
\renewcommand{\thechapter}{\oldthechapter}
\renewcommand{\thesection}{\oldthesection}

Comprendemos fácilmente cómo funcionan los comandos \csp{Roman} et \csp{arabic}. Reciben como argumento el nombre de un contador y muestran su valor, ya sea en números romanos en mayúsculas, ya sea en cifras árabes. 

Hay otros comandos similares:
\begin{description}
 \item[\csp{alph}]que muestra un contador cuyo valor está comprendido entre 1 y 26 bajo la forma de una letra minúscula del alfabeto latino. 
  \item[\csp{Alph}]que muestra un contador cuyo valor está comprendido entre 1 y 26 bajo la forma de una letra mayúscula del alfabeto latino. 
 \item[\csp{roman}]que muestra un contador bajo la forma de números romanos en minúsculas.
\end{description}

\begin{plusloins}
El package \package{polyglossia} ofrece además comandos para mostrar números en forma de caracteres griegos, árabes, hebreos, siríacos y farsi\footcite{polyglossia_alphabetic_numbering}. 

Estos argumentos, a diferencia de los anteriores, no reciben como argumento un contador, sino un número. Así, si queremos formatear el número de capítulo bajo la forma de un número griego, tenemos que cargar el griego como lengua y usar el comando de esta manera \csp{greeknumeral}:

\begin{latexcode}
\setotherlanguage{greek}
\renewcommand{\thechapter}{\greeknumeral{\arabic{chapter}}}
\end{latexcode}
\end{plusloins}

\begin{plusloins}
Hablamos más adelante de la forma de manipular los contadores\renvoi{manipcompteurs}.
\end{plusloins}

\subsection{Definir el aspecto: apartados y niveles inferiores}\label{apparencetitre}

Para personalizar el aspecto de los títulos, lo mejor es observar lo que nos ofrecen las clases estándar. Tomemos el caso de la clase \classe{book}, definida en el fichero\renvoi{trouverfichier} \fichier{book.cls}. Si buscamos un poco, encontramos\footnote{Línea 414 con fecha 19 de septiembre de 2011.}:

\begin{latexcode}
\newcommand\section{\@startsection {section}{1}{\z@}%
                             {-3.5ex \@plus -1ex \@minus -.2ex}%
                             {2.3ex \@plus.2ex}%
                             {\normalfont\Large\bfseries}}
\end{latexcode}

Comentemos estas pocas líneas:

\begin{description}
\item[línea 1]El comando \cs{section} llama a \csp{@startsection}. Este último comando añade un nivel de título al índice de contenido. A pesar de su nombre, se puede utilizar para cualquier nivel de título (por ejemplo, para los \cs{subapartado}s). Recibe varios argumentos. El primero es el nivel: aquí \verb|section|. El segundo es el nivel de profundidad: \verb|1|. El tercero es el sangrado previo: aquí \csp{z@}, es decir, una longitud nula.

\item[línea 2] cuarto argumento del comando \cs{@startsection}, que indica el espacio vertical \emph{antes} del título. Este espacio es idealmente de 3,5~ex\renvoi{unite}, pero puede estar entree 1,5~ex ($3,5 - 2$) y 4,5~ex ($3,5 + 1$):  este margen de maniobra se indica mediante los comandos \cs{@minus} et \cs{@plus}\renvoi{elastique}.
\item[línea 3] quinto argumento de \cs{@startsection}, que indica el espacio vertical tras el título. Como nuestra longitud es positiva, el texto siguiente inicia un nuevo párrafo. Una longitud negativa (como es el caso para la definición del comando \cs{paragraph}) indica que no hay cambio de párrafo. Aquí, por tanto, el espacio tras el título es de 2,3 ex, pero puede extenderse a 2,5 ex (2,3 + 0,2).
\item[ligne 4] sexto y último argumento de \cs{@startsection}, que indica el aspecto propiamente dicho de nuestro título. Se trata de un texto en fuente normal --- es decir, la que define el comando \cs{setmainfont} ---, en tamaño \cs{Large}, y en negrita (\cs{bfseries}). Aquí todos nuestros comandos se interpetan como comandos de alternancia, alternancia que acaba con la llave de cierre. El comando \csp{bfseries} es el equivalente alternante del comando \cs{textbf}.En los estilos de título se recomienda usar solamente comandos de alternancia para evitar espacios indeseados.\label{bfseries}
\end{description}

Imaginemos ahora que queremos tener nuestro título en itálica y en una fuente a palo seco. Nos basta con redefinir el comando \cs{section}.
En nuestra (re)-definición, llamaremos a comandos que contienen el carácter @. Pero los comandos que contienen el carácter @ no se pueden utilizar en un fichero \ext{tex}, sino solamente en los ficheros de definición de clase (\ext{cls}) o de package (\ext{sty}). Sin embargo, se puede hacer una excepción si rodeamos estos comandos  de \csp{makeatletter} y \csp{makeatother}.

\begin{latexcode}
\makeatletter
\renewcommand\section{\@startsection {section}{1}{\z@}%
                             {-3.5ex \@plus -1ex \@minus -.2ex}%
                             {2.3ex \@plus.2ex}%
                             {\sffamily\Large\itshape}}
\makeatother
\end{latexcode}

El comando \csp{ssfamily} produce una alternancia a la fuente a palo seco definida por \cs{setsansfont}, mientras que \csp{itshape} produce una alternancia a un texto en itálica.

\section{Definir el aspecto: capítulos y niveles superiores}

Si buscamos en el fichero \fichier{book.cls} el comando \cs{chapter}, encontramos\footnote{Líneas 360 y siguientes con fecha 6 de agosto de 2011.}:

\begin{latexcode}
\newcommand\chapter{\if@openright\cleardoublepage\else\clearpage\fi
                    \thispagestyle{plain}%
                    \global\@topnum\z@
                    \@afterindentfalse
                    \secdef\@chapter\@schapter}
\end{latexcode}

Aquí nos interesan dos elementos. 

El primero es la línea~2 : \cs{thispagestyle}\verb|{plain}|. Se comprende así el motivo por el que las páginas de inicio de capítulo no se numeran en el mismo lugar que las otras: el estilo de esta página es \forme{plain} y no \forme{heading}, como para el resto de las páginas\renvoi{styleentetes}. En consecuencia, si queremos de las páginas de títulos tengan los mismos encabezados y pies de página que las páginas estándar, hay que redefinir el comando \cs{chapter} suprimiendo \verb|\thispagestyle{plain}|.\label{entetechapter}\label{chapitrepagestyle}

\begin{latexcode}
\makeatletter
\renewcommand\chapter{\if@openright\cleardoublepage %
                    \else\clearpage\fi
                    \global\@topnum\z@
                    \@afterindentfalse
                    \secdef\@chapter\@schapter}
\makeatother
\end{latexcode}

El segundo elemento interesante es la última línea. El comando \csp{secdef} remite a \csp{@chapter} si utilizamos \cs{chapter} y a \csp{@schapter} si utilizamos \cs{chapter*}.

Ojeando los códigos de los comandos \cs{@chapter} y \cs{@schapter} encontramos que invocan a \csp{@makechapterhead} y a \csp{@makeschapterhead}, respectivamente.

\begin{plusloins}
\cs{@chapter}, \cs{@schapter}, \cs{@makechapterhead} y \cs{@makeschapterhead} se definen mediante el comando \cs{def} y no \cs{newcommand}. 

En efecto, la definición se hace en \TeX y no en \LaTeX. Para nuestra intención, eso no tiene mayor importancia.
\end{plusloins} 

Fijémonos en el comando \cs{@makechapterhead}\footnote{Línea 386, con fecha 19 de septiembre de 2011.}.

\begin{latexcode}
\def\@makechapterhead#1{%
  \vspace*{50\p@}%
  {\parindent \z@ \raggedright \normalfont
    \ifnum \c@secnumdepth >\m@ne
      \if@mainmatter
        \huge\bfseries \@chapapp\space \thechapter
        \par\nobreak
        \vskip 20\p@
      \fi
    \fi
    \interlinepenalty\@M
    \Huge \bfseries #1\par\nobreak
    \vskip 40\p@
  }}
\end{latexcode}

Analicémoslo:

\begin{description}
\item[línea 2]el comando  \cs{vspace} produce un espacio vertical, aquí de 50~pt, como indica el 50\csp{p@}. La presencia de un asterisco indica que nuestro espacio vertical continúa tras un cambio de página.
\item[línea 3]el comando \cs{parindent} señala el sangrado del párrafo en este punto exacto: aquí, una longitud nula (\cs{z@}). El comando \csp{raggedright} significa que el texto se alineará a la izquierda --- a la inversa, \csp{raggedleft} significaría que el texto se alinearía a la derecha. En cuanto al comando \csp{normalfont}, ni que decir tiene que significa que la fuente estándar --- la que define \cs{setmainfont} --- es la utilizada.
\item[líneas 4-5, 9, 10]se condiciona la presentación de un número de capítulo al posicionamiento en la parte principal del documento, la que sigue a \cs{mainmatter}\renvoi{sectionbook}, y al valor del contador \compteur{secnumdepth}, que podemos redefinir para impedir la numeración de algunos niveles de título\renvoi{manipcompteurs}.
\item[línea 6]se muestra en tamaño \cs{huge}\renvoi{taille} y en negrita (\cs{bfseries}) la cadena de lengua del comienzo del capítulo (\csp{@chapapp}) seguida de un espacio y del número del capítulo: el contador \compteur{chapter}, aparece mediante el comando \cs{thechapter}.
\item[línea 7]se incluye un párrafo (\csp{par}) y se impide un cambio de línea (\csp{nobreak}).
\item[línea 8]se incluye un espacio vertical de 20~pt.
\item[ligne 11]en el nivel a que apunta este libro la comprensión de esta línea no es esencial. No obstante, para los curiosos, esta línea sirve para impedir que un título se extienda en varias líneas. En resumen, podemos decir que para prevenir que haya rupturas de línea o cambios de página, \LaTeX utiliza parámetros llamados \forme{penalty}. Cuanto mayor sea \forme{penalty}, menor será la probabilidad de una ruptura. Aquí, el \forme{penalty} de interlineadoe --- de ruptura de línea --- se define a 1000 (\csp{@M}). 
\item[línea 12]se muestra en tamaño \cs{Huge} y en negrita (\cs{bfseries}) el título del capítulo (\#1). Se introduce luego un párrafo (\csp{par}), procurando evitar los cambios de página (\csp{nobreak}).
\item[línea 13]se incluye un espacio vertical de 40~pt.
\end{description}

Vamos ahora a construir un nuevo estilo\footcite[Nos inspiramos aquí en el estilo de la clase \classe{bredele}][]{bredele}, en el que queremos tener:
\begin{itemize}
\item El texto \forme{capítulo} y el título alineados a la derecha.
\item El texto \forme{capítulo} en versalitas, sin negrita y en tamaño normal.
\item Una raya horizontal bajo el título.
\end{itemize}

Vamos, por tanto, a tomar el código existente para copiarlo y pegarlo en nuestro preámbulo entre \cs{makeatletter} y \cs{makeatother}. Luego haremos los cambios siguientes:

\begin{itemize}
\item Sustituir \csp{raggedright} por \csp{raggedleft} para tener nuestros textos a la derecha..
\item Sustituir el primer \cs{huge}\cs{bfseries} por \csp{scshape}, para tener \forme{capítulo} en versalitas y en tamaño normal.
\item Incluir el comando \cs{hrulefill}, que produce una raya horizontal entre las líneas 12 y 13\renvoi{filets}.
\end{itemize}

Esto nos da al final:

\begin{latexcode}
\makeatletter
\def\@makechapterhead#1{%
  \vspace*{50\p@}%
  {\parindent \z@ \raggedleft \normalfont
    \ifnum \c@secnumdepth >\m@ne
      \if@mainmatter
       \scshape \@chapapp\space \thechapter
        \par\nobreak
        \vskip 20\p@
      \fi
    \fi
    \interlinepenalty\@M
    \Huge \bfseries #1\par\nobreak
    \hrulefill
    \vskip 40\p@
  }}
\makeatother
\end{latexcode}

Evidentemene, habrá que aplicar el mismo tipo de cambios en el comando \csp{makeschapterhead}. También podríamos entretenernos cambiando lo que produce \cs{part}. Para ello, del mismo modo hay que ojear el fichero \fichier{book.cls}.

\section[Manipular los contadores]{Manipular los contadores: el caso de las notas a pie de página}\label{manipcompteurs}

Hemos visto cómo modificar el aspecto de un contador\renvoi{apparencecompteur}. Pero, ¿cómo modificar el valor de un contador? Para eso hay dos comandos: \cs{setcounter}\marg{contador}\marg{valor} para asignar el valor \arg{valor} al contador \arg{contador}, y \csp{addtocounter}\marg{contador}\marg{valor} para sumar el valor \arg{valor} al contador \arg{contador}. Como el valor añadido puede ser negativo, se puede así revertir un contador.

Además, existe un comando \csp{refstepcounter} que permite incrementar en una unidad un contador. Es el que utilizan los comandos de nivel de título\renvoi{niveautitre}.



Un nuevo contador se crea con
 \csp{newcounter}\marg{contador}\oarg{contador2}. 
Si el argumento \arg{contador2} está presente, entonces indica que el valor del contador \arg{contador} se debe reiniciar cuando el comando \csp{refstepcounter} se aplica al contador \compteur{contador2}.

El comando  \csp{@addtoreset}\marg{contador}\marg{contador2}  permite aplicar esta norma a un contador ya existente.

Así, en el fichero \fichier{book.cls}, observaremos la línea siguiente\footnote{Línea 718, con fecha 24 de agosto de 2011} :

\begin{latexcode}
\@addtoreset{footnote}{chapter}
\end{latexcode} 

Esto significa que el contador  \compteur{footnote}, que corresponde a la numeración de las notas a pie de página, se reinicia con cada nuevo capítulo numerado. 

Si queremos anular este comando, para tener notas a pie de página con numeración continua, tenemos que utilizar el package \package{remreset} y su comando \csp{@removefromreset}, que anula una configuración de reinicio.

\begin{latexcode}
\usepackage{remreset}
\makeatletter
\@removefromreset{footnote}{chapter}
\makeatother
\end{latexcode}
