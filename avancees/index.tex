\chapter{Formatear el índice}\label{styleindex}

\begin{intro}
El diseño por defecto del índice no es muy satisfactorio. Por suerte, podemos modificar el índice para hacerlo más legible.
\end{intro}

\section{Comment faire}
Los comandos para formatear el índice hay que ponerlos en un fichero \fichier{miestilo.ist} que debes crear tú mismo.

\begin{attention}
Tienes que colocar este fichero en un lugar en el que \LaTeX lo pueda encontrar, por ejemplo, al lado del fichero principal.
\end{attention}

Sin el package \package{imakedix}, había que compilar en la línea de comandos\renvoi{terminal}, de acuerdo con esta sintaxis:

\begin{bashcode}
makeindex -s miestilo.ist mifichero.idx
\end{bashcode}

Todavía hay que acerlo así cuando pasamos la opción \option{noautomatic} al comando \cs{makeindex} para un índice determinado, como, por ejemplo, para el de las fuentes primarias\renvoi{noautomatic}.

La opción -s indicaba a Makeindex  que debía utilizar el fichero de estilo \fichier{miestilo.ist}.

Al contrario, si no pasas la opción \option{noautomatic} a  \cs{makeindex}, debes pasar la opción \option{options=-s  miestilo.ist}  al comando\cs{makeindex}:

\begin{latexcode}
\makeindex[options=-s  miestilo.ist]
\end{latexcode}

\section{Algunos comandos}

Vamos a dar aquí algunos ejemplos de comandos que podemos incluir en nuestro fichero \fichier{miestilo.ist}. 

\begin{attention}
En el fichero de estilos \ext{ist}, no se escribe en \LaTeX{} sino en \TeX . Los comandos se encierran así entre comillas inglesas \verb|"|. En lo que respecta a la barra invertida, solo se tiene en cuenta si va precedida de otra barra invertida.
\end{attention}

El manual de comandos de MakeIndex\footnote{Que se puede leer tecleando en la terminal\renvoi{terminal} {\ttfamily man makeindex} y que se puede abandonar pulsando la letra  {\ttfamily q}, o, si utilizamos Windows, se puede consultar en esta dirección \url{http://www.fiveanddime.net/man-pages/makeindex.1.html}.} nos indica que, para modificar lo que se incluye entre el primer nivel de item (la entrada) --- o nivel 0 --- y el o los números de página, se emplea la opción \verb|delim_0|; de la misma manera, para definir lo que se incluye entre el segundo nivel de item (la subentrada) --- o nivel 1, y los números, se emplea la opción \verb+delim_1+, y la opción \verb|delim_2| para las subsubentradas --- o nivel 2 de itemización. La elección por defecto es una coma seguida de un espacio en blanco: \verb|", "|

Así, si queremos que el número de la página esté justificado a la derecha y no simplemente saparado de la entrada por una coma y un espacio en blanco, tenemos que utilizar el comando \csp{hfill}\renvoi{hfill} cosa que indicamos en nuestro fichero de estilo con las líneas siguientes:

\begin{latexcode}
delim_0 "\\hfill "
delim_1 "\\hfill "
delim_2 "\\hfill "
\end{latexcode}

Con lo que obtenemos: 

\begin{quotation}

\noindent Adriano\hfill 5 \\
\\
Carlomagno \hfill 5 \\
\\
Duques \\
\hspace*{1em} Tasilón \hfill 5\\
\\
Obispos \\
\hspace*{1em} Dámaso \hfill 5\\
\hspace*{1em} Formoso \hfill  5
\end{quotation}

Si preferimos una línea de puntos, tenemos que sustituir \verb|"|\textbackslash\cs{hfill}\verb| "| por \verb|"|\textbackslash\cs{dotfill}\verb| "|.

\begin{attention}
Los espacios ubicados entre los comandos y las segundas comillas de cierre son indispensables. En efecto,si no incluimos estos espacios, MakeIndex une, en el fichero \ext{ind} generado, los comandos a los números de página, lo que bloquea entonces la complilación, ya que \LaTeX{} se encuentra en ese momento con comandos desconocidos del tipo \csp{hfill1}.
\end{attention}

Si continuamos la lectura del manual, también sabremos cómo hacer que aparezca \enquote{ss.}, o lo que queramos, cuando una misma entrada se indexa en tres o más páginas sucesivas. Basta con indicar \verb+suffix_3p "~ss."+

Veamos ahora cómo se pueden incluir las letras del alfabeto entre los grupos de entradas. Esta modificación permite hacer más legible un índice un poco largo. 

El manual indica que la opción por defecto \verb+headings_flag 0+ no incluye separadores entre los grupos, que la opción \verb+headings_flag 1+ permite obtener letras mayúsculas como separador y la opción  \verb+headings_flag -1+ letras minúsculas.

Definamos a continuación la forma en que aparecerán estas letras. Utilizamos el comando \verb|heading_prefix|. Imaginemos que queremos hacer que estas letras de separación aparezcan en negrita. No olvidemos que hay que escribir el código en \TeX … Así que escribimos:

\begin{latexcode}
heading_prefix "{\\bfseries "
\end{latexcode} 

Si queremos que aparezcan en itálica, escribimos \verb|"|\textbackslash\cs{itshape}\verb|"| (utilizamos \cs{bfseries} e \csp{itshape}, que son comandos \TeX, y no \cs{textbf} \cs{textit}).

Incluso podemos cambiar el tamaño de las letras de separación añadiendo \textbackslash\cs{large} o \textbackslash\cs{Large}, y así sucesivamente. \renvoi{taille}

Pero hemos abierto una llave… tenemos que cerrarla:

\begin{latexcode}
heading_suffix " }\\nopagebreak\n " 
\end{latexcode}

\textbackslash\csp{nopagebreak} evita que una letra de separación se encuentre sola al final de páginal La \verb|\n| es una pequeña coquetería para permitirnos tener, en nuestro fichero \ext{ind}, un cambio de línea tras \csp{nopagebreak}.



No hemos acabado. Nuestras letras están ahora justificadas a la izquierda, como las entradas del índice. Pero queremos que estén centradas: 

\begin{latexcode}
headings_flag 1
heading_prefix " {\\bfseries\\large\\hfill " 
heading_suffix " \\hfill}\\nopagebreak\n " 
\end{latexcode}

Los dos \textbackslash\csp{hfill} estiran, en el primer caso, el espacio que hay antes de la letra a la derecha y, en el segundo, el espacio que sigue a la letra a la izquierda -- y ya están nuestras letras centradas.

\begin{quotation}
\parindent=0pt
\addtolength{\parskip}{\baselineskip}
Adriano \hfill 5 

\noindent{\hfill\large{\textbf{C}}\hfill}

Carlomagno \hfill 5 

{\hfill\large{\textbf{D}}\hfill}

Duques \\
\hspace*{1em} Tasilón \hfill 5

{\hfill\large{\textbf{O}}\hfill}

Obispos \\
\hspace*{1em} Dámaso \hfill 5\\
\hspace*{1em} Formoso \hfill  5
\end{quotation}


Si preferimos que estén simplemente un poco escalonadas a la derecha, podemos utilizar, en vez de \textbackslash\cs{hfill}, \textbackslash\cs{hspace*}\verb|{1em}| que añade a la izquierda un espacio de un cuadratín\renvoi{unite}[espace].


Una vez comprendido este sistema, es bastante sencillo dar formato al índice como queramo. Por tanto, para las demás opciones, remitimos aquí a la lectura del manual\footcite[También se puede consultar][]{frama_index}. 

Ponemos todo junto, a modo de ejemplo, el fichero final:

\begin{latexcode}
headings_flag 1
suffix_3p "~ss."
heading_prefix " {\\bfseries\\large\\hfill " 
heading_suffix " \\hfill}\\nopagebreak\n " 
delim_0 "\\hfill "
delim_1 "\\hfill "
delim_2 "\\hfill "
\end{latexcode}




