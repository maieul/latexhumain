\chapter{Elementos de diseño de página}\label{espacement}

\begin{intro}
En este capítulo, vamos a abordar algunos comandos que permiten modificar el diseño de página.
Recordemos que, normalmente, no hay que utilizar estos comandos tal cual, sino imbricarlos en comandos para dar sentido\renvoi{sensforme}.

Aquí solo se trata de una introducción muy breve: otras obras hablan de ello mejor que nosotros\footcites(En particular)(){frama}{latex_graphic_companion}.
En lo que respecta a las unidades de longitudes, remitimos al anexo~\ref{unite}\renvoi{unite}.
\end{intro}

\section{Espacios}\label{espace}

Se puede producir un espacio horizontal mediante \csp{hspace}\marg{longitud}. Si utilizamos \csp{hspace*}, el final de línea no detiene el espacio.

Un espacio vertical se produce con \csp{vspace}\marg{longitud}. Si utilizamos \csp{vspace*}, el cambio de página no afecta al espacio, que continúa en la página siguiente. Sin embargo, hemos podido constatar que un comando así plantea en ocasiones problemas para el diseño del párrafo que lo precede.



\section{Longitudes de diseño de página}

Para redefinir parámetros como el espacio vertical entre dos párrafos o el sangrado inicial de un párrafo, no hay que utilizar los comandos de espaciado, sino sencillamente redefinir las longitudes de \LaTeX.

Hemos visto que para el interlineado era mejor utilizar el package \package{setspace}\renvoi{interligne}. Nos quedan, por tanto, dos longitudes: el sangrado inicial del párrafo y el espacio entre los párrafos. Estas longitudes son \cs{parindent} y \cs{parskip}.

Para redefinirlas, se utiliza \csp{setlength}\marg{longitud}\marg{valor}. Por ejemplo, para redefinir la longitud del sangrado a 3 cuadratines:\label{setlength}

\begin{latexcode}
\setlength{\parindent}{3ex}
\end{latexcode}

También tenemos \csp{addtolength}\marg{longitud}\marg{valor} que añade \arg{valor} a la longitud ya existente.

Así, para añadir 1 pt al espacio entre dos párrafos:

\begin{latexcode}
\addtolength{\parskip}{1pt}
\end{latexcode}

Estas redefiniciones hay que practicarlas \emph{después} del preámbulo, pues solo se aplican al grupo lógico en que aparecen, un grupo lógico que se corresponde a un entorno o a un espacio delimitado por llaves.

\begin{plusloins}
Podemos suprimir el sangrado de un párrafo determinado si lo iniciamos con el comando \csp{noindent}.
\end{plusloins}

\section{Márgenes}

Aunque no es recomendable hacerlo, se pueden redefinir los márgenes con el package \package{geometry}. Al cargar el package, tenemos las opciones siguientes:
\begin{description}
\item[lmargin]para el margen izquierdo en una impresión a una cara, para el margen interior en una impresión a doble cara.
\item[rmargin]para el margen derecho en una impresión a una cara, para el margen exterior en una impresión a doble cara.
\item[tmargin]para el margen superior.
\item[bmargin]para el margen inferior.
\end{description}

Así, par una impresión con solo un cm de márgen --- cosa que no es aconsejable, a no ser para un borrador --- podemos cargar de este modo el package :

\begin{latexcode}
\usepackage[lmargin=1cm,rmargin=1cm,tmargin=1cm,bmargin=1cm]{geometry}
\end{latexcode}

También podemos modificar temporalmente las medidas para una o varias páginas, mediante el comando \csp{newgeometry}\marg{medidas}. Luego restauramos las medidas iniciales con el comando \csp{restoregeometry}.

Por ejemplo, para tener una página sin marge, escribimos:

\begin{latexcode}
\newgeometry{lmargin=0cm,rmargin=0cm,tmargin=0cm,bmargin=0cm}
Una página sin margen.
\newpage
\restoregeometry
\end{latexcode}

El package \package{geometry} tiene muchas otras funcionalidades: remitimos a su documentación\footcite{geometry}.

\begin{attention}
Si utilizas \package{geometry} y las notas al margen producidas mediante \cs{marginpar}, puedes ver que las páginas de la izquierda (pares) no muestran las notas. El problema surge del hecho de que \package{geometry} redefine cierto número de longitudes. Para que aparezcan correctamente las notas, tienes que modificar la longitud \csp{marginparwidth}, asignándole el mismo valor que \csp{leftmargin}.

\begin{latexcode}
\setlength{\marginparwidth}{\leftmargin}
\end{latexcode}
\end{attention}

\section{Textos sangrados}

Puede que queramos tener un entorno que produzca un texto sangrado, semejante al entorno \enviro{quotation}, como un entorno \enviro{ejemplo} con un sangrado izquierdo de 3~em y un sangrado derecho de 1em.

Si echamos un vistazo al fichero\renvoi{trouverfichier} \fichier{book.cls}, vemos que el entorno quitation se define de la siguiente manera\footnote{Línea 486, el 19 de septiembre de 2011.}:

\begin{latexcode}
\newenvironment{quotation}
               {\list{}{\listparindent 1.5em %
                        \itemindent    \listparindent
                        \rightmargin   \leftmargin
                        \parsep        \z@ \@plus\p@}%
                \item\relax}
               {\endlist}
\end{latexcode}

Que podemos analizar así:
\begin{description}
\item[líneas 2 a 5] código que se ejecuta el comienzo del entorno.
\item[línea 7] código que se ejecuta al final del entorno. Cerramos un entorno de tipo \enviro{list} --- aunque, efectivamente, ¡no se trata de una lista!
\item[línea 2] el comando \csp{list}\marg{itemdefaut}\marg{medidas tipográficas} abre un entorno \enviro{list}. El primer argumento corresponde a lo que muestra el comando \cs{item} en ausencia de argumento optativo. Como en \enviro{quotation} no se utiliza \cs{item}, este primer argumento es nulo. El segundo argumento contiene comandos que definen el diseño. El primero, \csp{listparindent}, indica el sangrado de los párrafos en el interior de la lista: aquí 1,5~em.
\item[línea 3] el segundo comando, \csp{itemindent}, indica el sangrado del primer elemento de cada \cs{item}: aquí a \csp{itemindent} se le asigna el valor de \csp{listparindent}.
\item[línea 4] al margen derecho de nuestra lista (\csp{rightmargin}) se le asigna el mismo valor que al margen izquierdo (\csp{leftmargin}). Este valor se ha definido antes en el fichero \fichier{book.cls}.
\item[línea 5]la longitud que separa dos párrafos en el interior de la lista (\csp{parsep}) se fija a 0~pt (\cs{z@}), pero puede extenderse (\cs{@plus}) hasta 1~pt (\cs{p@})\renvoi{elastique}. El segundo argumento del comando \cs{list} acaba aquí.
\item[línea 7]una lista no puede funcionar sin comando \cs{item}. Por eso se invoca aquí este comando. Para evitar problemas, se indica que el texto de este \cs{item} es nulo, gracias al comando \csp{relax}, que es un comando para no hacer nada mientras \TeX espera un contenido.
\end{description}

Ahora, queremos nuestro entorno \enviro{ejemplo}, con:
\begin{itemize}
\item Un margen izquierdo mayor, de 3 cuadratines (3~em).
\item Un margen derecho de un cuadratín (1~em).
\item Un texto en itálica.
\end{itemize}

Para ello, incluiremos en el segundo argumento del comanto \cs{list} las líneas siguientes:

\begin{latexcode}
\leftmargin 3em
\rightmargin 1em
\itshape
\end{latexcode}

Lo que nos da al final:

\begin{latexcode}
\makeatletter
\newenvironment{exemple}
               {\list{}{\listparindent 1.5em%
                        \itemindent    \listparindent
                        \leftmargin 3em
            \rightmargin 1em
            \itshape
                        \parsep        \z@ \@plus\p@}%
                \item\relax}
               {\endlist}
\makeatother
\end{latexcode}

\begin{plusloins}
Para textos enmarcados o con fondo coloreado, podemos utilizar el package \package{mdframed} y~/~o el package \package{fancybox}\footcite[También podemos consultar, si nos gustan los rétos del código en \TeX~/~\LaTeX,][]{frama_boites}.
\end{plusloins}

\section{Raya horizontal}\label{filets}

Una raya horizontal se trada mediante \csp{rule}\oarg{profundidad}\marg{longitud}\marg{épais\-seur}, donde \arg{profundidad} corresponde a la distancia entre el trazo y la parte inferior de la línea de texto.

También se pueden usar los comandos \csp{hrulefill} y \csp{dotfill} que permiten trazar, respectivamente, líneas y puntos, \enquote{comprimiendo} el texto que va detrás, es decir, estirando al máximo la longitud de la línea o de la secuencia de puntos. También disponemos del comando \csp{hfill}\label{hfill} que permite comprimir el texto que va detrás incluyendo espacios delante.

Ejemplo:

\begin{latexcode}
\hfill Del texto comprimido.

\hrulefill Del texto comprimido.

\dotfill Del texto comprimido.
\end{latexcode}

\begin{quotation}
\hfill Del texto comprimido.

\hrulefill Del texto comprimido.

\dotfill Del texto comprimido.
\end{quotation}

Para trazos más complejos, por ejemplo, en diagonal o punteados, acudiremos al package \package{eepic}\footcite{eepic} o, según el caso, a \package{TikZ}\footcite{tikz}.



