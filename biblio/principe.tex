\chapter[Introducción]{Introducción a la gestión bibliográfica con \LaTeX{}}
\begin{intro}
Abordamos ahora uno de los puntos fuertes de \LaTeX{} para las humanidades: la posibilidad de gestionar de forma muy sencilla una bibliografía compleja, manteniendo cierta flexibilidad en su presentación.

\end{intro}
\section{Principio general}

Hay un módulo para gestionar la bibliografía en la mayoría de los programas WYSIWYG, pero éstos suelen carecer de flexibilidad.

Como consecuencia de ello, la mayoría de los usuarios de estos programas gestionan su bibliografía \enquote{a mano} ---las referencias bibliográficas se escriben textualmente de manera directa en el fichero, lo que entraña múltiples limitaciones:
\begin{itemize}
\item si se quiere cambiar el orden de las informaciones o eliminar algunas, tiene que rehacer uno mismo el conjunto de las referencias;
\item la gestión de las versiones abreviadas de las referencias resulta complicada: hay que asegurarse de que la referencia ya ha sido citada, pero no muy lejos, etc.
\item Tiene que copiar y pegar uno mismo al final el conjunto de las referencias bibliogáficas. 
\end{itemize}

La lógica de la gestión bibliográfica en \LaTeX{} es sencilla. Un fichero \ext{bib} ---eventualmente, varios--- contiene el conjunto de las referencias bibliográficas. En cada referencia, llamada \forme{entrada}, se indican los elementos útiles:  tipo de referencia (libro, artículo, actas de congreso, etc.), título, autor, paginación, subtítulo, editor, etc.\footnote{Además, este fichero puede contener comentarios útiles para la preparación del trabajo: resumen, notas de lectura, signatura en una biblioteca.} Cada referencia tiene una clave única que permite diferenciarla de otra.

En el documento \LaTeX{}, se indica en el preámbulo la ruta al fichero \ext{bib}. Cada vez que se quiera citar una referencia, se emplea un comando al que se le pasa la clave como argumento. El package \package{biblatex} se encarga entonces de mostrar la referencia de acuerdo con un estilo de cita ---es decir, una presentación--- determinado. Así, si queremos cambiar el orden de presentación, por ejemplo, trasponer el editor y la ciudad de publicación, basta con modificar el estilo, cosa que se puede hacer con facilidad, ya que los estilos son una serie de comandos \LaTeX{}. Por otro lado \package{biblatex} gestiona automáticamente las referencias ya citadas e introduce, en función del estilo escogido, las abreviaturas académicas. 

\begin{plusloins}
Se podría prescindir de \package{biblatex}  para gestionar una bibliografía con \LaTeX{}. Sin embargo, las funcionalidades estándar de \LaTeX{} para la gestión bibliográfica son limitadas e inapropiadas para las humanidades. 

Puedes encontrar en internet ficheros \ext{bst}. Estos ficheros son estilos bibliográficos pero para las funcionalidades estándar de  \LaTeX{}. No pueden usarse para \package{biblatex}. Además, es fácil tener tus propios estilos \package{biblatex}, mientras que la sintaxis de los ficheros \ext{bst} es complicada\footnote{Se escribe en notación polaca inversa.}.
\end{plusloins}

Por último, al final del documento o en cualquier otro lugar que se considere apropiado, un comando (o varios) permite mostrar la bibliografía, que recoge las referencias citadas\footnote{También se puede, si se quiere, añadir una o varias referencias no citadas.} en el documento, clasificándolas y presentándolas de acuerdo con un estilo bibliográfico, también él susceptible de personalización.

\section{Una triple compilación}\label{3compil}

Hasta ahora, has compilado una sola vez tu fichero \ext{tex} con \XeLaTeX. Como la gestión bibliográfica es una cuestión bastante complicado, hay que realizar una triple compilación:
\begin{enumerate}
\item el fichero \ext{tex} con \XeLaTeX. Además del fichero \ext{pdf}, esta compilación produce un fichero auxiliar \ext{bbl};
\item este fichero hay que compilarlo con el programa Biber;
\item después hay que volver a compilar el fichero \ext{tex} con \XeLaTeX.
\end{enumerate}


\begin{attention}
Este manual se ha redactado tomando como referencia la versión~2.0 del package \package{biblatex}. Así que tienes que comprobar que dispones de esta versión o de una versión más reciente. Llegado el caso, actualiza \package{biblatex} así como Biber\renvoi{maj} ---por lo general, una versión determinada de \package{biblatex} sólo funciona con una versión determinada de Biber.
\end{attention}

\begin{attention}
La mayoría de los editores especializados en \LaTeX no permite compilar con Biber, sino sólo con BibTeX, que es, en cierta medida, su antecesor. Explicamos en nuestro blog cómo configurar algunos de esos editores para usar Biber en vez de  BibTeX\footcite{biber_logiciels}. No obstante, el método más sencillo consiste en utilizar el programa Latexmk\renvoi{latexmk}, que se encargará de realizar las compilaciones.
\end{attention}






