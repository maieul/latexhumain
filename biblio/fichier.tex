\chapter{Rellenar la base de datos bibliográfica}\label{bddbiblio}

\begin{intro}
La primera etapa para gestionar una bibliografía con \LaTeX es la elaboración de una base de datos bibliográfica que contenga el conjunto de las referencias. Se trata simplemente de un fichero de texto que tiene una extensión \ext{bib} y posee una estructura particular.
\end{intro}


\section{Estructura básica de un fichero \ext{bib}}

Cada elemento de una bibliografía se denomina \forme{entrada}. Una entrada bibliográfica se caracteriza por:
\begin{glossaire}
\item[Un tipo]¿se trata de un artículo, de un libro, de las actas de un congreso?
\item[Una clave única]que permite distinguir entre una entrada bibliográfica y otra. Es la que incorporas en tus ficheros \ext{tex} cuando quieres incluir una referencia bibliográfica.
\item[Campos]que contienen indicaciones sobre la obra, como: autor, título, editor, etc. 
\end{glossaire}

Es fácil de comprender la ventaja de este sistema. En efecto, una corrección en la base de datos bibliográfica se transfiere automáticamente al trabajo cada vez que se cita la entrada, mediante una nueva compilación con Biber.

Un fichero \ext{bib} es un fichero de texto que contiene un conjunto de instrucciones, de la manera siguiente:


\begin{latexcode}
@type{clave,
    Campo1 = {Valor 1},
    Campo… = {Valor …},
    Campon = {Valor n}
}
\end{latexcode}



Por ejemplo:


\inputminted{exemples/biblio/fichier/urner.bib}

Con estas líneas declaramos un libro (\type{book}) al que atribuimos la clave \verb|Urner1952|. Este libro es una obra de Urner, de nombre Hans; se titula \emph{Die ausserbiblische Lesung in Christlichen Gottesdienst} y se publicó en  1952 en Gottigen (Gotinga).


Se puede mostrar en el fichero \ext{pdf} bajo la forma:

\begin{quotation}
\cite{Urner1952}
\end{quotation}


Un fichero \ext{bib} no es más que una serie de entradas de este tipo. Sin embargo, como es bastante fácil \enquote{cruzar los cables}, especialmente en las aperturas y cierres de llaves, merece la pena usar un programa de gestión de bibliografías \renvoi{logicielbiblio}, que permite crear con facilidad ficheros de formato \ext{bib}. Además, estos programas suelen permitir búsquedas cruzadas en la base de datos. Presentamos dos de ellos en un anexo.

La documentación de \package{biblatex} contiene el conjunto de los campos y tipos posibles. Para evitar repetir ese manual, te presentaremos los camplos clasificados por categoría. Pero antes, tenemos que dar algunos consejos para elegir la clave.

\section{Elegir la clave}

Una entrada bibliográfica debe tener una clave única. Esta clave sólo puede tener caracteres alfanuméricos no acentuados, eventualmente con guiones. 
Algunos consejos para elegir la clave:
\subsection{Para las obras contemporáneas}
Tomar el nombre del autor seguido de la fecha. Por ejempo, \verb|Urner1952|.

\subsection{Para las obras antiguas}

Tomar el nombre de la obra, eventualmente abreviado. Así \verb|ContraFelix| para hablar del \citetitle{ContraFelix} de Augustín\footcite{ContraFelix}. Si varias obras de autores diferentes tienen el mismo nombre, poner delante el nombre del autor. Para las obras sin título  y sin autor, se puede indicar también un número en un catálogo de referencia.



\section{Los distintos tipos de entradas}
Veamos los distintos tipos de entradas disponibles\footnote{A decir verdad, puedes añadir las tuyas propias, pero no te lo recomendamos.}.


\begin{choix}

	\item[\type{article}] 
Artículo de una revista. Para las contribuciones de un congreso, véase \type{inproceedings}.
	\item[\type{book}] 
Libro. Para las actas de un congreso, véase \type{proceedings}. 
	\item[\type{booklet}]
	Folleto. Texto en forma de libro, pero sin editor comercial oficial.
	\item[\type{bookinbook}]
	Libro dentro de un libro. Por ejemplo, cuando un autor reune
        sus escritos en un solo volumen físico. 	
	\item[\type{collection}]
	Obra colectiva pero con partes distintas por autor.
	\item[\type{inbook}]
	Parte de un libro.
	\item[\type{incollection}]
	Parte individual (con un autor propio) en una obra colectiva.
	\item[\type{inproceedings}]
	Contribución a un congreso.
	 \item[\type{inreference}]
	Entrada de diccionario, de enciclopedia, etc.
	\item[\type{manual}]
	Manual, no necesariamente en forma impresa.
	 \item[\type{mvbook}]
	Libro en varios volúmenes.
	 \item[\type{mvcollection}]
	Obras colectivas en varios volúmenes, con partes distintas por
        autores.
	 \item[\type{mvproceedings}]
	Actas de un congreso en varios volúmenes.
	 \item[\type{mvreference}]
	Diccionario, enciclopedia, etc. en varios volúmenes.
	\item[\type{misc}]
	 Entrada genérica para todo tipo de entrada de difícil
         clasificación. Por ejemplo: cuadro, manuscrito, \emph{ostracon}. 
	\item[\type{online}]
	Recurso de internet. Si una obra existe también con otra
        forma, elegir el tipo de entrada que le corresponde y utilizar
        el campo \champ{url}.
	\item[\type{patent}]
	Patente industrial.
	\item[\type{periodical}]
	Número exacto de una revista.	
	\item[\type{report}]	
	Inform técnico o de investigación.
	\item[\type{proceedings}]
	Acta de un congreso.
	\item[\type{suppbook}]
	Parte aneja de un libro, como, por ejemplo, el prefacio o los apéndices.
	\item[\type{supperiodical}]
	Suplemento a un número de revista.
	\item[\type{reference}]
	Diccionario, enciclopedia, etc.
	\item[\type{thesis}]
	Tesis doctoral, trabajo de fin de máster o cualquier trabajo
        encaminado a la obtención de un título escolar o universitario.
	\item[\type{unpublished}]
	Obra no publicada.
\end{choix}

Para las ediciones contemporáneas de obras antiguas, la elección de
una entrada de tipo \type{book} puede justificarse, pese al carácter
no libresco de algunas de ellas, como es el caso, por ejemplo, de las
cartas de Agustín. Por el contrario, recomendamos el tipo
\type{misc} para designar una obra no editada.

\begin{plusloins}
  Por lo que respecta a manuscritos, especialmente en el marco de un
  trabajo filológico, se puede utilizar el tipo
  \type{manuscript}. Este tipo no lo define  \package{biblatex} sino
  una extensión de éste último:
  \package{biblatex-manuscripts-philology}\footcite{biblatex-manuscripts-philology}.

  Si un investigador ha editado una obra antigua en un artículo, se
  puede usar el tipo \type{bookinarticle} mediante
  \package{biblatex-bookinarticle}\footcite{biblatex-bookinarticle}.
\end{plusloins}

\section{Los diferentes campos posibles}

No enumeramos aquí todos los campos, sino solamente los que pueden
entrar en las categorías más útiles.

\subsection{Los campos de personas}

Estos campos sirven para designar a personas que han participado en el
proceso de producción de la obra: autor, comentarista, editor
(científico), etc. Por supuesto, no es necesario rellenar todos estos campos.

\begin{choix}
	\item[afterword] Autor(es) del epílogo. 
   	\item[annotator] Autor(es) de anotaciones. 
   	\item[author] Autor(es) de la obra.    
   	\item[bookauthor] Autor(es) del libro en que se incluye la obra. 
   	\item[commentator] Autor(es) de los comentarios. 
   	\item[editor] Editor(es) científico(s). Se puede precisar su
          papel gracias al campo \champ{editortype}.	
   	\item[editora] Editor(es) científico(s) que tiene(n) otro
          papel. Se puede precisar su papel gracias al campo \champ{editoratype}.  
   	\item[editorb] Editor(es) científico(s) que tiene(n) otro
          papel. Se puede precisar su papel gracias al campo \champ{editorbtype}.  
   	\item[editorc] Editor(es) científico(s) que tiene(n) otro
          papel. Se puede precisar su papel gracias al campo
          \champ{editorctype}\footcite[Para estos cuatro campos, véase][]{biblatex_editortype}.
	\item[foreword] Autor(es) del prefacio.
   	\item[holder] Titular de una patente industrial. 
   	\item[introduction] Autor(es) de la introducción. 
   	\item[translator] Traductor(es). 		
\end{choix}

Cuando hay campos que tienen valores iguales, por ejemplo, los campos
\champ{editor} y \champ{translator}, \package{biblatex} fusiona estos
campos en la presentación. Tomemos así la entrada siguiente:

\inputminted{exemples/biblio/fichier/augustin_editeur.bib}

Se presenta de la siguiente manera: 

\begin{quotation}
\cite{DoctrineChretienne}
\end{quotation}

\begin{plusloins}
  Cuando una obra se publica de forma anónima o seudónima, pero se
  desea indicar el nombre del autor que conocen los estudiosos, se
  puede utilizar el campo \champ{realauthor} empleando el  package
  \package{biblatex-realauthor}\footcite{biblatex-realauthor}.
\end{plusloins}

\subsubsection{Cómo incluir un nombre de persona}

Estos campos diferentes toman como valor uno o varios nombres de
persona. Si hay varios nombres, basta con separarlos con la
palabra-clave \forme{and}. Por ejemplo, para los autores de la obra
que tienes en tus manos:

\begin{latexcode}
author = {Maïeul Rouquette and Enimie Rouquette and Brendan Chabannes}
\end{latexcode}

Un nombre contiene los elementos siguientes:
\begin{choix}
	\item[Nombre(s)]La inicial ha de ir en mayúscula, el resto, en
          minúsculas. Así: \forme{Albert}. Es mejor poner el nombre
          completo: se puede confiar la tarea de limitarlo a una
          mayúscula a \package{biblatex}\footnote{Para hacerlo, se
            debe pasar la opción \option{firstinits=true} cuando se
            cargue el paquete.}.
	\item[Apellido(s)]La inicial ha de ir en mayúscula, el resto,
          en minúsculas. Así:
          \forme{Londres}. \package[biblatex]{Biblatex} se encarga,
          llegado el caso, de escribirlo en versalitas.
	\item[Partícula (opción)]Ha de ir entera en minúsculas. Así: \forme{de}.
	\item[Sufijo (opción)]La inicial ha de ir en mayúscula. Este
          tipo de dato es más bien anglosajón. Ejemplo: \forme{Junior}.
\end{choix}


En lo que afecta al orden de los elementos, puede ser:
\begin{itemize}
\item\enquote{Nombres  (partícula)  Apellido};
\item\enquote{partícula Apellido, (sufijo) Nombres}.
\end{itemize}

Así, las entradas \verb|José María Pereda| y
\verb|Pereda, José María| son equivalentes.  En el primer caso,
Biber considera que la última palabra que comienza con una mayúscula
es el apellido familiar. En el segundo caso, considera el conjunto que
hay delante de la coma como el apellido familiar, lo que es útil para
los apellidos dobles. Así, para hablar de Federico García Lorca:
\verb|García Lorca, Federico|.

Si hablamos de Miguel de Cervantes, al ser  \forme{de} una partícula,
podemos emplear la primera sintaxis: en ese caso Biber considera todo
lo que sigue a la particula como el apellido: \verb|Miguel de
Cervantes|.  Pero también funciona la segunda sintaxis: \verb|de Cervantes, Miguel|.


Para distinguir a Alejandro Dumas padre de Alejandro Dumas hijo, se
puede emplear el sufijo: \verb|Dumas, Hijo, Alejandro|.


El caso de los autores antiguos, en los que se escribe con frecuencia
el nombre seguido de la ciudad en que nació o ejerció su labor, como,
por ejemplo, Gregorio de Tours, es problemático. Si escribo: \verb|Gregorio de Tours|
Biber interpretará que se trata de una persona llamada
\forme{Gregorio}, cuyo apellido es \forme{Tours} y la partícula
\forme{de}. En consecuencia, lo presentará con la forma
\forme{Tours, Gregorio, de}. Para evitar este problema, basta con
emplear llaves: \verb|{Gregorio de Tours}|.

Este método también se puede emplear para las instituciones que son
autoras de obras.
Por ejemplo:\verb|{Consejo Superior de Investigaciones Científicas}|.

Por resumir, veamos un ejemplo de algunas entradas
correctas\footnote{El lector exigente disculpará que los autores no
  pongan aquí más que los nombres de autores y títulos de obras.}.

\inputminted{exemples/biblio/fichier/noms.bib}

\begin{quotation}
   \cite{HugoMiserable} 	
   
   \cite{HugoLegende}		
   
   \cite{DeGaulle}			
   
   \cite{BeauvoirSexe}		
   
   \cite{BeauvoirMemoires}	
   
   \cite{Dumas}			
   
   \cite{Gregoire}			
\end{quotation}

\begin{plusloins}
  En lo que atañe a las obras anónimas, la solución más evidente
  consiste en no poner nada en el campo \champ{author}. Sin embargo,
  la presentación por defecto de esas obras anónimas no es del todo
  satisfactoria. Se puede modificar: explicamos como hacerlo en
  nuestro blog\footcite{oeuvresanonymes},
  pero, para comprender nuestro artículo, tendrás que leer los
  siguientes capítulos de esta parte.
\end{plusloins}

\subsection{Campos de título}


\begin{choix}
	\item[booksubtitle]Subtítulo del libro en el que se sitúa la entrada. 
   	\item[booktitle] Título del libro en el que se sitúa la entrada. 		
   	\item[booktitleaddon] Añadido al título del libro en el que se
          sitúa la entrada. 
   	\item[chapter] Capítulo de un libro. Para las entradas de tipo \type{inbook}.	
   	\item[eventitle] Título del congreso, para las entradas de
          tipo \type{proceedings} e \type{inproceedings}.
   	\item[issuesubtitle] Subtítulo de un número específico de una
          revista.  Para las entradas de tipo \type{periodical}, el
          subtítulo de la revista debe ir en el campo
          \champ{subtitle}, y el del número, en el campo \champ{issuesubtitle}		
   	\item[issuetitle] Título de un número específico de una
          revista. Para las entradas de tipo \type{periodical}, el
          título de la revista debe ir en el campo \champ{title}, y el
          del título del número, en el campo \champ{issuetitle}.		
   	\item[journalsubtitle] Subtítulo de una revista.							
   	\item[journaltitle] Título de una revista. El campo
          \champ{journal} es un alias de este campo\footnote{Esto
            significa que completar el campo \champ{journal} equivale
            a completar este campo.}.
   	\item[mainsubtitle] Subtítulo de una obra en varios volúmenes.			
   	\item[maintitle] Título de una obra en varios volúmenes. El
          título del volumen específico de una entrada corresponde al campo \champ{title}.						
   	\item[maintitleaddon]  Añadido al título de una obra en varios
          volúmenes.		
   	\item[origtitle] Título original de la obra, si es una
          traducción. No aparece normalmente. 	
   	\item[subtitle] Subtítulo de la obra.									
   	\item[title] Título de la obra.									
   	\item[titleaddon] Añadido al título de la obra. En este
          manual, recomendamos usarlo para las divisiones de
          fuentes\renvoi{divisionsource}.
\end{choix}

Veamos algunos ejemplos para comprender cómo utilizar estos campos:

\subsubsection{Libro con subtítulo}

\inputminted{exemples/biblio/fichier/saxer.bib}

\begin{quotation}
\cite{Saxer1980}
\end{quotation}

\subsubsection{Libro ubicado en una colección}

\inputminted{exemples/biblio/fichier/felix.bib}

\begin{quotation}
\cite{ContraFelix}
\end{quotation}

\subsubsection{Artículo en una revista}

\inputminted{exemples/biblio/fichier/junod.bib}

\begin{quotation}
\cite{Junod1992}
\end{quotation}

\subsection{Campos de informaciones sobre la publicación}

\begin{choix}
	\item[address]
	Lugar de publicación. Alias del campo \champ{location}.	
	\item[date] 
	Fecha de publicación, con la forma \verb|AAAA-MM-DD/AAAA-MM-DD|.
	La primera fecha indicada corresponde a la fecha de inicio, la
        segunda, a la de fin. Para indicar sólo una fecha de inicio,
        hay que poner \verb|AAAA-MM-DD/|. 
	Podemos omitir el mes o el día. También podemos utilizar los
        campos \champ{year} y \champ{month} en su lugar. 
   	\item[edition]
	Número de edición si hay varias. Tiene que ser un entero o una
        cadena de caracteres.
      \item[eventdate] Fecha de un congreso para las entradas de tipo
        \type{proceedings} e \type{inproceedings}.
	\item[howpublished] Para las entradas de tipo \type{misc},
          indica el modo de publicación.
   	\item[institution] Institución en la que se ha producido la
          obra. Normalmente, para las entradas de tipo \type{thesis}. 
   	\item[issue] Detalle de un número específico de una revista
          (por ejemplo, \enquote{número de verano}). Se han de
          preferir los campos \champ{year}, \champ{month}.	
   	\item[language] Lengua de la obra. Lo ideal es que el nombre
          de la lengua se ponga tal como indica la documentación de
          \package{polyglossia}\footcite{polyglossia}.
   	\item[location] Lugar de publicación.  					
   	\item[month] Mes de publicación. Debe ser un entero entre 1 y 12. 
   	\item[number] Número de una revista o número dentro de una colección. 	
   	\item[organization] Organización que promueve un manual o una
          página de internet.	
   	\item[origdate] Fecha de la edición original.						
   	\item[origlanguage] Lengua original. Lo ideal que el nombre de
          la lengua se ponga tal como se indica en la documentación de
          \package{polyglossia}\footcite{polyglossia}.
   	\item[origlocation] Lugar de impresión de la edición original.		
   	\item[origpublisher] Editor (comercial) de la edición original.		
	\item[pages] Páginas del artículo o de la parte del libro estudiada. 
	\item[pagetotal] Número total de páginas.
   	\item[part] Para los libros con varios volúmenes
          \emph{físicos}, indica el número de volumen físico. El
          número del volumen  \emph{lógico} hay que indicarlo en el campo \champ{volume}.
   	\item[publisher] Editor comercial.					
   	\item[pubstate] Para las obras que todavía no se han
          imprimido, indica su estado:
					\begin{description}
						\item[inpress]obra en imprenta.
						\item[inpreparation]obra en preparación.
						\item[submitted]obra sometida a evaluación.
					\end{description}
					
					
   	\item[type] Para las entradas de tipo  \type{thesis},
          \type{patent} y \type{report}, precisa el tipo de trabajo.
	
	Se puede poner un valor personal o bien tomar uno de los
        valores predefinidos. En ese caso, el valor se traduce
        automáticamente a la lengua del documento.
	
	 Para las entradas de tipo \type{thesis}, dos posibilidades:
					\begin{description}
						\item[mathesis]memoria
                                                  de fin de máster.
						\item[phdthesis]tesis doctoral.
					\end{description}
					
					Para las entradas de tipo
                                        \type{patent}, varias posibilidades: 
					 
					 
					 \begin{description}
						\item[patentde]
                                                  Patente alemana.
						\item[patenteu]
                                                  Patente europea.
						\item[patentfr]
                                                  Patente francesa.
						\item[patentuk]
                                                  Patente británica.
						\item[patentus]
                                                  Patente estadounidense.
						\item[etc.]
					 \end{description}
					
					Para las entradas de tipo \type{report}: \nopagebreak
					\begin{description}
						\item[techreport]informe
                                                  téncnico.
						\item[resreport]informe
                                                  de investigación.
					\end{description}
					
	\item[url] Url (dirección electrónica) de una publicación on line. 
   	\item[urldate] Fecha de consulta de una publicación electrónica. 
   	\item[venue] Lugar del congreso para las entradas de tipo \type{proceedings} e \type{inproceedings}. 
   	\item[version] Número de revisión de un manual, de un programa. 
   	\item[volume] Volumen en una obra en varios volúmenes. Volumen
          de una revista. 
   	\item[volumes] Número de volúmenes en una obra en varios volúmenes. 
   	\item[year] Año de publicación. 				
\end{choix}


Por supuesto, no hay que rellenar siempre de forma sistemática todos
estos campos: el lector sabe mejor que nosotros cuáles rellenar en
función de sus necesidades. Algunos campos pueden contener varios
valores que basta separar con la palabra clave \forme{and}. Tomemos un
libro copublicado por las ediciones \forme{Labor et Fides} y
\forme{Cerf}: para indicar los dos editores, hay que escribir:
\verb|publisher ={Labor et Fides and Cerf}|.


\begin{attention}
Para las revistas que publican un mismo volument en varias etapas, hay
que utilizar:
\begin{description}
	\item[\champ{Volume}] para la numeración principal. Dentro de
          un volumen, la paginación suele comenzar en~1.
	\item[\champ{Number}] para la numeración secundaria. Dentro de
          un número, la paginación comienza en el punto en que se ha
          parado en el número anterior. Normalmente, cada nuevo
          volumen reinicia el valor del campo \champ{number}.
\end{description}

Si un número no tiene generalmente índice de contenidos, un volumen
tiene uno que comprende todos los números del volumen.

Así:

\inputminted{exemples/biblio/fichier/Gounelle2006.bib}

\begin{quotation}
\cite{Gounelle2006}
\end{quotation}
\end{attention}

\subsection{Los campos de identificación legal}

Puede resultar útil indicar informaciones como el ISBN, etc. Veamos
los campos posibles\footcite[Por defecto, \package{biblatex} imprime
estos campo si se rellenan. Pero siempre se puede impedir que que se
vean, pasando la opción \option{isbn=false} cuando se carga el
package, véase:][]{biblatex_isbn}.  El lector curioso puede encontrar
fácilmente información sobre su significado.

\begin{choix}
	\item[eid] Identificador electrónico de una entrada de tipo \type{article}. 
   	\item[isan] \emph{\textenglish{International Standard
              Audiovisual Number}}, para las entradas de tipo audiovisual.
   	\item[isbn] \emph{\textenglish{International Standard Book
              Number}}, para los libros. 
   	\item[ismn] \emph{\textenglish{International Standard Music
              Number}}, para la música impresa, como, por ejemplo, las
          partituras. 
   	\item[isrn] \emph{\textenglish{International Standard
              Technical Report Number}}, para los informes técnicos. 
   	\item[issn] \emph{\textenglish{International Standard Serial
              Number}}, para los números de revistas. 
   	\item[iswc] \emph{\textenglish{International Standard Work
              Code}} para las obras musicales.
\end{choix}

\subsection{Campos de notas}

Por defecto estos campos no se imprimen. 

\begin{choix}
	\item[abstract] Resumen de la obra. 
   	\item[annotation] Nota sobre la obra.
   	\item[file] Dirección de una versión informática local del trabajo. 
   	\item[library] Nota sobre la disponibilidad en una biblioteca,
          por ejemplo, el nombre de la biblioteca y la signatura.
\end{choix}

\begin{plusloins}
Puede resultar interesante producir de esta forma una biliografía
comentada. El autor de estas líneas ha publicado en su página web un
método para hacerlo\footcite{biblio_commentee}. Sin embargo, antes de
ponerse manos a la obra con este método, recomendamos leer los
capítulos que vienen a continuación.
\end{plusloins}

Hay otros campos: si el corazón te impulsa a ello, siempre puedes
consultar el manual de \package{biblatex}\footcite{biblatex_champs}.
