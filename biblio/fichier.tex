\chapter{Remplir sa base de donnée bibliographique}\renvoi{bibliofichier}

\begin{prealable}
La première étape pour gérer une bibliographique avec \LaTeX est d'entrer l'ensemble des références dans une base de donnée bibliographique. La base de donnée bibliographique est un simple fichier texte qui porte une extension \ext{bib} et possède une structure particulière.
\end{prealable}


\section{Structure de base d'un fichier \ext{bib}}

Chaque élément d'une bibliographie est appelé entrée. Une entrée bibliographique se caractérise par :
\begin{description}
\item[Un type d'entrée]s'agit-il d'un article, d'un livre, d'actes de colloques ?
\item[Une clef unique] qui permet de distinguer une entrée bibliographique d'une autre. C'est elle qui sera mentionnée dans vos fichiers \ext{tex} lorsque vous voudrez insérer une référence bibliographique.
\item[Des champs] contenant des indications sur l'ouvrage, tels que : auteur, titre, éditeur etc. 
\end{description}

L'avantage de ce système est aisément compréhensible. En effet, une correction dans la base de donnée bibliographique sera automatiquement reportée dans le travail\footnote{Moyennant une nouvelle compilation avec BibTeX.} à chaque fois que l'entrée est mentionnée.

Un fichier \ext{bib} est un fichier texte contenant une suite d'instruction, de la forme suivante.


\begin{minted}{latex}
@type{clef,
	Champ1 = {Valeur 1},
	Champ… = {Valeur …},
	Champn = {Valeur n}
}
\end{minted}



Par exemple

\inputminted{exemples/biblio/fichier/urner.bib}

Signifie que je déclare un livre (\verb|@book|) auquel j'attribue la clef \verb|Urner1952|. Ce livre est une œuvre de Urner, Hans de son prénom ; il s'intitule \emph{Die ausserbiblische Lesung in Christlichen Gottesdienst} et a été publié en 1952 à Gottigen.


Il pourra être affiché dans le fichier .pdf sous la forme :

\begin{quotation}
\cite{Urner1952}
\end{quotation}



Un fichier \ext{bib} n'est rien d'autre qu'une \renvoi{logiciel:biblio} série d'entrées de ce type. Cependant, comme il est assez aisé de \enquote{s'emmêler les pinceaux}, notamment dans les ouvertures et fermetures des accolades, il vaut la peine d'utiliser un logiciel de gestion de bibliographies, capable de gérer  des fichiers au formats \ext{bib}. Un certain nombre sont présentés en annexe. En outre ces logiciels permettent en générale des recherches croisés dans la base de donnée.

La documentation de \package{BibLaTeX} contient l'ensemble des champs et des types possibles. Pour éviter de répéter ce manuel, nous vous présenterons les champs en les classant par catégorie. Mais auparavant, nous donnerons quelques conseils sur le choix de la clef et sur les différents types de catégories.

\section{Le choix de la clef}

Une entrée bibliographique doit avoir une clef unique. Cette clef doit comporter uniquement des caractères alphanumériques non accentués, éventuellement avec des tirets. 




Voici les conseils que nous donnons pour le choix de la clef :
\subsection{Pour les œuvres contemporaines}
Notamment les sources secondaires, il faut prendre le nom de l'auteur suivi de la date. Par exemple \verb|Urner1952|.

\subsection{Pour les œuvres anciennes}

Prendre le nom de l'œuvre  éventuellement abrégée : par exemple \verb|ContraFelix| pour parler du \citetitle{ContraFelix} d'Augustin\footcite{ContraFelix}. Si plusieurs œuvres d'auteurs différents portent le même nom, faire précéder du nom de l'auteur. On peut aussi pour les œuvres sans titres et sans auteurs indiquer un numéro dans un catalogue de référence.



\section{Les différentes types d'entrées}
Voici les différents types d'entrées disponibles\footnote{A vrai dire il est possible de rajouter les siennes, mais nous vous le déconseillons}.


\begin{fieldlist}

	\fielditem{article} 
Article d'un périodique. Pour les communications de colloques, voir @inproceedings.

	\fielditem{book} 
Livre. Pour les actes de colloques, voir plus bas @proceedings. 

	\fielditem{booklet}
	
	Livret. Texte sous la forme d'un livre mais sans éditeur (commercial) officiel.
	
	\fielditem{bookinbook}
	
	Livre dans un livre. Par exemple lorsque un auteur  voit ses livres rassemblés en un seul volume physique. 	
	
	\fielditem{collection}
	
	Ouvrage collectif mais avec des parties distinctes par auteurs.
	
	\fielditem{inbook}
	
	Partie d'un livre.
	
	\fielditem{incollection}
	
	Partie individuelle (avec un auteur propre) dans un ouvrage collectif.
	
	\fielditem{inproceedings}
	 
	 Contribution à un colloque.
	 

		 
	 \fielditem{inreference}
	 
	 Article de dictionnaire, d'encyclopédie ou apparenté.
	 

	\fielditem{manuel}
	
	 Manuel, pas nécessairement sous forme imprimée.
	 
	 \fielditem{mvbook}
	 Livre en plusieurs volumes.
	 
	 \fielditem{mvcollection}
	 
	 Ouvrages collectifs en plusieurs volumes, avec des parties distinctes par auteurs.
	 
	 \fielditem{mvproceedings}
	 
	 Actes de colloques en plusieurs volumes.
	 
	 \fielditem{mvreference}
	 
	 Dictionnaire, encylopédie ou apparenté en plusieurs volumes.


	\fielditem{misc}
	 Entrée générique, pour tout type d'entrée non catégorisable. Par Exemple : tableau, manuscrit, ostracon. 
	\fielditem{online}
	Ressource interne. Si une une œuvre existe sous deux formes, prendre l'une des autres entrées et utilise le champ \champ{url}.
	
	\fielditem{patent}
	Brevet industriel.

	\fielditem{periodical}
	
	Numéro précis d'un périodique.
	
	\fielditem{report}	
	
	Rapport technique ou de recherche.
		
	\fielditem{proceedings}
	Actes de colloques.
	
	
		
	\fielditem{suppbook}
	
	Partie annexe d'une livre, comme par exemple la préface ou les annexes.
	
	\fielditem{supperiodical}
	Supplément à un numéro de périodique.
	
	\fielditem{reference}
	
	Dictionnaire, encyclopédie ou apparenté.

	\fielditem{thesis}
	
	Thèse de doctorat, mémoire de maîtrise ou tout ouvrage rédigé en vue de l'obtention d'un titre scolaire ou universitaire.
	
	\fielditem{unpublished}
	
	Ouvrage non publié.
	
\end{fieldlist}


\section{Les différents champs possibles}

Nous ne listons pas ici tout les champs, mais seulement ceux qui peuvent rentrer dans les catégories les plus utiles.
\subsection{Les champs de personnes}

Ces champs servent à désigner des personnes qui ont participé au processus de production de l'œuvre : auteur, annotateur, éditeur (scientifique) etc. Tous ces champs ne sont évidemment pas à remplir. Voici la liste  :


\begin{fieldlist}
	\fielditem{afterword} Auteur(s) de la postface. 
   	\fielditem{annotator} Auteur(s) des annotations. 
   	\fielditem{author} Auteur(s) de l'œuvre.    
   	\fielditem{bookauthor} Auteur(s) du livre dans lequel l'œuvre est insérée. 
   	\fielditem{commentator} Auteur(s) des commentaires. 
   	\fielditem{editor} Éditeur(s) scientifique(s). On peut préciser son rôle grâce au champ \champ{editortype}.	
   	\fielditem{editora} Éditeur(s) scientifiques ayant un autre rôle. On peut préciser son rôle grâce au champ \champ{editoratype}.  
   	\fielditem{editorb} Éditeur(s) scientifiques ayant un autre rôle. On peut préciser son rôle grâce au champ \champ{editorbtype}.  
   	\fielditem{editorc} Éditeur(s) scientifiques ayant un autre rôle. On peut préciser son rôle grâce au champ \champ{editorctype}\footcite[Pour ces quatres champs, se reporter à][]{biblatex_editortype}.
	\fielditem{foreword} Auteur(s) de la préface.
   	\fielditem{holder} Titulaire d'un brevet industriel. 
   	\fielditem{introduction} Auteur(s) de l'introduction. 
   	\fielditem{translator} Traducteur(s). 		
\end{fieldlist}

Lorsque des champs possèdent des valeurs identiques (par exemples les champs \champ{publisher} et \champ{translator}), \package{BibLaTeX} va  fusionner ces champs lors de l'affichage. Prenons ainsi l'entrée suivante : 

\inputminted{exemples/biblio/fichier/augustin_editeur.bib}

Elle sera affichée ainsi : 

\begin{quote}
\cite{DoctrineChretienne}
\end{quote}

\subsubsection{Comment entrer un nom de personne}

Ces différents champs prennent comme valeur un ou plusieurs noms de personnes. S'il y a plusieurs noms, il suffit de les séparer par le mot-clef \forme{and}. Par exemple pour les auteurs de l'ouvrage que vous avez entre les mains : 

\begin{minted}{latex}
author = {Maïeul Rouquette and Enimie Rouquette}
\end{minted}

Un nom se caractérise en LaTeX par :
\begin{description}
	\item[Un ou plusieurs prénoms.]L'initiale doit être en majuscule, le reste en minuscule. Exemple : \forme{Albert}. Il vaut mieux mettre le prénom complet : on peut confier le travail de limitation à une minuscule  à \package{BibLaTex}\footnote{Il faut pour cela passer l'option \option{firstinits=true} lors du chargement du package.}.
	\item[Un nom.]L'initiale doit être en majuscule, le reste en minuscule. Exemple : \forme{Londres} et non pas \forme{LONDRES}. \package{BibLaTex} se chargera le cas échéant de mettre en petites capitales.
	\item[Éventuellement une particule.]Elle doit être entièrement en minuscule. Par exemple : \forme{de}, \forme{von}.
	\item[Éventuellement un suffixe] avec l'initiale en majuscule. Ce type de donnée est plutôt anglo-saxonne. Exemple : \forme{Junior}.
\end{description}

En ce qui concerne l'ordre des éléments, il  peut être :
\begin{itemize}
\item\enquote{Prénoms  (particule)  Nom}
\item\enquote{particule Nom, (suffixe) Prénoms} 
\end{itemize}

Ainsi les entrées suivantes sont équivalentes :

\begin{minted}{latex}
Victor Marie Hugo
\end{minted}

\begin{minted}{latex}
Hugo, Victor Marie
\end{minted}

Dans le premier cas, BibTeX considère que le dernier mot commençant par une majuscule est le nom de famille. Dans le second cas, il considère l'ensemble situé avant la virgule comme le nom de famille. Ce qui est utile pour les noms composés. Ainsi pour parler de Charles De Gaulle :

\begin{minted}{latex}
De Gaulle, Charles
\end{minted}

Si je parle de Simone de Beauvoir, le \forme{de} étant une particule je peux utiliser la première syntaxe : dans ce cas BibLaTeX considère tout ce qui suit la particule comme constituant le nom :

\begin{minted}{latex}
Simone de Beauvoir
\end{minted}

Mais la seconde syntaxe fonctionne également :

\begin{minted}{latex}
de Beauvoir, Simone
\end{minted}

Pour distinguer Alexandre Dumas père d'Alexandre Dumas fils, on peut utiliser le suffixe :

\begin{minted}{latex}
Dumas, Fils, Alexandre
\end{minted}

Un cas problématique est celui des auteurs anciens, où l'on écrit souvent le prénom suivi de la ville d'exercice ou de naissance, comme par exemple pour Grégoire de Tours. Si j'écris : 

\begin{minted}{latex}
Grégoire de Tour
\end{minted}

\package{BibLaTeX} va comprendre qu'il s'agit d'une personne prénommée \forme{Grégoire}, dont le nom est \forme{Tours} et la particule \forme{de}. Par conséquent il va l'afficher sous la forme \forme{Tours, Grégoire, de.}. Pour éviter ce problème, il suffit d'utiliser des accolades :

\begin{minted}{latex}
{Grégoire de Tour}
\end{minted}

À noter que cela peut servir aussi pour les institutions auteurs d'ouvrages.

Par exemple :
\begin{minted}{latex}
{Centre Nationale de la Recherche Scientifique}
\end{minted}

Pour résumer, voici un exemples de quelques entrées correctes\footnote{Le lecteur exigeant pardonnera aux auteurs de ne mettre ici que les noms d'auteurs et titres d'œuvres.}.

\inputminted{exemples/biblio/fichier/noms.bib}

\begin{quote}
   \cite{HugoMiserable} 	\\
   \cite{HugoLegende}		\\ 
   \cite{DeGaulle}			\\	
   \cite{BeauvoirSexe}		\\ 
   \cite{BeauvoirMemoires}	\\ 
   \cite{Dumas}				\\
   \cite{Gregoire}			\\
\end{quote}

\subsection{Champs de titre}


\begin{fieldlist}
	\fielditem{booksubtitle}Sous-titre du livre dans lequel l'entrée se situe. 
   	\fielditem{booktitle} Titre du livre dans lequel l'entrée se situe. 		
   	\fielditem{booktitleaddon} Ajout au titre du livre dans lequel l'entrée se situe 
   	\fielditem{chapter} Chapitre d'un livre. Pour les entrées de type @inbook	
   	\fielditem{eventitle} Titre du colloque, pour les entrées de type @proceedings et @inproceedings 
   	\fielditem{issuesubtitle} Sous-titre d'un numéro spécifique d'un périodique. 	Pour les entrées de type @periodical, le sous-titre du périodique doit aller dans le champ \champ{subtitle}, celui du numéro dans le champ \champ{issusubtitle}		
   	\fielditem{issuetitle} Titre d'un numéro spécifique d'un périodique. Pour les entrées de type @periodical, le titre du périodique doit aller dans le champ \champ{title}, celui titre du numéro dans le champ \champ{issutitle}		
   	\fielditem{journalsubtitle} Sous-titre d'un périodique.							
   	\fielditem{journaltitle} Titre d'un périodique. Le champ \champ{journal} est un alias de ce champs\footnote{C'est à dire que remplir le \champ{journal} revient à remplir ce champ.}.							
   	\fielditem{mainsubtitle} Sous-titre d'une œuvre en plusieurs volumes.			
   	\fielditem{maintitle} Titre d'une œuvre en plusieurs volumes. Le titre du volume spécifique à une entrée correspond au champs \champ{title}.						
   	\fielditem{maintitleaddon}  Ajout au titre d'une œuvre en plusieurs volumes.		
   	\fielditem{origtitle} Titre original de l'œuvre, si traduction. N'est pas affiché en standard. 
   	\fielditem{reprinttitle} Titre d'un reprint. N'est pas affiché en standard.	
   	\fielditem{subtitle} Sous-titre de l'œuvre.									
   	\fielditem{title} Titre de l'œuvre.									
   	\fielditem{titleaddon} Ajout au titre de l'œuvre. Dans ce manuel, nous conseillons de l'utiliser pour les divisions de source. \renvoi{divisionsource}
\end{fieldlist}

Voici quelques exemples afin de comprendre comment se servir de ces champs.

\subsubsection{Un livre avec un sous-titre}

\inputminted{exemples/biblio/fichier/saxer.bib}

\begin{quote}
\cite{Saxer1980}
\end{quote}

\subsubsection{Un livre situé dans un recueil}

\inputminted{exemples/biblio/fichier/felix.bib}

\begin{quote}
\cite{ContraFelix}
\end{quote}

\subsubsection{Un article dans une revue}

\inputminted{exemples/biblio/fichier/junod.bib}

\begin{quote}
\cite{Junod1992}
\end{quote}

\subsection{Champs d'informations sur la publication}

\begin{fieldlist}
	\fielditem{adress} 
	
	Lieu de publication. Alias du champ \champ{location}.	
   	
	\fielditem{date} 
	Date de publication, sous la forme \verb|AAAA-MM-JJ/AAAA-MM-JJ|.
	La première date indiqué correspondant à la date de début, la seconde à la date de fin. Pour n'indiquer qu'une date de début, mettre \verb|AAAA-MM-JJ/|. 
	
	On peut ne pas indiquer le mois ou le jour. On peut également utiliser les champs \champ{year} et {month} à la place. 
   	\fielditem{edition} 
	
	Numéro d'édition, si plusieurs éditions existent. Doit être un entier ou bien une chaîne de caractères.				
   	\fielditem{eventdate} Date du colloque pour les entrées du type @proceedings et @inproceedings. 
	\fielditem{howpublished} Pour les entrées de type @misc, précise le mode de publication.
   	\fielditem{institution} Institution dans laquelle l'œuvre a été produite. Typiquement pour les entrées de type @thesis. 
   	\fielditem{issue} Spécification d'un numéro spécifique d'un périodique (par exemple \enquote{numéro d'été}). On préférera les champs \champ{year}, \champ{month}.	
   	\fielditem{language} Langue de l'œuvre. Le nom de la langue doit, idéalement, être comme indiqué dans la documentation de \package{polyglossia}.					
   	\fielditem{location} Lieu d'impression.  					
   	\fielditem{month} Mois de publication. Doit être un entier compris entre 1 et 12. 
   	\fielditem{number} Numéro d'un périodique ou numéro au sein d'une collection. 	
   	\fielditem{organization} Organisation à l'origine d'un manuel ou d'une page internet.	
   	\fielditem{origdate} Date de l'édition originale.						
   	\fielditem{origlanguage} Langue originelle. Le nom de la langue doit, idéalement, être comme indiqué dans la documentation de \package{polyglossia}. 
   	\fielditem{origlocation} Lieu d'impression de l'édition originelle.		
   	\fielditem{origpublisher} Éditeur (commercial) de l'édition originelle.		
	\fielditem{pages} Pages de l'article ou de la partie du livre étudiée. 
	\fielditem{pagetotal} Nombre total de pages.
   	\fielditem{part} Pour les livres en plusieurs volumes, indique le numéro du volume physique. 
   	\fielditem{publisher} Éditeur commercial.					
   	\fielditem{pubstate} Pour les œuvres non encore imprimées, indique le statut :
					\begin{description}
						\item[inpress]œuvre sous presse.
						\item[inpreparation]œuvre en préparation.
						\item[submitted]œuvre soumise à évaluation.
					\end{description}
					
					
   	\fielditem{type} Pour les entrées de type  @manual, @patent et @report, précise le type de travail.
	
	 Pour les entrées de type @manual, deux valeurs possibles :
					\begin{description}
						\item[mathesis]mémoire de master.
						\item[phdthesis]thèse de doctorat.
					\end{description}
					
					 Pour les entrées de type @patent, plusieurs  valeurs possibles : 
					 
					 
					 \begin{description}
						\item[patentde] Brevet allemand.
						\item[patenteu] Brevet européen.
						\item[patentfr] Brevet français.
						\item[patentuk] Brevet grand-breton.
						\item[patentus] Brevet États-Unien.
						\item[etc.]
					 \end{description}
					
					Pour les entrées de type @report : 
					\begin{description}
						\item[techreport]rapport technique.
						\item[resreport]rapport de recherche.
					\end{description}
					
	\fielditem{url} Url (adresse électronique) d'une publication en-ligne. 
   	\fielditem{urldate} Date à laquelle une publication électronique a été consultée. 
   	\fielditem{venue} Lieu du colloque pour les entrées du type @proceedings et @inproceedings. 
   	\fielditem{version} Numéro de révision d'un manuel, d'un logiciel. 
   	\fielditem{volume} Volume dans une œuvre en plusieurs volumes. 
   	\fielditem{volumes} Nombre de volume dans une œuvres en plusieurs volumes. 
   	\fielditem{year} Année de publication. 				
\end{fieldlist}


Évidemment tout ces champs ne sont pas à systématiquement remplir : le lecteur sera mieux que nous lesquels remplir. Certains champs peuvent contenir différentes valeurs, qu'il suffit de séparer par le mot-clef \forme{and}. Prenons un livre co-publié par les éditions \forme{Labor et Fides} et \forme{Cerf} : pour indiquer les deux éditeurs, il faut mettre :

\begin{minted}{latex}
publisher ={Labor et Fides and Cerf}
\end{minted}




\subsection{Les champs d'identifications formels}

Il peut être utile d'indiquer des informations comme l'ISBN etc. Voici les champs possibles\footcite[BibLaTeX imprimera ces champs par défaut, il est toutefois possible de ne pas les afficher en passant une option au chargement du package]{biblatex_isbn}. Nous signalons ici la liste de ces champs. Le lecteur curieux trouvera aisément des informations sur leurs significations.

\begin{fieldlist}
	\fielditem{eid} Identifiant électronique d'une entrée de type @article. 
   	\fielditem{isan} \textenglish{International Standard Audiovisual Number}, pour les entrées de type audiovisuel.
   	\fielditem{isbn} \textenglish{International Standard Book Number}, pour les livres. 
   	\fielditem{ismn} \textenglish{International Standard Music Numbe}, pour les musiques imprimées, comme par exemple les partitions. 
   	\fielditem{isrn} \textenglish{International Standard Technical Report Number}, pour les rapports techniques. 
   	\fielditem{issn} \textenglish{International Standard Serial Number}, pour les numéros de revue. 
   	\fielditem{iswc} \textenglish{International Standard Work Code} pour les œuvres musicales.
\end{fieldlist}

\subsection{Champs d'annotations}

Ces champs ne sont pas imprimés par défaut. 

\begin{fieldlist}
	\fielditem{abstract} Résumé de l'œuvre. 
   	\fielditem{annotation} Annotation sur l'œuvre.
   	\fielditem{file} Adresse d'une version informatique du travail. 
   	\fielditem{library} Annotation sur la disponibilité en bibliothèque, par exemple mettre le nom de la bibliothèque et la cotation.
	
\end{fieldlist}

Il existe d'autres champs : nous les présenterons en temps utile. Si le cœur vous en dit, vous pouvez toujours consulter le manuel de \package{BibLateX}\footcite{biblatex_champs}.
