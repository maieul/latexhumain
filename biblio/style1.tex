\chapter{Modificar los estilos bibliográficos (1)}\label{style1}

    \begin{intro}
    
   Ya sabes emplear los comandos para citar. Ahora quizá desees modificar la forma en que aparecen las referencias: cambiar el orden de los campos, los separadores, los textos de tipo \forme{en}.
    
   Estos dos capítulos te indicarán los capítulos generales para hacer esas modificaciones. No tienen como finalidad presentar todo sobre el asunto, sino simplemente los principios básicos. Se remitirá al manual de \package{biblatex} para profundizar.
    
   El primero de estos capítulos te presentará las técnicas más sencillas que no requieren profundizar en los estilos bibliográficos estándar. 
    
    
    \end{intro}
    
    
\section{Separadores de unidad de sentido}\label{unitebiblio}
    
Partamos de un problema frecuente. Has observado que, por defecto, los distintos campos de una referencia bibliográfica\footnote{Con excepción de la fecha y de algunos otros campos.} se separan con puntos. Tenemos, por ejemplo:
    
    \bibverbose

    
    \renewcommand{\newunitpunct}[0]{\adddot\addspace} % pour l'exemple
    
    \begin{quotation}
    \cite{Urner1952}
    \end{quotation}
    
mientras que las reglas españolas exigirían tener:
    
\renewcommand{\newunitpunct}[0]{\addcomma\addspace} % on revient
    
    \begin{quotation}
    \cite{Urner1952}
    \end{quotation}
    
    \bibverbosetrad

Dentro de una referencia bibliog´rafica, \package{biblatex} distingue unidades de sentido\footnote{En realidad, hay unidades más amplias, llamadas bloques. Pero estas unidades sirven para configuraciones avanzadas de la presentación de la bibliografía final.}. Aquí: el autor, el título, el lugar de publicación, el editor y el año. 

    Estas unidades se separan con signos de puntuación. Los separadores los define el comando \csp{newunitpunct}.\label{newunitpunct}
    
   Así que basta con redefinir el comando incluyendo en el preámbulo\footnote{Esto no es estrictamente necesario, pero es un buen sistema incluir el conjunto de los comandos de personalización de estilo en el preámbulo. Recomendamos agruparlos en un solo fichero al que hay que llamar en el preámbulo.}: 
        
    \begin{latexcode}
\renewcommand{\newunitpunct}[0]{, }
    \end{latexcode}

\begin{plusloins}
        No hemos hablado hasta ahora del comando \csp{renewcommand}. Es idéntico al comando \cs{newcommand} con la diferencia de que permite redefinir un comando que ya existe.
        \end{plusloins}
    
    Esta solución parece funcionar. Pero plantea un problema: los estilos definen unidades de sentido sin plantearse la cuestión de saber si esas unidades están vacías o no.
    Tomemos la entrada siguiente:
    
        
    \inputminted{exemples/biblio/style1/afrique.bib}
    
    Se obtiene el siguiente resultado:
    \renewcommand{\newunitpunct}[0]{, }
    \begin{quotation}
        \cite{BrefHippone}
    \end{quotation}
    \renewcommand{\newunitpunct}[0]{\addcomma\addspace}
    
    La coma entre los dos puntos y \emph{Concilia Africae} se explica porque se ha creado una unidad que normalmente tiene que contener el valor del campo \champ{bookauthor}.
    
    Para evitar este problema, no hay que poner directamente en la definición del comando \cs{newunitpunct} los signos de puntuación, sino utilizar en su lugar los comandos de puntuación \footcite[Enumerados en][]{biblatex_ponctuation}. Estos comandos presentan de hecho la puntuación sólo si la unidad precedente tiene un contenido\footnote{Además, suprimen los múltiples espacios que la preceden.}.\label{commandeponctuation}
    Así, en nuestro caso, hay que poner:
    
    \begin{latexcode}
\renewcommand{\newunitpunct}[0]{\addcomma\addspace}
    \end{latexcode}
    

    
\subsection{Un separador específico para dos unidades}
    
   Puede que queramos que entre dos unidades determinadas, por ejemplo, entre el título y el subtítulo, el separador no sea el mismo que entre el resto de unidades. Sin duda has observado que es el caso que se da para el estilo \forme{verbose}: el lugar de publicación y el editor están separados por dos puntos.
    
    Se dan dos casos:
        \begin{enumerate}
            \item El package \package{biblatex} proporciona un comando específico para nuestro tipo de unidad. Es el caso para el separador que va ante el subtítulo\footcite[Estos comandos no son muy numerosos: aparecen en][]{biblatex_hooks}. En este caso, basta con redefinir el comando. Por ejemplo, queremos que aparezcan dos puntos entre el título y el subtítulo. Nos basta con indicar:
            
            \begin{latexcode}
\renewcommand{\subtitlepunct}[0]{\addspace\addcolon\addspace}
            \end{latexcode}

            
            \item El package no ofrece un comando específico para nuestro tipo de unidad. En ese caso, el problema es más complejo y exige nociones que aún no hemos estudiado. Por esto remitimos al capítulo siguiente\renvoi{unitepersonalisee}.
        \end{enumerate}
        
\section{Separador de referencias múltiples}\label{multicitedelim}

Cuando empleamos un comando de cita múltiple\renvoi{citemultiple}, los separadores por defecto son el punto y coma:


\begin{latexcode}
\autocites{Saxer1980}{Junod1992}
\end{latexcode}

\bibverbose
\begin{quotation}
\cites{Saxer1980}{Junod1992}
\end{quotation}
\bibverbosetrad

Aquí también se puede modificar el separador redefiniendo \csp{multicitedelim}. Por ejemplo, para tener una coma:

\begin{latexcode}
\renewcommand{\multicitedelim}[0]{\addcomma\addspace}
\end{latexcode}


\section{Estilos de algunos bloques}

La puntuación no es el único caso para el que \package{biblatex} ofrece herramientas para hacer cambios específicos. Algunas partes de las referencias bibliográficas también tienen la posibilidad de configurarse mediante comandos particulares\footcite{biblatex_hooks}. Tomemos, por ejemplo, la preposición de un autor.
    %%%%%%
    \bibverbose
    \begin{quotation}
    \cite{BeauvoirSexe}
    \end{quotation}
    
    \renewcommand{\mkbibnameprefix}[1]{\parentext{#1}}


A veces, se suele poner la partícula entre paréntesis. Para ello hay que redefinir el comando \csp{mkbibnameprefix}. Este comando recibe un argumento, que es la partícula. Para ponerla automáticamente entre paréntesis, basta con escribir lo siguiente:
    
    \begin{latexcode}
\renewcommand{\mkbibnameprefix}[1]{(#1)}
    \end{latexcode}

    
O, para ser aún más correcto:
    
    \begin{latexcode}
\renewcommand{\mkbibnameprefix}[1]{\parentext{#1}}
    \end{latexcode}


El comando \csp{parentext} pone entre paréntesis un texto que se le pasa como argumento. Y obtenemos el resultado deseado: 

    \begin{quotation}
    \cite{BeauvoirSexe}
    \end{quotation}
    
\bibverbose

\section{Cadenas de lengua}\label{i18nchaines}
    
    Las cadenas de lengua son esos fragmentos de texto fijos, pero que dependen de la lengua empleada. Tomemos una entrada de tipo \type{article}. Cuando la citamos, obtenemos:
    
     
     
    \begin{quotation}
        \cite{Junod1992}
    \end{quotation}
    
    La cadena \forme{en} separa el título del artículo y el nombre de la revista. Si escribiéramos en inglés, obtendríamos \forme{in} en su lugar.  
    
    
   Puede ocurrir que queramos cambiar estas cadenas de lenguas, por ejemplo, precisamente para que tengan la forma \forme{in}, término latino empleado a menudo en las bibliografías. Una cadena de lengua asocia a una clave invariable un valor que cambia según la lengua.

    
    Los estilos bibliográficos tienen cadenas de lengua por defecto que se pueden modificar caso por caso. Para ello, ante todo tenemos que conocer las claves, abriendo el fichero de lengua \fichier{spanish.lbx}. Para encontrar ese fichero, véase nuestra explicación en el anexo\renvoi{trouverfichier}.
    
    Una vez abierto este fichero, buscaremos la línea que comienza con:
    
\csp{DeclareBibliographyStrings}

    
    A esta línea la siguen otras que tienen la forma:
    
    \begin{latexcode}
clave    = {{cadena larga}{cadena corta}}
    \end{latexcode}

    
    \begin{plusloins}
    Cada cadena de lengua está disponible en versión larga o corta. Se puede pasar una opción si llamamos al package \package{biblatex} para decidir el uso de versiones largas.
    
    \begin{latexcode}
\usepackage[abbreviate=false]{biblatex}
    \end{latexcode}

    
    Sin embargo, cuando se redefine una cadena de lengua, solo se le asigna una versión.

    
    \end{plusloins}
    
    Tenemos, por tanto, que encontrar la clave correspondiente a nuestra cadena \enquote{en}. Enseguida descubriremos que se trata de la cadena \verb|in|.
    
   Una vez encontrada nuestra clave, es fácil asignarle un nuevo valor, gracias al comando \csp{DefineBibliographyStrings}\marg{lengua}\marg{cadena}, que hay que ubicar en el preámbulo del fichero.
    
    \begin{attention}
    Sobre todo, no hay que modificar los ficheros estándar: perderíamos las modificaciones con cada actualización
    \end{attention}
    
    \begin{latexcode}
\DefineBibliographyStrings{spanish}{%
    in = {\emph{in}}%
}
    \end{latexcode}

    Se pueden utilizar comandos \LaTeX{} en una cadena de lengua. Si nuestra cadena contiene signos de puntuación, hay que usar los comandos de puntuación mencionados más arriba. Los \% al final de cada línea están ahí para evitar espacios inconvenientes\renvoi{commandepourcent}.
    
    Si queremos redefinir varias cadenas de lengua, basta con separar cada definición con una coma:
    
    \begin{latexcode}
\DefineBibliographyStrings{french}{%
clave1 = {valor1},%
clave2 = {valor2},%
claveN = {valorN}%
}
    \end{latexcode}


\begin{plusloins}
Podríamos redefinir las cadenas de lengua consagradas en las abreviaturas académicas del tipo \emph{op. cit.} para tenerlas en cursiva. 

Hay una solución más sencilla\label{mkibid}. En efecto, \package{biblatex} pasa estas cadenas de lengua a un comando \csp{mkibidem}\marg{cadena} antes de mostrarlas.

Por tanto, nos basta con redefinir este comando:

\begin{latexcode}
\renewcommand{\mkibid}[1]{\emph{#1}}
\end{latexcode}

\end{plusloins}

\subsection{Introducción a las pruebas condicionales}\label{opcit}

Ahora vamos a anticiparnos un poco al capítulo siguiente.
Sabes que los estilos de la familia \forme{verbose-trad} emplean abreviaturas latinas como \emph{op. cit.}
Estas abreviaturas son las abreviaturas académicas habituales. No obstante,   \package{biblatex} emplea indistintamente \emph{op. cit.} para todos los tipos de entradas, incluidas las entradas de tipo \type{article}, mientras que se podría esperar una abreviatura como\emph{art. cit.}

Para resolver este problema, basta con redefinir la cadena de lengua que define \emph{op. cit.}, asignándole un valor diferente según el tipo de entrada.  Vamos a usar, por tanto, una de las numerosas pruebas condicionales disponibles en \package{biblatex}. 
Se trata del comando: \csp{iffieldequalstr}\marg{campo}\marg{texto}\marg{si-sí}\marg{si-no}.

Este comando solo se puede emplear en ciertas partes de un fichero \LaTeX: en los comandos de definición de estilos bibliográficos y en los de definición de una cadena de lengua. 
Verficia que el campo \arg{campo}  de la entrada bibliográfica citáda se corresponde con el argumento  \arg{texto}: si es así, devuelve el argumento \arg{si-sí}, y, en caso contrario, devuelve el argumento  \arg{si-no}.

Aquí probamos el pseudo-campo\footnote{\forme{Pseudo} pues no se sitúa realmente como un campo en la entrada bibliográfica.}\champ{entrytype}, que se corresponde al tipo de entrada sin la @. Comprobamos si se trata de una entrada de tipo artículo: si lo es, damos a la cadena de lengua el valor \forme{art. cit.},  y de lo contrario, le damos el valor \forme{op. cit.}.
Al final, esto da: 

\begin{latexcode}
\DefineBibliographyStrings{spanish}{%
    opcit = \iffieldequalstr{entrytype}{article}%
    {art\adddotspace cit\adddot}%
    {op\adddotspace cit\adddot}%
}
\end{latexcode}

Obsérvese el uso de los comandos de puntuación\renvoi{commandeponctuation}, como se ha señalado más arriba.



    
