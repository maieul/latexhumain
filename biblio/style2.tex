\chapter{Modificar los estilos bibliográficos (2)}\label{style2}

\renewbibmacro{cite:ibid}{\usebibmacro{cite:full}} % On modifie, pour ne pas avoir des ibidem à tout bout de champs

\begin{intro}
    Hemos visto que BibLaTeX ofrece una serie de comandos para personalizar rápidamente los estilos bibliográficos. No obstante, estos comandos no bastan para personalizaciones avanzadas.
    La posibilidad de elegir el orden en que se muestran los campos, por ejemplo, requiere ir más lejos en la comprensión de los estilos bibliográficos de \package{biblatex}.
\end{intro}



\section[¿Qué pasa cuando se usa \oldcs{\meta{prefijo}cite}?]{¿Qué pasa cuando se usa un comando de \cs{\meta{prefijo}cite}?}

Para entender cómo personalizar la manera en que se muestran las bibliografías, hay que tener un conocimiento básico de lo que pasa cuando se emplea un comando de cita.


Cuando se invoca un comando de cita del tipo \cs{\meta{prefix}cite}, este llamará:
    \begin{itemize}
        \item macros bibliográficas, encargadas de mostrar el argumento  \arg{prenote} o \arg{postnote}, o incluso de gestionar las citas repetidas. Las macros bibliográficas son tipos especiales de comandos, propios del  package \package{biblatex};\label{macrobiblio} 
        \item un driver\footnote{Aunque el término \enquote{driver} no es español, lo empleamos aquí porque se encuentra en los comandos internos de BibLaTeX.} bibliográfico. Un driver corresponde a un tipo de entrada (\type{article}, \type{book}, etc.), y se encarga de mostrar los campos de la entrada en el orden correcto. Para ello, llama:
        \begin{itemize}
            \item comandos de separación de unidades bibliográficas\renvoi{unitebiblio}, que hemos visto más arriba;
            \item macros bibliográficas. Estas macros bibliográficas llaman, a su vez:
            \begin{itemize}
                \item comandos de impresión de campos bibliográficos;
                \item eventualmente, a otras macros bibliográficas;
                \item cadenas de lenguas.
            \end{itemize}
        \end{itemize}
        
    \end{itemize}

Esto se puede resumir con el esquema~\ref{schemastylesbiblios} (p.~\pageref{schemastylesbiblios}). 

\begin{figure}[!]

\begin{tikzpicture}[
	level/.style={sibling distance=25em/#1,level distance=4cm},
	every node/.style={text width=3.5cm,align=center}
	]
	\node {Commande de citation} 
		child { node {Macros bibliographiques}}
		child { node {Driver bibliographiques}
			child { node {Macros bibliographiques}
			 	child {
					node{Impression de champs}
					}
			 	child {
					node{Appel à d'autre macros}
					}
				child {
					node{Appel à des chaînes de langues}
					}
				}
			child { node {Séparateur d'unité bibliographique}}
			}	
	;		
\end{tikzpicture}
\caption{Funcionamiento de los estilos bibliográficos}\label{schemastylesbiblios}
\end{figure}



\section[Redefinir una macro bibliográfica]{Redefinir una macro bibliográfica: ejemplo de los campos \champ{location} y \champs{publisher}}


El conjunto de estos elementos es completamente redefinible. Vamos a tomar un ejemplo concreto de la problemática existente.

Consideremos la entrada siguiente:

\inputminted{exemples/biblio/fichier/saxer.bib}

Aparece de la siguiente manera, con el editor comercial después del lugar de publicación:

\bibverbose
\begin{quotation}
\cite{Saxer1980}
\end{quotation}


Quizá deseemos tener el editor delante del lugar de publicación de la siguiente manera: 

\renewbibmacro*{publisher+location+date}{%
  \printlist{publisher}%
    \setunit*{\addcomma\space}%
  \printlist{location}%
  \setunit*{\addcomma\space}%
  \usebibmacro{date}%
  \newunit}

\begin{quotation}
\cite{Saxer1980}
\end{quotation}
\bibverbosetrad

Por tanto, lo primero que hay que hacer será identificar qué macro bibliográfica hay que modificar. Para ello, hay que encontrar los ficheros\renvoi{trouverfichier} de definición de estilos bibliográficos. Hay varios:

\begin{itemize}
\item un fichero \ext{def} que define los estilos invariables, independientemente del estilo de la bibliografía o de la cita elegido;
\item ficheros \ext{cbx} que definen los estilos utilizados cuando se emplean los comandos \cs{\meta{prefix}cite};
\item ficheros \ext{bbx} que definen los estilos utilizados cuando se llama al comando \cs{printbibliography}.
\end{itemize}

Algunos ficheros se llaman unos a otros: por ejemplo, los ficheros
 \ext{bbx} contienen los drivers bibliográficos. Por tanto, los llaman los ficheros \ext{cbx}. Estas llamadas mutuas entre ficheros permiten garantizar una uniformidad entre los estilos bibliográficos cuando se emplea  \cs{\meta{prefijo}cite} y cuando se emplea \cs{printbibliography}.

Supongamos que utilizas los estilos de la familia \forme{verbose}. Al abrir los ficheros estándar\renvoi{trouverfichier}, puedes remontar fácilmente al fichero \fichier{standard.bbx}, que contine los drivers bibliográficos de esta familia.
Puedes encontrar en las líneas siguientes\footnote{L.~58 el 9 de octubre de 2011.}:

\inputminted{exemples/biblio/styles2/driverbook.tex}

Se trata de un driver bibliográfico que explica cómo mostrar las entradas de tipo \type{book}. Llama macros bibliográficas por medio de los comandos \csp{usebibmacro}. Estas macros son comunes a varios drivers, lo que permite tener cierta uniformidad de estilo, para que, por ejemplo, los nombres de los autores se muestren sistemáticamente de la misma manera.

En el grupo de las macros invocadas, hay una que nos interesa en particular, la llamada a la macro \bibmacro{publisher+location+date} mediante:
 
\begin{latexcode*}{firstnumber=27}
\usebibmacro{publisher+location+date}
\end{latexcode*}

Si indagamos un poco en el mismo fichero, encontramos el lugar en el que se define la macro:

\inputminted{exemples/biblio/styles2/publisher+location+date.tex}

Comentaremos brevemente estas líneas, antes de explicar cómo hacer para invertir el orden de los dos campos. 

\begin{description}
\item[Línea 1] El comando \csp{newbibmacro*} indica que se declara una nueva macro bibliográfica, aquí \bibmacro{publisher+location+date}.  Para indicar que redefinimos una macro que ya existe, hay que utilizar en el preámbulo\footnote{O en cualquier parte del fichero \ext{tex} pero, en cualquier caso, no en los ficheros estándar.} el comando \csp{renewbibmacro*}.La definición de la macro se encuentra dentro de las llaves que aparecen a continuación.
\item[Línea 2] El comando \csp{printlist} indica que se muestra un campo que se podría presentar en forma de lista, es decir, donde la palabra clave \forme{and} tiene un sentido. Aquí se trata del campo \champ{location}.
\item[Línea 3] El comando \csp{iflistundef} prueba un campo que podría ser una lista, aquí el campo \champ{publisher}. Si este campo está vacío, ejecuta el contenido de la primera llave (línea 4), si no, el de la segunda (línea 5).%%%%%%AQUí
\item[Línea 4] Si el campo \champ{publisher} está vacío, se crea una nueva unidad bibliográfica\renvoi{unitebiblio}, por medio de  \csp{setunit*},\label{unitepersonalisee} separada de la anterior por una coma a la que se añade un espacio (\csp{addcomma}\csp{space}\renvoin{commandeponctuation}).
\item[Línea 5] Si el campo \champ{publisher} no está vacío, entonces se crea una nueva unidad bibliográfica, separada de la siguiente por dos puntos seguidos de un espacio (\csp{addcolon}\csp{space}).
\item[Línea 6] Se imprime el campo \champ{publisher}.
\item[Línea 7] Se crea una nueva unidad bibliográfica, separada de la anterior por una coma seguida de un espacio. Hay que tener en cuenta que si el campo \champ{publisher} está vacío, no hay más que una sola coma: remitimos a nuestra explicación anterior sobre los comandos de puntuación\renvoi{commandeponctuation}.
\item[Línea 8] Llamamos a una macro que se encarga de mostrar la fecha.
\item[Ligne 9] Se crea una nueva unidad bibliográfica. Le signo de separación se define con el comando \cs{newunitpunct}\renvoi{newunitpunct}, véase más arriba.
\end{description}

Para invertir el orden de nuestros campos, basta, por tanto, con redefinir la macro cambiando el orden de impresión de los campos. De paso, ya no queremos dos puntos como separadores, lo que nos permite suprimir una prueba condicional.


\begin{latexcode}
\renewbibmacro*{publisher+location+date}{%
  \printlist{publisher}%
    \setunit*{\addcomma\space}%
  \printlist{location}%
  \setunit*{\addcomma\space}%
  \usebibmacro{date}%
  \newunit}
\end{latexcode}


\newbibmacro*{publisher+location+date}{%
  \printlist{location}%
  \iflistundef{publisher}
    {\setunit*{\addcomma\space}}
    {\setunit*{\addcolon\space}}%
  \printlist{publisher}%
  \setunit*{\addcomma\space}%
  \usebibmacro{date}%
  \newunit}
    % Remettre les champs dans le bon ordre.


Presta mucha atención a los \% del final de las líneas: olvidarlos significa correr el riesgo de tener espacios no deseados en tus referencias bibliográficas.



\begin{plusloins}
	
Hay otros comandos además de \cs{printlist} para mostrar campos: \csp{printname} para imprimir un campo que contenga nombres de persona, y \csp{printfield} para imprimir un campo que no necesita un formato particular.

Para comprender mejor cuándo utilizar uno u otro de estos comandos, lo mejor es mirar los ficheros estándar.

\end{plusloins}
\begin{plusloins}
Si utilizas el campo \champ{address} en lugar del campo \champ{location}, ten en cuenta que el primero se considera como un alias del segundo: dicho de otro modo, utilizar \champ{address} equivale a utilizar \champ{location}.

Se pueden declarar nuevos alias del campo mediante el comando:

 \csp{DeclareFieldAlias}\marg{alias}\marg{original}.
\end{plusloins}

\section{Otros ejemplos: verdaderos \emph{op. cit.}}

Uno de los elementos problemáticos de los estilos bibliográficos estándar de la familia verbose es su forma de gestionar las abreviaturas académicas de tipo \emph{op. cit.} En efecto, los estilos indican los \emph{op. cit.} después de haber mostrado el autor y el título. Por ejemplo:

\begin{quotation}
\bibverbose	
\cite{Urner1952}

\cite{Saxer1980}
\bibverbosetrad

\cite{Urner1952}

\cite{Saxer1980}
\end{quotation}

Si no tenemos más que una entrada cuyo autor es Victor Saxer, resulta bastante inútil. Podríamos tener una versión abreviada con la forma:

\bibverbose

\begin{quotation}
\cite{Urner1952}

\cite{Saxer1980}

% Changeons %
\bibverbosetrad
\renewbibmacro*{cite:title}{%
  \printtext[bibhyperlink]{%
   \ifsingletitle{}{\printfield[citetitle]{labeltitle}}%
    \setunit{\nametitledelim}%
    \bibstring[\mkibid]{opcit}}}
% Changeons


\cite{Urner1952}

\cite{Saxer1980}
\end{quotation}

% Rétablissons
\bibverbosetrad
% Rétablissons

Para ello, vamos a modificar los estilos de \emph{biblatex}, empleando el comando:
\csp{ifsingletitle}\marg{si-sí}\marg{si-no}. 

Este comando verifica si una entrada es la única atribuida a su autor y devuelve \arg{si-sí} si es así, \arg{si-no} en el caso contrario. 

Para que funcione, hay que pasar la opción \option{singletitle=true} al cargar \package{biblatex}.

\begin{latexcode}
\usepackage[singletitle=true,...]{biblatex}
\end{latexcode}

Una vez hecho esto, hay que saber dónde aplicar este comando. Comencemos examinando el fichero \ext{cbx}, ya que se trata de un estilo para un comando \cs{\meta{prefijo}cite}. Bosquemos la expresión \forme{opcit} que corresponde a la cadena de lengua\renvoi{opcit} que devuelve \forme{op. cit.}.

La encontramos rápidamente en una macro que se llama  \bibmacro{cite:title}.

\begin{english}%en attendant que ce bug soit résolu
\inputminted{exemples/biblio/styles2/citetitle.tex}
\end{english}
Hagamos el análisis:

\begin{description}
\item[Línea 1] El nombre de la macro es \bibmacro{cite:title}.
\item[Línea 2] El comando \csp{printtext}  tiene dos usos:  poner directamente un texto asegurándose de que \emph{biblatex} gestiona la puntuación o bien reunir varios campos en un solo bloque tipográfico. Aquí estamos tratando del segundo uso: \contenuarg{bibhyperlink} significa que \package{biblatex} se va a encargar de poner un enlace de hipertesto dentro del documento PDF.
\item[Línea 3] El comando \cs{printfield} imprime un campo. Aquí el pseudo-campo \champ{labeltitle}: este devuelve el valor del campo \champ{shortitle} si está definidoi, si no, el de \champ{title}\renvoi{shortfields}. Lo muestra según el formato \contenuarg{citetitle}\footcite[Podríamos, si quisiéramos, definir otra forma de mostrar este campo con el comando \cs{DeclareFieldFormat}. Véase][]{biblatex_formating}.
\item[Línea 4] Nueva unidad bibliográfica, cuyo separador se define con el comando \csp{nametitledelim}.
\item[Línea 5] El comando \csp{bibstring} sirve para llamar una cadena de lengua, aquí \verb|opcit|. El primer argumento, cuyo valor es aquí \contenuarg{\cs{mkibid}}, indica que la cadena de lengua se pasa a  \csp{mkibid} antes de mostrarse. Este comando, que se puede redefinir, se encarga del formato: hemos hablado más arriba de la forma de utilizarlo para tener las abreviaturas latinas\renvoi{mkibid} en cursiva.
\end{description}

La línea que nos interesa es, por tanto, la línea 3, ya que queremos condicionar la visualización del campo título: si no hay más que una obra para el autor en cuestión, no se puede visualizar. Basta con volver a declarar la macro, incluyendo la prueba condicional:

\begin{english}%en attendant que ce bug soit résolu
\inputminted{exemples/biblio/styles2/singletitle.tex}
\end{english}

\begin{plusloins}
	Los editores y administradores de revistas pueden definir muy bien sus propios ficheros \ext{cbx} y \ext{bbx} para obtener un conjunto coherente de estilos. Esos ficheros, que contienen drivers y macros bibliográficas, deben empezar con ciertos comandos: remitimos a la documentación de \emph{biblatex}\footcite{biblatex_fichiers}.
\end{plusloins}

% On rétablis la gestion des ibidems

\renewbibmacro*{cite:ibid}{%
  \printtext{%
    \bibhyperlink{cite\csuse{cbx@lastcite@\thefield{entrykey}}}{%
      \bibstring[\mkibid]{ibidem}}}%
  \ifloccit
    {\global\toggletrue{cbx:loccit}}
    {}}


