\chapter{Entradas jerarquizadas}\label{sousentrees}

\begin{intro}
En este capítulo, vamos a ver cómo crear entradas jerarquizadas para organizar mejor tu base de datos y para gestionar las divisiones de las fuentes.
\end{intro}

\section{Principio de subentradas bibliográficas}

Supongamos que citas varias contribuciones de un mismo congreso. En principio, puedes crear varias entradas bibliográficas con la forma\footnote{Cualquier parecido con un congreso que haya tenido lugar es pura coincidencia.}:

\inputminted{exemples/biblio/sousentrees/reproduction.bib}

Así duplicas los datos sobre el congreso (título, editor, etc.). Si por casualidad te equivocas en una información, tendrás que modificarla en todas partes. Para evitar eso, Biber prevé la posibilidad de crear subentradas: entradas secundarias \enquote{herederas} de una entrada principal. La herencia se carga cuando se interpreta el fichero. Para hacerlo, basta con indicar en el campo \champ{crossref} de las entradas secundarias la clave de la entrada principal.


\inputminted{exemples/biblio/sousentrees/crossref.bib}

Tras la compilación, vemos que nuestras subentradas han heredado correctamente campos de la entrada principal:  


\begin{quotation}
\cite{clave1}

\cite{clave2}
\end{quotation}

Observarás que Biber procede a una herencia \enquote{inteligente}: el campo \champ{date} de la entrada madre se retoma como campo \champ{date} de la entrada hija, mientras que el campo \champ{title} se convierte en el campo \champ{booktitle}.

\begin{plusloins}
Se pueden crear tantos niveles de entrada bibliográfica como se desee. Sin embargo, hay que tener cuidado de no producir referencias circulares (A hereda de B, que hereda de C, que hereda de A). Además, algunos programas de gestión bibliográfica te impiden crear subsubentradas bibliográficas: tienes que modificar tus entradas con tu editor de texto.
\end{plusloins}

\section{Subentradas en la bibliografía final}

Cuando se cita una entrada hija, Biber agrega sistemáticamente la entrada madre en la bibliografía final. Sin embargo, se puede indicar un número mínimo de citas de entradas hijas para que se haga el añadido de la entrada madre en la bibliografía. Si deseamos que una entrada madre  no se añada a la bibliografía más que si las entradas hijas se citan por lo menos 10 veces (todas las entradas confundidas):


\begin{latexcode}
\usepackage[mincrossrefs=10,…]{biblatex}
\end{latexcode}





\section{Especificar las herencias de los campos}

Biber tiene cierto número de reglas de herencia de campos\footcite{biblatex_crossrefsetup}. Pero se pueden modificar estas herencias o añadirles otras. Los campos \champ{entrysubtype}, por ejemplo, no se heredan por defecto. 
Para modificar las reglas estándar, basta con utilizar en el preámbulo:
\csp{DeclareDataInheritance}\marg{entrada madre}\marg{entrada hija}\marg{reglas}.

\begin{description}
\item[\arg{entrada madre}] designa el tipo de entrada madre: \type{book}, \type{proceedings}, etc. El símbolo * designa cualquier tipo de entrada.
\item[\arg{entrada hija}] designa el tipo de entrada hija. También aquí el símbolo * designa cualquier tipo de entrada.
\item[\arg{reglas}] designa cierto número de reglas de herencia en forma de comandos.
\end{description}

Los tipos de entradas que hay que indicar en los argumentos \arg{entrada hija} y \arg{entrada madre} no tienen que ir precedidos del signo \verb|@|.

El primer comando de herencia es el siguiente \csp{noinherit}\marg{campo}.
Impide que una entrada hija herede un campo.

El segundo es
\csp{inherit}\oarg{opcion}\marg{campo fuente}\marg{campo de destino}.
No hay más que una opción posible: \verb|override=true|. Si se pasa esta opción, el valor del campo de la entrada hija lo aplasta el del campo de la entrada madre. En caso contrario, prevalece el valor del campo de la entrada hija.

Así, para permitir la herencia del campo \champ{entrysubtype} basta con escribir:

\begin{latexcode}
\DeclareDataInheritance{*}{*}{
    \inherit{entrysubtype}{entrysubtype}
}
\end{latexcode}

