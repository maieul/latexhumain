\chapter{Entrées hiérarchisées}\label{sousentrees}

\begin{intro}
Dans ce chapitre, nous allons voir comment créer des entrées hiérarchisées, pour mieux organiser sa base de données  et pour gérer les divisions de source.
\end{intro}

\section{Principe des sous-entrées bibliographiques}

Supposons que vous citiez diverses contributions d'un même colloque. Classiquement vous créez plusieurs entrées bibliographiques sous la forme\footnote{Toute ressemblance avec un colloque s'étant déjà tenu ne saurait être que pure coïncidence.}:

\inputminted{exemples/biblio/sousentrees/reproduction.bib}

Vous dupliquez ainsi les données sur le colloque (titre, éditeur, etc.). Si par hasard vous vous êtes trompés dans une information, vous devrez la modifier partout. Pour éviter cela Biber prévoit la possibilité de créer des filiations d'entrées : des entrées secondaires \enquote{héritant} d'une entrée principale. L'héritage étant pris en charge lors de l'interprétation du fichier. Pour ce faire, il suffit d'indiquer dans le champ \champ{crossref} des entrées secondaires la clef de l'entrée principale.


\inputminted{exemples/biblio/sousentrees/crossref.bib}

Après compilation, nous voyons que nos sous-entrées ont correctement hérité des champs de l'entrée principale :  


\begin{quotation}
\cite{clef1}

\cite{clef2}
\end{quotation}

Vous constaterez que Biber procède à un héritage \enquote{intelligent} : le champ \champ{date} de l'entrée mère est repris  comme champ \champ{date} de l'entrée fille, cependant que le champ \champ{title} devient le champ \champ{booktitle}.

\begin{plusloins}
Il est possible de créer autant de niveaux d'entrée bibliographique que souhaités. Attention cependant à ne pas produire des références circulaires (A héritant de B, héritant de C, héritant de A). Par ailleurs, certains logiciels de gestion bibliographique vous interdisent de créer des sous-sous-entrées bibliographiques : il vous faut alors modifier vos entrées avec votre éditeur de texte.
\end{plusloins}

\section{Sous-entrées dans la bibliographie finale}


Lorsqu'une entrée fille est citée, Biber rajoute systématiquement l'entrée mère dans la bibliographie finale. On peut toutefois indiquer un nombre minimal de citations d'entrées filles pour que se fasse l'ajout de l'entrée mère dans la bibliographie. Si nous souhaitons qu'une entrée mère ne soit ajoutée à la bibliographie que si les entrées filles sont citées au moins 10 fois (toutes entrées confondues) :


\begin{latexcode}
\usepackage[mincrossrefs=10,…]{biblatex}
\end{latexcode}





\section{Préciser les héritages de champs}

Biber possède un certain nombre de réglages d'héritage de champs\footcite{biblatex_crossrefsetup}. On peut cependant modifier ces héritages, ou en ajouter d'autres. Les champs \champ{entrysubtype}, par exemple, ne sont par défaut pas hérités. 
Pour modifier les réglages standards, il suffit d'utiliser, dans le préambule  :
\csp{DeclareDataInheritance}\marg{entrée mère}\marg{entrée fille}\marg{règles}.

\begin{description}
\item[\arg{entrée mère}] désigne le type d'entrée mère : \type{book}, \type{proceedings}, etc. Le symbole * désigne n'importe quel type d'entrée.
\item[\arg{entrée fille}] désigne le type d'entrée fille. Là aussi le symbole * désigne n'importe quel type d'entrée.
\item[\arg{règles}] désigne un certain nombre de règles d'héritages, sous forme de commandes.
\end{description}

Les types d'entrées à préciser dans les arguments \arg{entrée fille} et \arg{entrée mère} ne doivent pas être précédés du signe \verb|@|.

La première commande d'héritage est la suivante \csp{noinherit}\marg{champ}.
Elle empêche l'héritage d'un champ par une entrée fille.

La seconde  est 
\csp{inherit}\oarg{option}\marg{champ source}\marg{champ cible}.
Il n'y qu'une seul option possible : \verb|override=true|. Si cette option est passée, alors la valeur du champ de l'entrée fille est écrasée par celle du champ de l'entrée mère. Sinon, la valeur  du champ de l'entrée fille a priorité.

Ainsi, pour permettre l'héritage du  champ \champ{entrysubtype} il suffit d'écrire :

\begin{latexcode}
\DeclareDataInheritance{*}{*}{
    \inherit{entrysubtype}{entrysubtype}
}
\end{latexcode}

