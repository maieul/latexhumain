\chapter{Indicar referencias bibliográficas}

\bibverbose

\begin{intro}
Hemos visto antes las distintas formas de citar un
texto\renvoi{citertexte} en el cuerpo del trabajo. También hemos visto
cómo rellenar una base de datos bibliográfica\renvoi{bddbiblio}.

Sólo nos queda relacionar los textos citados con la base de datos
constituida para tener un trabajo correcto, indicando las referencias
de las citas. Este es el objetivo de este capítulo.

\end{intro}


\section[Llamar al  package]{Llamar al package \package{biblatex}}

La gestión bibliográfica se efectúa con el  package
\package{biblatex}. En el preámbulo, se llama al package en su forma
más sencilla por medio del comando:
\cs{usepackage}\verb|{biblatex}|.


Sin embargo, el  package tiene numerosas
opciones\footcite{biblatex_options}. La que nos interesa de momento es
\option{citestyle}, que permite definir la manera en que aparecerán
las referencias bibliográficas,  especialmente, cuando se citan de
forma repetida.

Hay un número importante de estilos de cita que vienen de
serie. Mencionamos aquí los más importantes\footcite[Para más
detalles, véase][]{biblatex_style}:
\begin{choix}
\item[numeric]a cada entrada se le atribuye un número que se llama
  cuando se remite a esta entrada. La bibliografía final indica las correspondencias.
\item[authortitle]se indica solamente el autor y el título de la obra.
\item[verbose]la primera vez se da una descripción completa de la
  entrada y, luego, se presenta una versión abreviada.
\item[verbose-ibid]se da la descripción completa de la entrada pero,
  si se cita varias veces seguidas un pasaje, se emplea la abreviatura \emph{ibidem}.
\item[verbose-note]se da en la primera mención la descripción completa
  de la entrada y luego se utiliza una versión abreviada.
\item[verbose-trad1; verbose-trad2; verbose-trad3]~se utiliza la
  descripción completa de la entrada y después se emplean las abreviaturas
  académicas de tipo \emph{op. cit.}, \emph{ibidem.} etc.;
  \package{biblatex} calcula automáticamente la oportunidad de emplear
  una abreviatura académica, en función del entorno en que se haya
  citado previamente la obra. Véase el manual para la descripción
  completa de las diferencias entre estos tres estilos.
\end{choix}

Se comprende aquí uno de los grandes intereses de \LaTeX{}: ya no hay
que perder tiempo preguntándose: \enquote{¿Tengo que usar aquí una
  versión abreviada de la referencia?}, \package{biblatex} lo hace por
ti.

Así que vamos a ver cómo hay que hacer la llamada al package si
elegimos el estilo de cita \verb|verbose-trad-2|:

\begin{latexcode}
\usepackage[citestyle=verbose-trad2]{biblatex}
\end{latexcode}

Para las humanidades recomendamos los estilos de la familia verbose.


\section{Llamar a la base de datos bibliográfica}


Para que \LaTeX{} sepa dónde buscar las referencias bibliográficas,
basta con indicarle cuál es el fichero \ext{bib} que ha de utilizar,
para lo que basta con emplear en el preámbulo el comando:
\csp{addbibresource}\marg{ruta al fichero.bib}.

La ruta al ficher se indica de la misma manera que cualquier ruta a
cualquier fichero\renvoi{chemin}.

\begin{attention}
Se pueden llamar varios ficheros bibliográficos. No lo recomendamos,
en la medida en que esto nos obliga a comprobar que no hay varias
entradas con la misma clave.
\end{attention}

\begin{plusloins}
Las versiones anteriores de \package{biblatex} empleaban el comando
\csp{bibliography}. Este comando ya resulta obsoleto. El comando
\csp{addbibresource} tiene la ventaja de que permite usar ficheros
remotos o en formatos distintos de \ext{bib}, como, por ejemplo,
exportaciones de Zotero\footcite{biblatex_resources}. 
\end{plusloins}

\section{Citar un elemento bibliográfico}

El conjunto de comandos de cita tienen la sintaxis: 

\csp{\meta{prefijo}cite}\oarg{prenota}\oarg{postnota}\marg{clave},
\meta{prefijo} indica dónde y cómo debe aparecer la referencia:
directamente en el texto, entre paréntesis, en nota a pie de página, etc.

Por ejemplo, queremos citar un libro de Victor Saxer que hemos
incluido de esta forma en la base de datos:


\inputminted{exemples/biblio/fichier/saxer.bib}

Escribimos \csp{cite}\verb|{Saxer1980}|. Tras la tercera
compilación\renvoi{3compil}, la referencia aparece según el estilo
elegido al llamar a  \package{biblatex}. Así, para un estilo de la familia verbose:

\begin{quotation}
\fullcite{Saxer1980}
\end{quotation}

Lo común en humanidades es citar en nota a pie de página. Por eso vale
más utilizar: \csp{footcite}\verb|{Saxer1980}|,\label{footcite} que
pone la referencia en nota a pie de página, añadiendo automáticamente
el punto final en la nota. También se puede emplear el comando
\csp{parencite}, para citar entre paréntesis. 

Pero también se puede elegir el comando \csp{autocite}. El modo de
citar (nota a pie de página, cita directa, cita entre paréntesis,
etc.) depente entonces del estilo de cita elegido: para los estilos de
la familia verbose, la referencia se pone en nota a pie de página. 

\begin{plusloins}
Por defecto, los campos bibliográficos se separan con puntos. Si te
ves obligado a poner comas en vez de puntos, remítete al capítulo~\ref{style1}.
\end{plusloins}

\subsection{Los argumentos \arg{prenota} y \arg{postnota}}

Supongamos que queremos presentar un texto delante de nuestra
referencia. Por ejemplo: \enquote{Véase también}. Se utiliza el
argumento optativo \arg{prenota}.

\begin{latexcode}
Blabla \autocite[Véase también][]{Saxer1980} blablabla.
\end{latexcode}

\begin{quotation}
Blabla \cite[Véase también][]{Saxer1980} blablabla.
\end{quotation}



También puede que queeremos poner algo después de la referencia. En
ese caso, se utiliza el argumento \arg{postnota}.

\begin{latexcode}
Blabla \autocite[Véase también][que trata sobre un tema parecido.]{Saxer1980} blabla
\end{latexcode}

\begin{quotation}
Blabla \cite[Véase también][que trata sobre un tema parecido.]{Saxer1980} blabla
\end{quotation}

\begin{attention}
  Si se uta el argumento \arg{prenota}, es obligatorio indicar un
  argumento  \arg{postnota}, aunque esté vacío. Si no aparece este
  argumento,  \package{biblatex} interpreta lo que pensabas que era
  \arg{prenota} como \arg{postnota}.
\end{attention}

\subsection{El argumento \arg{postnote} y los números de página}\label{pagespostnote}

El argumento \arg{postnote} puede servir para indicar las páginas
exactas del pasaje citado\footcite[Para más detalles, consúltese:
][]{biblatex_pages}. Para ello, basta con que contenga un valor de
tipo numérico, en cifras árabes o romanas, o bien una secuencia de
valores numéricos.

Por ejemplo, para citar la página 25: 

\begin{latexcode}
\autocite[25]{Saxer1980}
\end{latexcode}

 Para citar las páginas 25 a 35:

\begin{latexcode}
\autocite[25-35]{Saxer1980}
\end{latexcode}

O incluso las páginas 25 y 35:

\begin{latexcode}
\autocite[25 \& 35]{Saxer1980}
\end{latexcode}

Para citar no sólo la página exacta, sino también añadir otra cosa en
el argumento \arg{postnota}, empleamos el comando  \csp{pno},
para que \package{biblatex} inserte por sí mismo la \forme{p.} :

\begin{latexcode}
\autocite[\pno~22,  con el que estamos en desacuerdo.]{Saxer1980}
\end{latexcode}

\begin{quotation}
\cite[\pno~22, con el que estamos en desacuerdo.]{Saxer1980}
\end{quotation}

Tamvbién se pueden emplear los comandos \csp{nopp} para que no
aparezca prefijo de página, \csp{psq} para indicar que también hay que
incorporar la página siguiente, y \csp{psqq} para indicar que hay que
incorporar las páginas siguientes.

\subsubsection{Problema con el campo \champ{pages}}


Se plantea un problema cuando un campo \champ{pages} ya está cubierto,
para un artículo, por ejemplo. Nos encontramos, efectivamente, con dos
numeraciones de páginas: la del campo \champ{pages} y la del pasaje
exacto que se cita, indicado por la opción
\arg{postnota}. Tomemos la entrada siguiente:

\inputminted{exemples/biblio/fichier/junod.bib}

Y llamémosla con el código siguiente:

\begin{latexcode}
\cite[24]{Junod1992}
\end{latexcode}

Se obtiene este resultado:

\begin{quotation}
\fullcite[24]{Junod1992}
\end{quotation}



Para evitar este inconveniente, si utilizamos un estilo de tipo
verbose, hay que pasar una opción cuando llamamos al package
\option{citepages=omit}

\begin{latexcode}
\usepackage[citestyle=verbose,citepages=omit]{biblatex}
\end{latexcode}

Al escribir entonces \cs{cite}\verb|[24]{Junod1992}|
aparece correctamente:

\begin{quotation}
\cite[24]{Junod1992}
\end{quotation}

Por el contrario, el problema persiste si queremos incluir texto en el
argumento \arg{postnota} detrás del número de página.

\begin{latexcode}
\cite[\pno~24, pasaje muy interesante.]{Junod1992}
\end{latexcode}

\begin{quotation}
\cite[\pno~24, pasaje muy interesante.]{Junod1992}
\end{quotation}

Como puedes ver, el campo \champ{pages} sigue apareciendo. Se trata de
un límite de \emph{biblatex}. Para soslayar este problema, es
necesario cargar el package
\package{biblatex-true-citepages-omit}\footcite{biblatex-true-citepages-omit}.






\begin{plusloins}
  Para las obras en varios volúmenes, se pueden emplear los comandos
  \csp{volcite}, \csp{pvolcite}, \csp{fvolcite}, \csp{avolcite}
  que se corresponden con los comandos respectivos \cs{cite}, \cs{parencite},
  \cs{footcite}, \cs{autocite}.

La sintaxis es: 
\csp{\meta{prefijo}cita}\oarg{prenota}\marg{volumen}\oarg{página}\marg{clave}.
\end{plusloins}

\subsubsection{Tipo de paginación}

Podemos precisar en la base de datos el tipo de paginación de una
entrada: ¿está paginada en páginas, en columnas, etc.? Para ello, se
utiliza el campo \champ{pagination}, dándole uno de los valores siguientes :

\begin{description}
\item[page] la paginación está en forma de páginas. Es el valor estándar.
\item[column] la paginación está en forma de columnas.
\item[line] la paginación está en forma de líneas.
\item[verse] la paginación está en forma de versículos / versos.
\item[section] la paginación está en forma de apartados.
\item[paragraph] la paginación está en forma de párrafos.
\item[none] la paginación es libre.
\end{description}

El campo \champ{pagination} sirve cuando se pone una indicación
numérica en el argumento \arg{postnota}. Por el contrario, si no se
usa argumento \arg{postnota} y se deja el contenido del campo
\champ{pages}, queda sin efecto. Tomemos la entrada siguiente:

\inputminted{exemples/biblio/citation/desnos.bib}

Llamémosla utilizando el argumento \arg{postnota}, y luego sin utilizarlo.


\begin{latexcode}
\autocite[2]{desnos}

\autocite{desnos}
\end{latexcode}

% Se modifica para tener la elección estándar:
\DeclareFieldFormat{pages}{\mkpageprefix[bookpagination]{#1}}

\begin{quotation}
\cite[2]{desnos}

\cite{desnos}
\end{quotation}

% On rétablis
\DeclareFieldFormat{pages}{\mkpageprefix[pagination]{#1}}

Observamos que, en el segundo caso, \champ{pagination} no se tiene en
cuenta.  En ese caso, hay que poner el tipo de paginación en el campo \champ{bookpagination} :

\begin{latexcode}
@book{desnos,
    Author = {Robert Desnos},
    Pagination = {verse},
    Bookpagination = {verse},
    Pages = {1-3},
    Title = {La hormiga}}
\end{latexcode}

\begin{quotation}
\cite[2]{desnos}

\cite{desnos}
\end{quotation}


\begin{plusloins}
  Podemos evitar duplicar la información en el campo
  \champ{bookpagination} incorporando la línea siguiente al comienzo
  del fichero \ext{tex} :

\begin{minted}{latex}
\DeclareFieldFormat{pages}{\mkpageprefix[pagination]{#1}}
\end{minted}

El comando \csp{DeclareFieldFormat} indica la forma de presentar el
campo \champ{pages}: pasando su contenido (\verb|#1|) a una función
\csp{mkpageprefix}, a la que se indica que se base en el campo
\champ{pagination} para presentar el prefijo de página\footcite[Véase][]{biblatex_formating}.
\end{plusloins}

\begin{plusloins}
  Puedes definir tus propios tipos de paginación declarando nuevas
  cadenas de lenguas\renvoi{i18nchaines}. Explicamos el método en
  nuestro blog\footcite{biblio_pagination}.
\end{plusloins}


\subsubsection{División de fuente}\label{divisionsource}

Los textos antiguos tienen generalmente una división propia que
permite ---en teoría--- localizar un pasaje en cualquier edición. Así:

\begin{quotation}
\cite{DoctrineChretienneDivision}
\end{quotation}

significa que remitimos al \emph{De Doctrina Christiana}
de Agustín, libro II, capítulo \textsc{viii}, párrafos 12-13. En
teoría, se puede localizar el pasaje en cuestión en cualquier edición
del tratado. Esto se suele llamar \enquote{división de fuente}.


Por el contrario, si imaginamos que remitimos a una edición particular,
por ejemplo, la de la Biblioteca Augustiniana, nos gustaría, en ese
caso, indicar las páginas concretas.

\begin{quotation}
\cite[(II, \textsc{viii}, 12-13)150-155]{DeDoctChr}
\end{quotation}

Por desgracia, \package{biblatex} no permite modificar el número de
páginas ni ofrece la forma de indicar una división de fuente. Sin
embargo, se puede emplear el package
\package{biblatex-source-division}\footcite{biblatex-source-division}
para resolver este problema. Un texto escrito entre paréntesis al
comienzo del argumento \arg{prenote} se considerará como una división
de fuente y, en consecuencia, se indicará justo detrás del título.

\begin{latexcode}
\cite[(II, \textsc{viii}, 12-13)150-155]{DeDoctChr}
\end{latexcode}


\begin{plusloins}
  Desde el punto de vista formal, la división de fuente se transfiere
  a un campo \champ{titleaddon}.

  La primera edición es esta obra ofrecía otra solución a partir de
  subentradas bibliográficas\footcite[112-113]{Rouquette2012}. Esta
  solución era poco práctica. Por ese motivo hemos creado el package
  \package{biblatex-source-division}.
\end{plusloins}



\section{Citar varias obras}\label{citemultiple}

Se pueden citar varias obras de un tratado, mediante un comando cuya
sintaxis básica es:

\csp{\meta{prefijo}citas}\oarg{prenota}\oarg{postnota}\marg{clave}\oarg{prenota}\oarg{postnota}\marg{clave}…

Se pueden citar seguidas tantas obras como se desee. Cada elemento
citado tiene derecho a su argumento \arg{prenota} y \arg{postnota},
que se usa de la misma manera que para las citas simples. Veamos un
ejemplo de su uso:

\begin{latexcode}
\autocites{Saxer1980}{Junod1992}
\end{latexcode}

\begin{quotation}
\cites{Saxer1980}{Junod1992}
\end{quotation}

Se puede especificar un texto para que aparezca delante de la lista de
las referencias, así como un texto para que aparezca detrás.

\begin{latexcode}
\<prefijo>citas(Texto delante)(Texto detrás){Saxer1980}{Junod1992}
\end{latexcode}

Los separadores de citas son por defecto punto y coma. Se pueden
modificar de forma global, hablamos de ello más adelante\renvoi{multicitedelim}.

Sin embargo, si el argumento \arg{postnota} de un elemento del comando
de cita múltiple acaba con un signo de puntuación, en ese caso, el
punto y coma no aparece entre este elemento y el siguiente:

\begin{latexcode}
\autocites[consúltese también:]{Saxer1980}{Urner1952}
\end{latexcode}

Da:

\begin{quotation}
\cites[consúltese también:]{Saxer1980}{Urner1952}
\end{quotation}


\section{Elegir la forma abreviada}\label{shortfields}

Con el estilo  verbose, podemos hacer que aparezcan las referencias de
forma abreviada. Para ello hay varios campos que aún no hemos mencionado.

\begin{choix}
    \item[shortauthor] Nombre abreviado del autor.
    \item[shorteditor] Nombre abreviado del editor.
    \item[shorthand] Forma abreviada de la referencia.
    \item[shorthandintro] Cuando se cita por vez primera una entrada y
      si se usa el campo \champ{shorthand}, el campo
      \champ{shorthandintro} sirve para introducir la forma abreviada.
      Por ejemplo, \enquote{citado más adelante:}.
    \item[shorttitle] Forma abreviada del título.
\end{choix}

El comando \csp{printshorthands} permite imprimir la lista de abreviaturas.

\bibverbosetrad
