\newcommand{\vide}[0]{}

\newcommand{\classe}[1]{%
	%\index{#1}%
	\textbf{#1}%
}
\newcommand{\commande}[1]{%
	%\index{#1}%
	\textbf{#1}%
}

\newcommand{\concept}[2][]{%
	\ifthenelse{\equal{#1}{\vide}}%
	{%\index{#1}%
	\emph{#2}%
	}%
	{%
	%\index{#1}%
	\emph{#2}%
	}%
}

\newcommand{\enviro}[1]{%
	%\index{#1}%
	\emph{#1}%
}

\newcommand{\logiciel}[1]{%
	%\index{#1}%
	{\emph{#1}}%
}

\newcommand{\notion}[2][]{%Synonyme de concept
	\concept[#1]{#2}%
}

\newcommand{\revision}[1]{%
	\marginpar{Rev : #1}%
}
\newcommand{\package}[1]{%
	\emph{#1}%
}
\newcommand{\renvoi}[1]{\marginpar{Renvoi : #1}}


% Redefinir les commandes de minted
\let\oldinputminted\inputminted
\renewcommand{\inputminted}[1]{%
\oldinputminted[linenos]{latex}{#1}%
}


\let\oldmint\mint
\renewcommand{\mint}[1]{%
\oldmint{latex}{#1}%
}
%pour ecrire latex
\let\latex\LaTeX
\let\LaTex\LaTeX