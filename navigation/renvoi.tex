\chapter{Renvois internes}\label{label}

\begin{prealable}
Dans ce court chapitre nous allons nous intéresser à la manière d'opérer des renvois à l'intérieur d'un document.
Il s'agit de permettre d'afficher des textes du type : \enquote{Nous renvoyons au chapitre N. page P.}
\end{prealable}

\section{Étiqueter des  emplacements : la commande \commande{label}}

Pour permettre de faire des renvois internes, il est nécessaire de placer des \enquote{étiquettes} aux endroits vers lesquels on souhaite renvoyer.
Cet étiquetage  se fait avec la commande \commande{label{etiquette}}.



%\section{Se servir des étiquettes}

Après avoir avoir placé des étiquettes, on peut y renvoyer. 
Il suffit de placer des commandes de renvois aux endroits souhaités. 
Toutefois il est nécessaire de compiler deux fois avec \XeLaTeX{}.
À la première compilation,   \XeLaTeX{}%
note les emplacements des étiquettes dans un fichier. À la seconde compilation, il lit ce fichier afin de procéder aux renvois. 

\begin{attention}
	Si vous modifiez votre texte, il faudra de nouveau compiler deux fois. En effet, les numéros de pages, de titres, de légendes etc. peuvent avoir changé. Il faut donc  que \XeLaTeX{} les apprenne à nouveau. En résumé, pour avoir un document correct il faut :
	\begin{enumerate}
		\item Compiler avec \XeLaTeX{};
		\item Compiler avec BibTeX afin d'interpréter le fichier \ext{bib};
		\item Compiler avec \XeLaTeX{};
		\item Compiler avec \XeLaTeX{} (car l'ajout des références bibliographiques a pu modifié le positionnement des étiquettes).
	\end{enumerate}
	
	Tout ceci peut paraître bien compliqué et risque d'entraîner des oublis. C'est pourquoi les journaux de compilations, ces lignes de textes qui apparaissent lors de la compilation avec \XeLaTeX{}, disent dans les dernières lignes s'il est nécessaire de compiler, et avec quel logiciel.
\end{attention}

\subsection{Renvoyer à une page}

Pour renvoyer au numéro de page correspondant à l'étiquette \forme{etiquette} il suffit d'utiliser la commande \commande{pageref}.

\begin{minted}{latex}
blabla \label{etiquette}
…

Nous renvoyons à la page  \pageref{etiquette}.
\end{minted}

\subsection{Renvoyer à un numéro de section}

Pour renvoyer à un numéro de section, il suffit d'utiliser la commande \commande{ref}.

\begin{minted}{latex}
\section{Section} \label{etiquette}
…
Nous renvoyons à la section  \ref{etiquette}.
\end{minted}
 
\subsection{Renvoyer à un titre de section}

Pour renvoyer à un titre de section, il suffit d'utiliser la commande \commande{nameref}.

\begin{minted}{latex}
\section{Section} \label{etiquette}
…
Nous renvoyons à la section  \enquote{\nameref{etiquette}}.
\end{minted}

Toutefois cette commande n'est pas disponible en standard : elle est proposée par le package \package{hyperref}. Il faut donc charger ce package dans le préambule. Nous documentons plus loin quelques fonctionnalité de ce package\renvoi{hyperref}.

\begin{anedocte}
 Grâce au package \package{hyperref} 
\end{anedocte}
\section{Où placer la commande \commande{label} ?}

Pour le moment, nous avons vu des renvois vers des sections, mais le système de renvois est beaucoup plus souple.

Une étiquette permet de renvoyer à tout élément numéroté, comme un titre, une note de base de page, une légende de flottant. Elle peut aussi renvoyer à un endroit précis, uniquement pour la page.

\begin{itemize}
\item Si l'étiquette \commande{label} est placée \emph{dans}  une commande  d'élément numéroté, elle renvoie à cette élément. Par exemple, pour renvoyer à une figure \renvoi{legende} :
\begin{minted}{latex}
\begin{figure}[paramètre de placement]
	Insertion de la figure
	\caption{Légende\label{figure}}
\end{figure} 
…

Nous renvoyons à la figure \ref{figure} situé page \pageref{figure}.
\end{minted}
\item Si la commande est placée ailleurs, elle renvoie à la page courante \emph{et à la section courante}\footnote{En réalité il est possible de placer \emph{immédiatement} après un élément numéroté pour y renvoyer, mais cela ne s'applique pas aux notes de bas de pages.}.
\end{itemize}

\section{Comment nommer ses étiquettes ?}

Vous êtes bien sûr libre de trouver votre propre système de nommage d'étiquette. Toutefois il est d'avoir quelque chose de la forme : \forme{type:nom}, où \forme{forme} désigne le type d'élément vers lequel on renvoie.

Exemples :

\begin{minted}{latex}
\footnote{Blabla \label{note:nom}}
\section{Titre \label{section:nom}}
\end{minted}


