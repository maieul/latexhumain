\thispagestyle{empty}

Durante mucho tiempo \LaTeX sólo se ha utilizado en el dominio de las ciencias llamadas \enquote{exactas}. Sin embargo, desde hace un tiempo las humanidades pueden utilizar con eficacia esta extraordinaria herramienta de composición de textos.


Lamentablemente, la mayoría de las introducciones a \LaTeX no suelen abordar las herramientas útiles para las humanidades. Así que este libro es el primer manual en español de introducción al uso de \LaTeX en humanidades.

Obtener una tipografía de alta calidad, gestionar biografías copiosas, ofrecer ediciones críticas y traducciones paralelas: éstas son algunas de las muchas razones que  tendrían que impulsar a estos especialistas de los textos que son los estudiantes e investigadores de humanidades a pasarse a \LaTeX.

Tales son las razones que nos han impulsado a escribir este libro que acompañará ---así lo esperamos--- a los especialistas en humanidades, tras el descubrimiento de \LaTeX, en su variante  \XeLaTeX,  para lograr la personalización de la apariencia de los textos, pasando por la gestión de una bibliografía con muchos títulos, sin olvidar las ediciones críticas de textos y todos esos detalles que constituyen las dificultades y el encanto de la escritura en humanidades \ldots

Otras razones puede que impulsen al lector a mejorarlo un día, pues, al compartir el saber no se divide: se multiplica.

\vspace{4ex}

\scriptsize
\emph{
Maïeul Rouquette es asistente diplomando en historia del cristianismo antiguo y literatura apócrifa cristiana en la Facultad de Teología y Ciencias de las Religiones en la Universidad de Lausanne.} 
\emph{Mantiene los packages \emph{reledmac}, \emph{bibleref-french} y hace varias contribuciones a \emph{biblatex}.}


\emph{Brendan Chabannes, professeur de lettres modernes, et Enimie Rouquette, doctorante en latin médiéval, l'ont accompagné.}

\normalsize

\vspace{4ex}
Ce livre est diffusé sous Licence \enquote{Creative Commons - Paternité - Partage des conditions initiales à l'identique}

\vspace{2ex}
 \raggedleft\includegraphics[height=3ex]{images/cc.png}
