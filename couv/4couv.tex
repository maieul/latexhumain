\thispagestyle{empty}

Pendant longtemps \LaTeX n'a été utilisé que dans le domaine des sciences dites \enquote{exactes}.
Pourtant il est possible depuis peu d'utiliser efficacement en sciences humaines ce formidable outils de composition de textes.

Malheureusement, la plupart des introductions à \LaTeX présentent mal les outils utiles aux humanités. Ce livre est donc le premier manuel francophone d'introduction à l'usage de \LaTeX en sciences humaines.

Obtenir une typographie de haute qualité, gérer une bibliographie prolifique, proposer des éditions critiques de textes et des traductions en parallèle : telles sont les nombreuses raisons qui devraient pousser ces spécialistes de l'écriture que sont les étudiants et chercheurs en humanités à se tourner vers \LaTeX.

Telles sont les raisons qui ont poussé à la rédaction de ce livre, qui accompagnera --- nous l'espérons --- les humanistes depuis la découverte de \LaTeX, dans sa variante \XeLaTeX, jusqu'à la personnalisation de l'apparence des textes, en passant par la gestion d'une bibliographie nombreuse sans oublier les éditions critiques de textes et toutes ces petites choses qui font les difficultés et le charme de l'écriture en sciences humaines \ldots

D'autres raisons pousseront le lecteur à l'améliorer un jour, car en se partageant le savoir ne se divise pas : il se multiplie.

\vspace{4ex}

\scriptsize
\emph{
Maïeul Rouquette prépare un mémoire de master en science des religions à l'Université de Strasbourg.} 
\emph{Il est mainteneur des packages \emph{ledmac} et \emph{bibleref-french}.}


\emph{Brendan Chabanne et Enimie Rouquette, tous deux étudiants  de second et troisième cycle en lettres, l'ont accompagné.}

\normalsize

\vspace{4ex}
Ce livre est diffusé sous Licence \enquote{Creative Commons - Paternité - Partage des conditions initiales à l'identique}

\vspace{2ex}
 \raggedleft\includegraphics[height=3ex]{images/cc.png}
